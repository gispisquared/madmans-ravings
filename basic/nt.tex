\section{Number Theory}
\subsection{Divisibility}
  For integers $a$ and $b$, we say $a\mid b$ (read ``$a$ divides
  $b$'') if there is some integer $c$ with $b=a\times c$.
\begin{result}{\label{r:b:n:d:1}}
    If $a$ and $b$ are positive integers with $a\mid b$ then
      $a\le b$.
\end{result}
\begin{result}{\label{r:b:n:d:2}}
    If $a\mid b$ and $a\mid c$ then $a\mid bx+cy$ for all
      integers $x$ and $y$.
\end{result}
\begin{problem}{\label{p:b:n:d:1}}
    Find all integers $n$ such that $n^2+1\mid n^3+n^2-n-15$.
\end{problem}
\subsection{Primes}
  We define a \emph{prime} as a positive integer larger than 1 which is not
  divisible by any positive integer other than 1 and itself.
\begin{result}{\label{r:b:n:p:1}}
  Every positive integer larger than 1 can be written as a product of primes.
\end{result}
This was what allowed us to prove Result~\ref{r:b:m:2:1}.

The Fundamental Theorem of Arithmetic (proved in Results~\ref{r:b:n:p:1}
and~\ref{r:i:n:f:2})
states that each positive integer has a unique prime factorisation; that is,
we can write a positive integer $n$ uniquely (up to permuting the $p_i$s) as
\[n=p_1^{e_1}p_2^{e_2}\cdots p_k^{e_k},\]
where $p_i$ are all prime and $e_i$ are all positive integers.

Prime factorisations allow us to view statements about divisibility and
multiplication in terms of the exponents $e_i$.

In what follows, let 
\[a=p_1^{e_1}p_2^{e_2}\cdots p_m^{e_m},\
      b=q_1^{f_1}q_2^{f_2}\cdots q_k^{e_k}.\]
\begin{result}{\label{r:b:n:p:2}}
    $a\mid b$ if and only if for each $i$ we have that $p_i=q_j$
      for some $j$, and that $e_i\le f_j$.
\end{result}
\begin{result}{\label{r:b:n:p:3}}
    $a$ is a perfect $k$th power if and only if $k\mid e_i$ for all $i$.
\end{result}
\begin{result}{\label{r:b:n:d:3}}
    Let $m$ and $n$ be positive integers. Then, $\sqrt[m]n$ is an integer or
    irrational.
\end{result}
\begin{result}{\label{r:b:n:p:4}}
    The lcm is found by taking the maximum power of each prime that
    divides either $a$ or $b$; the gcd is found by taking the minimum power of
    each prime that divides both $a$ and $b$.
\end{result}
\begin{result}{\label{r:b:n:p:5}}
    $\gcd(a,b)\times\lcm(a,b)=ab$.
\end{result}
\subsection{Factorisations}
Factorisations allow us to make equations nicer to work with. For example,
solving the equation $xy+x+y=3$ over the integers becomes much easier when we
express it as $(x+1)(y+1)=4$.

Here are the most useful factorisations:
\begin{align*}
  axy+bx+cy=d\iff (ax+c)(ay+b)=ad+bc \\
  a^k-b^k=(a-b)\left(a^{k-1}+a^{k-2}b+\cdots+b^{k-1}\right)
\end{align*}
\begin{problem}{\label{p:b:n:f:1}}
     Find all right-angled triangles with positive integer sides 
      such that their area and perimeter are equal.
\end{problem}
\begin{problem}{\label{p:b:n:f:2}}
     Prove that $1^k+2^k+\cdots+n^k$ is divisible by $1+2+\cdots+n$ for
      all positive integers $n$ and odd positive integers $k$.
\end{problem}
\begin{problem}{\label{p:b:n:f:3}}
    Prove that if $2^n+1$ is prime for a positive integer $n$, then $n$ is
      a power of 2.
\end{problem}
\subsection{Number bases}\label{b:n:bases}
\begin{result}[Base $n$ representation]{\label{r:b:n:b:1}}
    Given positive integers $n>1$ and $k$, prove that there are unique
      nonnegative integers $m,a_0,a_1,\ldots,a_m$ such that $a_m>0,\ 0\le
      a_i<n$ for all $i$, and
      \[k=a_0 n^0+a_1 n^1+\cdots+a_m n^m.\]
\end{result}
This representation is often written as
\[(\overline{a_m a_{m-1}\cdots a_0})_n.\]
When the $a_i$s are specific digits, the parentheses and the bar over the $a_i$s
are often dropped. So for example, the decimal number 16 can be expressed as
$10000_2$.
\begin{problem}{\label{p:b:n:b:1}}
   If $234_{b+1}-234_{b-1}=70_{10}$, what is $234_b$ in base 10?
\end{problem}
\begin{problem}{\label{p:b:n:b:2}}
  A sequence $\{a_i\}$ begins with $a_1=0$, and for each $i$ the number
    $a_{i+1}$ is the smallest integer larger than $a_i$ which is not
    equal to $2a_k-a_j$ for any $j,k$ with $1\le j<k\le i$. So the sequence begins
    0,1,3,4,9,\ldots

    Find the 2000th term of this sequence.
\end{problem}
\subsection{Euclid's Algorithm}\label{b:n:euclid}
\begin{result}[Division Algorithm]{\label{r:b:n:e:1}}
    If $a$ is an integer and $b$ is a positive integer, there
      is a unique pair $(q,r)$ of integers such that $0\le r<b$ and $a=qb+r$.
\end{result}
\begin{result}[Euclid's Algorithm]{\label{r:b:n:e:2}}
    If $a=qb+r$, then $\gcd(a,b)=\gcd(b,r)$.
\end{result}
\begin{result}[Bezout's Identity]{\label{r:b:n:e:3}}
    There are integers $c$ and $d$ such that $ac+bd=\gcd(a,b)$.
\end{result}
\begin{problem}{\label{p:b:n:e:1}}
    The denominators of two irreducible fractions are $x$ and $y$. Find the
    minimum possible value of the denominator of their sum.
\end{problem}
\begin{result}[GCD Trick]{\label{r:b:n:e:4}}
    Prove that for all positive integers $a,b,m,n$ with $a>b$ and $\gcd(a,b)=1$ we have
    \[\gcd(a^m-b^m,a^n-b^n)=a^{\gcd(m,n)}-b^{\gcd(m,n)}.\]
\end{result}
\begin{problem}{\label{p:b:n:e:2}}
    Let $S$ be a nonempty set of integers such that if $a$ and $b$ are in $S$, then
      so is $2a-b$. Prove that $S$ is an arithmetic progression.
\end{problem}
