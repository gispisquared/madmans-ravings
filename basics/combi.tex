\section{Combinatorics}
Combinatorics (the branch of maths that deals with, among other things, finding
smart ways to count stuff) is a field with relatively few standard techniques.
Often, these are the problems that require the least technical skill and the
most ingenuity. With every field, but especially with combinatorics, there is no
substitute for practice.
\subsection{Addition and Multiplication Principles}
This is the most basic idea in combinatorics. If you can make one from $a$
choices and then one from $b$ choices, the total number of ways you can do this
is $ab$. If you can make either one from $a$ choices or one from $b$ choices,
the total number of ways you can do this is $a+b$.

Many combinatorics problems boil down to splitting them into cases, and
then applying addition and multiplication principles to count each case.
\begin{result}[Permutations]\label{r:b:c:am:1}
  The number of ways of choosing $k$ things from $n$, where order matters, is
  \[n(n-1)\cdots(n-k+1)=\frac{n!}{k!}.\]
  \hyperlink{sr:b:c:am:1}{Solution}
\end{result}
\begin{problem}\label{p:b:c:am:1}
  A hockey game between two teams is `relatively close' if the numbers of goals
  scored by the two teams never differ by more than two. In how many ways can
  the first 12 goals of a game be scored if the game is `relatively close'?
  \hyperlink{sp:b:c:am:1}{Solution}
\end{problem}
\begin{problem}\label{p:b:c:am:2}
  An ant's walk starts at the apex of a regular octahedron. It walks along five
  edges, never retracing its path. It visits each of the other five vertices
  exactly once. In how many ways can it do this?
\end{problem}
\subsection{Combinations}
\begin{result}[Combinations]\label{r:b:c:c:1}
  The number of ways of choosing $k$ things from $n$, where order doesn't
  matter, is \[\frac{n!}{(n-k)!k!}.\]
  \hyperlink{sr:b:c:c:1}{Solution}
\end{result}
\begin{problem}{\label{p:b:c:c:1}}
  How many pairs $(a,b)$ of 3-digit palindromes are there with $a>b$ and
  $a-b$ also a 3-digit palindrome?
  \hyperlink{sp:b:c:c:1}{Solution}
\end{problem}
\subsection{Inclusion-Exclusion}
So you try to count something using the addition principle, but you realise
you've counted some things twice. Obviously, the way you fix it is you subtract
the stuff you doubled up. For more tedious problems this could also be
overcounted, but this can be continued and hopefully the stuff you've
overcounted gets easier to count each time.
\begin{problem}{\label{p:b:c:pie:1}}
  How many integers between 1 and 1000 inclusive are divisible by at least one
  of 4, 6 and 10?
\end{problem}
\subsection{Bijections}
A bijection is simply a way of counting one set of stuff by identifying each
element in the set with an element from another set that's easier to count.
We've already used a bijection in our solution to Problem~\ref{p:b:c:c:1}.

To come up with a bijection, it's often useful to list out the elements of each
set for some small cases, and pair them up in what looks like a natural way.
Often that natural way will generalise.

\begin{result}[Multinomial Coefficient]
  The number of ways of putting $n$ identical things into $k$ boxes is
  \[\binom{n+k-1}{k-1}.\]
\end{result}
\begin{problem}{\label{p:b:c:b:1}}
  How many paths are there from the bottom left to the top right of a $4\times
  7$ grid, if you can only go up or right?
\end{problem}
\subsection{Recurrences}
These are a useful way to solve problems where you're asked for the number
of ways of doing something that depends on $n$. To set up a recurrence, we make
a sequence where the $i$th term is the answer for $n=i$. Then, we find a way of
relating each terms to previous terms.

Perhaps an example or two will serve to clarify things:
\begin{problem}{\label{p:b:c:r:1}}
  How many ways are there of tiling a $2\times n$ grid with dominoes?
\end{problem}
\begin{problem}{\label{p:b:c:r:2}}
  How many sequences of 10 traffic lights (each green, yellow or red) are there
  such that a green light is always followed by a yellow light, while a red light
  is never followed by a red light?
\end{problem}
\subsection{Pigeonhole Principle}
The Pigeonhole Principle is a very simple but widely applicable idea. It is
often useful in problems which ask you to prove the existence of one or more
things from a set, but there's no obvious way to pick a ``special'' object from
the set.
\begin{result}[Pigeonhole Principle]{\label{r:b:c:p:1}}
  If $A_1,A_2,\ldots,A_n$ are a collection of sets whose union is $A$, and $A$
  has more than $nk$ elements, then some $A_i$ has more than $k$ elements.
\end{result}
An important special case is when $A$ has infinitely many elements; then some
$A_i$ has infinitely many elements.
\begin{problem}{\label{p:b:c:p:1}}
  Let $M$ be a set of positive integers, none of which is larger than 100. Prove
  that there are two disjoint subsets of $M$ with the same sum.
\end{problem}
\begin{problem}{\label{p:b:c:p:2}}
  A class has $20$ students. Any two of them have a common grandfather. Prove
  that there are $14$ students all of whom have a common grandfather.
\end{problem}
\begin{problem}{\label{p:b:c:p:3}}
  There are $101$ chess players who participated in several tournaments. There
  was no tournament in which all of them participated. Each pair of those
  players met exactly once during these tournaments. Each pair of players in a
  tournament meet exactly once in that tournament. Prove that one of them
  participated in at least $11$ tournaments.
\end{problem}
