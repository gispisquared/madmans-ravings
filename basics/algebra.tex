\section{Algebra}
\subsection{Factorisations}\label{b:a:factor}
I won't have any problems attached to these, but they tend to pop up everywhere
so keep an eye out. Here are some common factorisations:
\begin{itemize}
  \item $x^2-a^2=(x+a)(x-a)$
  \item $x^2-2ax+a^2=(x-a)^2$
  \item $x^2+2ax+a^2=(x+a)^2$
  \item $x^3-a^3=(x-a)(x^2+ax+a^2)$
  \item $x^3+a^3=(x+a)(x^2-ax+a^2)$
  \item $x^3+3x^2a+3xa^2+a^3=(x+a)^3$
  \item $x^3-3x^2a+3xa^2-a^3=(x-a)^3$
\end{itemize}
Many of these are special cases of the formula
\[x^n-a^n=(x-a)(x^{n-1}+ax^{n-2}+a^2x^{n-3}+\cdots+a^{n-1}).\]

The cases $a=1$ are especially common.
\subsection{Systems of equations}
There are a couple of ways of solving these systems --- either you can
isolate one variable, substitute into the rest of the equations, and repeat, or
you can try and combine the equations in such a way that stuff cancels. The
first method is usually fine in school maths and the AMC, but the second is more
likely to be useful in harder Olympiad questions.

Sometimes these techniques won't be enough --- see Section~\ref{n:a:systems}.
\begin{problem}\label{p:b:a:systems:1}
  The difference between two numbers is 20. When 4 is added to each number the
  larger is three times the smaller. What is the larger of the two original
  numbers?
  \hyperlink{s:b:a:systems:1}{Solution}
\end{problem}
\begin{problem}\label{p:b:a:systems:2}
  Find all triples $(x,y,z)$ of real numbers that simultaneously satisfy the
  equations
  \begin{align*}
    xy+1&=2z \\
    yz+1&=2x \\
    zx+1&=2y \\
  \end{align*}
  \hyperlink{s:b:a:systems:2}{Solution}
\end{problem}
\subsection{Quadratics}
\begin{result}\label{r:b:a:quad:1}
  Let $m$ and $p$ be given real numbers. All real numbers $x$ such that
  \[x^2-2mx+p=0\] are given by $x=m\pm\sqrt{m^2-p}$.
  \hyperlink{sr:b:a:quad:1}{Solution}
\end{result}
\begin{result}\label{r:b:a:quad:2}
  Let $a,\ b,\ c$ be given real numbers. All real numbers $x$ such that
  \[ax^2+bx+c=0\] are given by \[x=\frac{-b\pm\sqrt{b^2-4ac}}{2a}.\]
  In particular, if we let $\Delta=b^2-4ac$, then the equation has no real roots
  if $\Delta<0$, exactly one real root if $\Delta=0$, and two real roots if
  $\Delta>0$.
  \hyperlink{sr:b:a:quad:2}{Solution}
\end{result}
This number $\Delta$ is called the \emph{discriminant} of the quadratic.

Now, a couple of problems which show how useful both the results and the method
are.
\begin{problem}\label{p:b:a:quad:1}
  [See Section~\ref{b:n:bases} if you don't know what number bases are.]

  The number $x$ is 111 when written in base $b$, but it is 212 when written in
  base $b-2$. What is $x$ is base 10?
  \hyperlink{sp:b:a:quad:1}{Solution}
\end{problem}
\begin{problem}\label{p:b:a:quad:2}
  For each pair of real numbers $(r,s)$, prove that there exists a real numer
  $x$ that satisfies at least one of the following two equations.
  \begin{align*}
    x^2+(r+1)x+s&=0 \\
    rx^2+2sx+s&=0
  \end{align*}
  \hyperlink{sp:b:a:quad:2}{Solution}
\end{problem}
\begin{problem}\label{p:b:a:quad:3}
  Find all real numbers $x$ for which $x^3+3x^2+3x+5=0$.
  \hyperlink{sp:b:a:quad:3}{Solution}
\end{problem}
\subsection{Inequalities}
At this level, inequalities are mostly about making stuff into squares or, well,
``mostly squares''. The guiding principle is to try and find an expression which
you want to be always nonnegative, figure out where it's 0, and write it in
terms of stuff that's 0 there and obviously nonnegative elsewhere.

\begin{result}\label{r:b:a:ineq:1}
  If $a$ and $b$ are real numbers, then \[\frac{a+b}2\ge\sqrt{ab}.\]
  \hyperlink{sr:b:a:ineq:1}{Solution}
\end{result}

\begin{problem}\label{p:b:a:ineq:1}
  The set $S$ consists of distinct integers such that the smallest is 0 and the
  largest is 2015. What is the minimum possible average value of the numbers in
  $S$?
  \hyperlink{sp:b:a:ineq:1}{Solution}
\end{problem}
\subsection{Sums of sequences}
\begin{result}\label{r:b:a:sums:1}
  If $n$ is a positive integer and $a$ and $b$ are real numbers, then
  \[\sum_{i=0}^n(a+bi)=\frac{(n+1)(2a+bn)}2.\]
  \hyperlink{sr:b:a:sums:1}{Solution}
\end{result}
\begin{result}\label{r:b:a:sums:2}
  If $n$ is a positive integer and $r$ is a real number distinct from 1, then
  \[\sum_{i=0}^n r^i=\frac{1-r^{n+1}}{1-r}.\]
  \hyperlink{sr:b:a:sums:2}{Solution}
\end{result}
\begin{problem}\label{p:b:a:sums:1}
  Prove that for any positive integers $n$ and $a$, the sum
  \[n+(n+1)+(n+2)+\cdots+(n+a)\] is never a power of 2.
  \hyperlink{sp:b:a:sums:1}{Solution}
\end{problem}
