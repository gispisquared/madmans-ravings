\section{Methods of Proof}
If you haven't seen proofs before, chapters 4, 5, 6, and 10 of
\href{https://www.people.vcu.edu/~rhammack/BookOfProof/Main.pdf}{The Book of
  Proof} provide a gentler and more complete introduction.
\subsection{Direct Proof}
This is perhaps the simplest type of proof. The idea is to start with the stuff
you're given, do some logical deduction, and finish with what you want to prove.
\begin{result}\label{r:b:m:1:1}
  The remainder when a perfect square is divided by 4 is either 0 or 1.
  \hyperlink{s:m:1:1}{Solution}
\end{result}
\subsection{Contradiction}
This is where you assume that what you're trying to prove is wrong and try to
derive some kind of logical impossibility. Then the only place where the logic
could have gone wrong was in the assumption so the statement you were trying to
prove must be true.
\begin{result}[Infinitude of primes]\label{r:b:m:2:1}
  There are infinitely many primes.
  \hyperlink{s:m:2:1}{Solution}
\end{result}
\subsection{Contrapositive}
It turns out that the statement $A\implies B$ is logically equivalent to the
statement $\neg B\implies\neg A$. This is probably easiest to see intuitively
with an example: ``If $x$ is an integer, then $x$ is rational'' is
logically equivalent to ``If $x$ is not rational, then $x$ is not an integer''.
Therefore, if we're asked to prove $A\implies B$, it's enough to
prove $\neg B\implies\neg A$, which is sometimes easier.
\begin{problem}\label{p:b:m:3:1}
  Let $a,b\in\Rr$ such that $a+b$ is irrational. Prove that at least one of $a$
  and $b$ is irrational.
  \hyperlink{s:m:3:1}{Solution}
\end{problem}
\subsection{Induction}
Perhaps the hardest to understand of the basic proof techniques, this can be
used to prove properties of positive integers where the property for each
integer can be related to those of previous integers.

Here is the Principle of Mathematical Induction (PMI):
\begin{tcolorbox}
  Let $S$ be a set of positive integers such that $1\in S$ and for each $k\in S$,
  $k+1\in S$. Then $S$ contains all positive integers.
\end{tcolorbox}

To prove a statement for all positive integers, we let $S$ be the set of all
positive integers for which the statement is true. Then it's enough to prove:
\begin{itemize}
  \item $1\in S$. This is called the \emph{base case}.
  \item If $k\in S$ (the \emph{inductive hypothesis}), then $k+1\in S$. This is
    called the \emph{inductive case}.
\end{itemize}
Then by PMI, $S$ will contain all positive integers.

There are two ways to make induction superficially more powerful, though they're
both equivalent to the usual form of induction:
\begin{itemize}
  \item Say we want to prove a statement for all integers larger than $n$, for
    some $n$. Then it's enough to prove:
    \begin{itemize}
      \item The statement is true for $n+1$.
      \item If the statement is true for some integer $k>n$, then it's true for
        $k+1$.
    \end{itemize}
    This is equivalent to the normal PMI\@: to see this, let $S$ be the set of
    all integers $m$ for which the statement is true for all $m+n$.
  \item Say we want to use not just the inductive assumption not just for $k$,
    but for smaller integers as well. Intuitively this should be fine, since
    we've in some sense ``proved this already'' by the time we get to $k+1$.
    Formally, to prove a statement $P(n)$ for all positive integers $n$, it's
    enough to prove:
    \begin{itemize}
      \item $P(1)$.
      \item If $P(1),\ldots,P(k)$ are all true, then $P(k+1)$ is also true.
    \end{itemize}
    Once again this is equivalent to the normal PMI\@: let $S$ be the set of all
    integers $m$ for which $P(a)$ is true for all $a\le m$.

    This form of proof by induction is called \emph{strong induction}, and
    although most proofs by induction only explicitly use $P(k)$, there's no
    reason to try to make your proof inductive over strong inductive since
    strong induction gives you more assumptions to work with ``for free''.
\end{itemize}
The key idea in both of these reductions to PMI is to somehow encapsulate the
extra information you're trying to assume into the framework of standard PMI\@.
\begin{result}\label{r:b:m:4:1}
  For all positive integers $n$, \[1+2+\cdots+n=\frac{n(n+1)}2.\]
  \hyperlink{s:m:4:1}{Solution}
\end{result}

There's also an equivalent statement to PMI --- the Well-Ordering Principle.
\begin{tcolorbox}
  Let $S$ be a nonempty set of postiive integers. Then there exists some $x\in
  S$ such that for all $y\in S$ we have $y\ge x$.
\end{tcolorbox}
The Well-Ordering Principle is often used in conjunction with contradiction or
contrapositive, since it is mainly ``about'' the existence of some element
rather than the properties of the rest of the elements.

We can think of PMI as enabling us to prove something is true by building up
larger cases from smaller cases, while well-ordering allows us to prove
something is true by assuming there's a counterexample and finding a smaller
one. It's nontrivial and instructive to prove they're actually
equivalent.

First, we'll need a lemma (auxiliary result) so basic this is probably the only
time we'll ever need to justify it:
\begin{result}\label{r:b:m:4:2}
  Both PMI and well-ordering imply that if $n$ is a positive integer then $n\ge
  1$.
  \hyperlink{s:m:4:2}{Solution}
\end{result}
\begin{result}\label{r:b:m:4:3}
  PMI and well-ordering imply each other.
  \hyperlink{s:m:4:3}{Solution}
\end{result}
I find it intriguing that induction and minimality are really just two sides of
the same coin. Often you will find that a solution is much more natural to think
about and write up in terms of one than the other.
