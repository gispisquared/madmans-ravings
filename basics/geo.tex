\section{Geometry}
Whenever you see a geometrical statement or problem (including everywhere in the
geometry sections of this book), your first instinct should
always be to draw a diagram to understand it. Diagrams should be
\begin{itemize}
  \item Accurate --- use ruler and compass
  \item Large --- they should take up the whole page. Go landscape if it gives
    you more space.
\end{itemize}
This is important because your chances of solving, or even understanding, a
problem are proportional to how easily you can see things in your diagram.
\subsection{Parallels}
You should know what parallel lines are: they point in the same direction and
never meet.

Let $ABC$ and $DEF$ be points on two parallel lines (in that order, and in the
same direction along each line), and let
$X$ and $Y$ be points on $BE$ such that $XBEY$ is in that order.
Then
\[\angle ABX=\angle CBE=\angle DEX=\angle FEY,\] the other four angles are
equal, and the angles in these two sets are supplementary (add up to 
$180^\circ$).
\begin{result}{\label{r:b:g:p:1}}
  For any three points $A, B, C$, \[\angle ABC+\angle
  BCA+\angle CAB=180^\circ.\]
\end{result}
\subsection{Congruence}
Two triangles $ABC$ and $XYZ$ are called \emph{congruent} if \[AB=XY,\ AC=XZ,\
BC=YZ,\]
\[\angle BAC=\angle YXZ,\ \angle ABC=\angle XYZ,\ \angle ACB=\angle
XZY.\]
We write ``$ABC$ is congruent to $XYZ$'' as $\triangle ABC\cong\triangle XYZ$.

There are congruence tests which enable us to determine when two triangles are
congruent, so that knowing some of these equalities we can deduce the others.

Triangles $ABC$ and $XYZ$ are congruent if any of the following hold:
\begin{itemize}
  \item $AB=XY$, $AC=XZ$ and $BC=YZ$. (SSS)
  \item $AB=XY$, $AC=XZ$ and $\angle BAC=\angle YXZ$. (SAS)
  \item $AB=XY$ and $\angle BAC=\angle YXZ$ and $\angle ABC=\angle XYZ$. (AAS)
\end{itemize}
Note: SSA (two sides and an unincluded angle) on its own isn't enough, but we
can fix it:
\begin{itemize}
  \item $AB=XY$ and $AC=XZ$ and $\angle ABC=\angle XYZ$ and $AB<AC$. (Fixed SSA)
\end{itemize}
In the special case where $\angle ABC=\angle XYZ=90^\circ$, this is known as RHS.\@
\begin{result}{\label{r:b:g:c:1}}
  Let $ABC$ be a triangle, and let $M$ be the midpoint of $BC$.
  The following statements are equivalent:
  \begin{itemize}
    \item $AB=AC$
    \item $\angle ABC=\angle ACB$
    \item $AM\perp BC$.
  \end{itemize}
\end{result}
If any of those statements are true the triangle is called \emph{isosceles}.
\begin{problem}
  Let $ABC$ be an isosceles triangle with $AB=BC$. Let $D$ be a point on $BC$
  such that $\angle DBC=20^\circ$. Let $E$ be a point on $AB$ with $AE=AD$. What
  is $\angle BDE$?
\end{problem}
\begin{result}{\label{r:b:g:c:2}}
  Let $ABCD$ be a parallelogram (that is, $AB||CD$ and $BC||DA$). If $AC$
  intersects $BD$ at $P$, then $P$ is the midpoint of $AC$ and of $BD$.
\end{result}
\subsection{Circles}
A \emph{circle} is the set of points a distance $r$ from a central point $O$.
Thus, any two points on the circle form an isosceles triangle with $O$.

The circumference (perimeter) of the circle is $2\pi r$, while the area is $\pi
r^2$.
\begin{result}{\label{r:b:g:ci:1}}
  Let $A,\ B,\ C,\ M$ be points in the plane such that $M$ is the midpoint of
  $AB$. Then $C$ lies on the circle centred at $M$ passing through $A$ and $B$
  if and only if $\angle ACB=90^\circ$.
\end{result}

The next few results are generalisations of this. The proofs are linked in
Section~\ref{n:g:directed}.
\begin{result}{\label{r:b:g:ci:2}}
  If $A,\ B,\ C$ are points on a circle centred at $O$ with $A$ on the same side
  of $BC$ as $O$, then $\angle BOC=2\angle BAC$.
\end{result}
\begin{result}{\label{r:b:g:ci:3}}
  If $A,\ B,\ C$ are points on a circle centred at $O$ with $A$ and $O$ on
  opposite sides of $BC$, then $\angle BOC=360^\circ-2\angle BAC$.
\end{result}
\begin{result}{\label{r:b:g:ci:4}}
  If $B$ and $C$ lie on the same side of $AD$ then there is a circle passing
  through all of $A,\ B,\ C,\ D$ if and only if $\angle ABD=\angle ACD$.
\end{result}
\begin{result}{\label{r:b:g:ci:5}}
  If $B$ and $D$ lie on opposite sides of $AC$, then there is a circle passing
  through all of $A,\ B,\ C,\ D$ if and only if $\angle ABC+\angle
  ADC=180^\circ$.
\end{result}
\begin{problem}{\label{p:b:g:ci:1}}
  For $n\ge 3$, a pattern can be made by overlapping $n$ circles, each of
  circumference 1 unit, so that each circle passes through a central point and
  the resulting pattern has order-$n$ rotational symmetry. For instance, the
  diagram shows the pattern where $n=7$.

  If the total length of visible arcs is 60 units, what is $n$?
\end{problem}
\subsection{Similarity}
  We say that triangles $ABC$ and $XYZ$ are \emph{similar} if for some real $r$
  (called the ratio of similitude),
      \[\frac{BC}{YZ}=\frac{CA}{ZX}=\frac{AB}{XY}=r,\]
      \[\angle ABC=\angle XYZ,\
      \angle BCA=\angle YZX,\ \angle CAB=\angle ZXY.\]

    We write ``$ABC$ is similar to $XYZ$'' as $\triangle ABC\sim\triangle XYZ$.

    Triangles $ABC$ and $XYZ$ are similar if any of the following hold:
  \begin{itemize}
    \item $\frac{BC}{YZ}=\frac{CA}{ZX}=\frac{AB}{XY}$ (PPP)
    \item $\frac{AB}{XY}=\frac{BC}{YZ},\ \angle ABC=\angle XYZ$ (PAP)
    \item $\angle ABC=\angle XYZ,\ \angle BCA=\angle YZX$ (AA)
  \end{itemize}
  Once again, PPA (two sides and an unincluded angle) doesn't work. However, we
  can fix it in the same way:
  \begin{itemize}
    \item $\frac{AB}{XY}=\frac{AC}{XZ},\ \angle ABC=\angle XYZ$ and $AB<AC$.
      (Fixed PPA)
  \end{itemize}
  In the special case where $\angle ABC=\angle XYZ=90^\circ$, this is
  known as RHS\@.
\begin{result}{\label{r:b:g:s:1}}
  Let $ABC$ and $ADE$ be similar triangles with the same orientation (that is,
  $\triangle ABC\sim\triangle ADE$ and both triangles are labelled either
  clockwise or anticlockwise). Then $ABD$ and $ACE$ are also similar and
  similarly oriented.
\end{result}
\begin{problem}{\label{p:b:g:s:1}}
  Let $A,B,C,D$ be points on a circle such that $\triangle ABC$ is equilateral,
  and $D$ lies on minor arc $BC$. Prove that $AD=BD+CD$.
\end{problem}
\subsection{Areas}
We assign a positive real number, known as an \emph{area}, to each polygon in
the plane such that
\begin{itemize}
  \item the area of a rectangle with sidelengths $a$ and $b$ is $ab$;
  \item the areas of two congruent triangles are equal; and
  \item if two polygons have disjoint interiors then the area of their union
    equals the sum of their areas.
\end{itemize}
The area of a polygon $P_1,P_2,\ldots,P_n$ is denoted $|P_1P_2\cdots P_n|$. We
may use these properties to deduce some well-known facts about areas.
\begin{result}{\label{r:b:g:a:1}}
  Let $ABC$ be a triangle such that $\angle ABC=90^\circ$. Then,
  \[|ABC|=\frac 12\times AB\times BC.\]
\end{result}
\begin{result}{\label{r:b:g:a:2}}
  Let $ABC$ be a triangle, and let $D$ be a point on line $BC$ such that
  $AD\perp BC$. Then,
  \[|ABC|=\frac 12\times AD\times BC.\]
\end{result}
An important special case of this is that two triangles with the same height
have areas in the same ratio as their bases, and two triangles with the same
base have areas in the same ratio as their heights.
\begin{result}{\label{r:b:g:a:3}}
  Let $ABC$ and $XYZ$ be similar triangles with ratio of similitude $r$. Then,
  \[\frac{|ABC|}{|XYZ|}=r^2.\]
\end{result}
\begin{problem}{\label{p:b:g:a:1}}
  A triangle $ABC$ is divided into four regions by three lines parallel to $BC$.
  The lines divide $AB$ into four equal segments. If the second largest region
  has area 225, what is the area of $ABC$?
\end{problem}
\begin{problem}{\label{p:b:g:a:2}}
  Let $ABCD$ be a parallelogram. Point $P$ is on $AB$ produced such that $DP$
  bisects $BC$ at $X$. Point $Q$ is on $BA$ produced such that $CQ$ bisects $AD$
  at $M$. Lines $DP$ and $CQ$ meet at $O$. If the area of parallelogram $ABCD$
  is 192, find the area of triangle $POQ$.
\end{problem}
\begin{problem}{\label{p:b:g:a:3}}
  The area of triangle $ABC$ is 300. In triangle $ABC$, $Q$ is the midpoint of
  $BC$, $P$ is a point on $AC$ between $C$ and $A$ such that $CP=3PA$, $R$ is a
  point on side $AB$ such that the area of $\triangle PQR$ is twice the area of
  $\triangle RBQ$. Find the area of $\triangle PQR$.
\end{problem}
\begin{result}{\label{r:b:g:a:4}}
  In triangle $ABC$, point $P$ is on $AB$ such that $AP$ bisects $\angle BPC$.
  Then, $\frac{BP}{PC}=\frac{BA}{AC}$.
\end{result}
\subsection{Pythagoras}
Pythagoras' Theorem is arguably the most famous theorem in mathematics. You
should aim to find multiple proofs of it, to cement your understanding of the
techniques developed in this section on geometry.

Four proofs are given in the linked solutions, with a fifth in
Section~\ref{n:g:incircle}.
\begin{result}{\label{r:b:g:py:1}}
    If $\angle ABC=90^\circ$, then $AB^2+BC^2=AC^2$.
\end{result}
\begin{problem}{\label{p:b:g:py:1}}
  Let $ABCD$ be a square and let $E$ and $F$ be points on $BC$ and $CD$,
  respectively, such that $AEF$ is an equilateral triangle.

  Find the length $BE$.
\end{problem}
\subsection{Trigonometry}
These concepts are easiest to define working on the coordinate plane. Let $O$ be
the origin, and let $A$ be the point $(1,0)$. 
Let $P=(x,y)$ be a point on the unit circle (that is, $x^2+y^2=1$) such that
the counterclockwise angle $\angle AOP$ is $\theta$ (see the diagram). We define
\[\cos(\theta)=x,\ \sin(\theta)=y,\ \tan(\theta)=\frac yx.\]
\begin{result}{\label{r:b:g:t:1}}
  For all $\theta$ we have
  $\cos(-\theta)=\cos(\theta)=\sin(90-\theta)=-\cos(180-\theta)$.
\end{result}
\begin{result}{\label{r:b:g:t:2}}
  Let $ABC$ be a triangle with $\angle ABC=90^\circ$. Then,
  \[\sin(\angle ACB)=\frac{AB}{AC},\ \cos(\angle ACB)=\frac{BC}{AC},\
    \tan(\angle ACB)=\frac{AB}{BC}.\]
\end{result}
\begin{result}{\label{r:b:g:t:3}}
  We have
  \[\sin(0)=0,\ \sin(30)=\frac 12,\ \sin(45)=\frac{\sqrt2}2,\
    \sin(60)=\frac{\sqrt3}2,\ \sin(90)=1.\]
\end{result}
For the next two results we define $ABC$ to be a triangle with sidelengths
$a=BC,\ b=CA,\ c=AB$. We use the shorthand $\angle A=\angle BAC$ and similarly
for $\angle B$ and $\angle C$.
\begin{result}{\label{r:b:g:t:4}}
  \[\frac a{\sin \angle A}=\frac b{\sin\angle B}=\frac c{\sin\angle C}.\]
  Further, if $A,B,C$ lie on a circle with radius $R$ then these quantities are
  all equal to $2R$.
\end{result}
\begin{result}{\label{r:b:g:t:5}}
  \[\cos\angle A=\frac{b^2+c^2-a^2}{2bc}.\]
\end{result}
\begin{problem}{\label{p:b:g:t:1}}
 In quadrilateral $PQRS$ we have $PS=5,\ SR=6,\ RQ=4$, and $\angle P=\angle
 Q=60^\circ$. Find the length of $PQ$.
\end{problem}
\begin{problem}{\label{p:b:g:t:2}}
  Let $ABCD$ be a trapezium with $AB\|CD$ such that its vertices $A,B,C,D$ lie
  on a circle with centre $O$. Let the diagonals $AC$ and $BD$ intersect at a
  point $M$. Assume that $\angle AMD=60^\circ$ and $MO=1$.

  What is the difference between the lengths $AB$ and $CD$?
\end{problem}
