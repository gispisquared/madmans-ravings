\subsection{Arithmetic functions}
A function $f:\mathbb N\to\mathbb R$ is called multiplicative if for any
coprime positive integers $a$ and $b$, we have
\[f(a)f(b)=f(ab).\]
It's called completely multiplicative if this equation holds for \emph{any}
positive integers $a$ and $b$.
\begin{itemize}
\item Prove that the values at the primes of a completely multiplicative
  function completely define the function (unless these values are all 0, in
  which case $f(1)$ can be 0 or 1).
\item Prove that the values at prime powers of a multiplicative function
  completely define it. 
\item Find a formula for the number of solutions $1\le x\le n$ of $n\mid x^2-1$.
\end{itemize}
Problems:
\begin{itemize}
    \item Prove that $\sigma(n)<n\sqrt{2d(n)}$ for all positive integers $n$.
    \item Find all completely multiplicative functions $f:\mathbb N\to\mathbb N$
    such that for all $a,b\in\mathbb N$, at least two of $f(a),f(b),f(a+b)$ are
    equal.
\end{itemize}
