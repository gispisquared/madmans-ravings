\subsection{Quadratic Residues}
If $x$ can be written as $y^2$ for $y\in\Zz_n^*$, then we say that $x$ is a
\emph{quadratic residue (QR)} mod $n$. Note that $0$ is not a QR mod $n$.

Prove that $x$ is a QR mod $n$ iff both
\begin{itemize}
  \item $x\equiv 1\pmod{\gcd(8,n)}$, and
  \item for each odd prime $p\mid n$, $x$ is a QR mod $p$.
\end{itemize}

Hence, we now restrict ourselves to considering QRs mod $p$.
Define the \emph{Jacobi symbol}
\[\left(\frac ap\right)=\begin{cases} 0&p\mid a \\ 1&a\text{ is a QR mod }p \\
-1&\text{otherwise}\end{cases}.\]

\begin{itemize}
  \item (Euler's criterion) Prove that
    $\left(\frac ap\right)\equiv a^{\frac{p-1}2}\pmod p$.
    Hence, $\left(\frac ap\right)\left(\frac
    bp\right)=\left(\frac{ab}p\right)$.
  \item (Gauss' Lemma) Let $a\in\Zz_p^*$, and let $S\subseteq\Zz_p^*$ such that $x\in S\iff
    -x\not\in S$. Let $T=\{ay:y\in S\}$. Then
    $\left(\frac ap\right)=(-1)^{|T\setminus S|}$.
  \item Find $\left(\frac 2p\right)$ and $\left(\frac{-1}p\right)$.
\end{itemize}

Finally, there is quadratic reciprocity.
Let $p$ and $q$ be distinct odd primes. Then,
\[\left(\frac qp\right)\left(\frac pq\right)=(-1)^{\frac{(p-1)(q-1)}4}.\]
Problems:
\begin{itemize}
  \item Let $p$ be an odd prime and let $a$ be an integer with $\gcd(a,p)=1$.
    Prove that
    \[\sum_{n=1}^p\left(\frac{n^2+a}p\right)=-1.\]
  \item Prove that for any prime $p$ and positive integer $a$ with $p\nmid a$
    there are at least $p-1$ solutions in $\Zz_p$ to $x^2+y^2\equiv a\pmod p$.
  \item If $p>3$ is a prime such that $\varphi(p-1)>\frac{p-1}3$, prove that
    there are two consecutive generators mod $p$.
\end{itemize}
