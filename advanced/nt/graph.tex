\subsection{Build a graph}
\begin{itemize}
    \item Fifty numbers are chosen from the set $\{1,2,\ldots,99\}$, no two of
        which sum to $99$ or $100$. Prove that the numbers must be
        $50,51,\ldots,99$.
    \item Let $p$ be a prime, and let $a_1, \dots, a_p$ be integers. Show that
        there exists an integer $k$ such that the numbers
        \[a_1 + k, a_2 + 2k, \dots, a_p + pk\]produce at least $\tfrac{1}{2} p$
        distinct remainders upon division by $p$.
    \item An international society has its members from six different
        countries. The list of members has 1978 names, numbered $1, 2,
        \ldots, 1978$. Prove that there is at least one member whose
        number is the sum of the numbers of two (not necessarily
    distinct) members from his own country. 
\item A set of positive integers is called \emph{fragrant} if it contains at least two
    elements and each of its elements has a prime factor in common with at least
    one of the other elements. Let $P(n)=n^2+n+1$. What is the least possible
    positive integer value of $b$ such that there exists a non-negative integer
    $a$ for which the set $\{P(a+1),P(a+2),\ldots,P(a+b)\}$ is fragrant?
    \item The Fibonacci numbers $F_0, F_1, F_2, . . .$ are defined inductively
        by $F_0=0, F_1=1$, and $F_{n+1}=F_n+F_{n-1}$ for $n \ge 1$. Given an
        integer $n \ge 2$, determine the smallest size of a set $S$ of integers
        such that for every $k=2, 3, \ldots , n$ there exist some $x, y \in S$
        such that $x-y=F_k$.
    \item  There are $4n$ pebbles of weights $1, 2, 3, \dots, 4n.$ Each pebble
        is coloured in one of $n$ colours and there are four pebbles of each
        colour. Show that we can arrange the pebbles into two piles so that the
        following two conditions are both satisfied:

        \begin{itemize}
            \item 
            The total weights of both piles are the same.
            \item
                Each pile contains two pebbles of each colour.
        \end{itemize}
    \item Let $k, m, n$ be integers satisfying $1 < n \le m-1 \le k$. Determine 
        the maximum size of a subset $S$ of the set $\{1, 2, \ldots, k\}$ such that 
        no $n$ distinct elements of $S$ add up to $m$.
    \item Let n be an even positive integer. Show that there is a permutation 
        $(x_1, x_2, \ldots, x_n)$ of $(1, 2, \ldots, n)$ such that for every $i \in (1, 2, \ldots,
        n)$,
        the 
        number $x_{i+1}$ is one of the numbers $2x_i, 2x_i-1, 2x_i-n, 2x_i-n-1$. Here 
        we use the cyclic subscript convention, so that $x_{n+1}$ means $x_1$.  
\end{itemize}
