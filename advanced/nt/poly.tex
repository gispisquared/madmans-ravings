\subsection{Polynomials}
\begin{result}[Hensel's Lemma]{\label{r:a:n:p:1}}
Let $P$ be an integer polynomial.
Let $r$ be an integer such that $P(r)\equiv 0\pmod n$ but $\gcd(P'(r),n)=1$.
For any positive integer $m$, there is a unique $s$ mod $n^m$ such
that $s\equiv r\pmod n$ and $P(s)\equiv 0\pmod {n^m}$.
\end{result}
\begin{itemize}
  \item Let $p$ be prime and let $a$ and $b$ be positive integers such that
    $p\nmid b$. Prove that there exists a positive integer $n$ such that
    $p^a\mid n^n-b$.
  \item Let $P$ be a nonconstant polynomial with integer coefficients. Prove that for any
    integer $m$ there exist an integer $n$ and a prime $p$ such that $p^m\mid
    P(n)$.
\end{itemize}
An integer polynomial is \emph{primitive} if its coefficients have gcd $1$.
\begin{itemize}
  \item Every nonzero rational polynomial has exactly one primitive multiple
    with positive leading coefficient.
  \item (Gauss' Lemma) The product of two primitive polynomials is primitive.
  \item (Gauss' Lemma, alternate form) If an integer polynomial is the product
    of two nonconstant rational polynomials then it is the product of two
    nonconstant integer polynomials.
\end{itemize}
An integer polynomial is \emph{irreducible} over $\mathbb Z$ (for the rest of
this handout, I'll shorten this to irreducible) if it is not the product of two
nonconstant integer polynomials.

Usually, to prove irreducibility you will assume for contradiction that the
polynomial is reducible. Modular arithmetic arguments on the coefficients are
often useful.
\begin{itemize}
  \item (Unique factorisation) Every integer polynomial can be factorised into a
    product of a constant and primitive irreducible integer
    polynomials. This factorisation is unique up to the permutation and sign of
    these polynomials.
  \item (Eisenstein's Criterion) If there exists a prime $q$ such that $q^2\nmid
    a_0$, $q\mid a_i$ for each $i$ from $0$ to $n-1$, and $q\nmid a_n$, then $p$
    is irreducible.
  \item If an integer polynomial is irreducible mod $n$ for any positive integer $n$,
      then it is irreducible over $\Zz$.
\end{itemize}
Let $p$ be prime.
\begin{itemize}
  \item Prove that unique factorisation holds for polynomials mod $p$. (This is
    not true for all integers --- for instance,
    $(x-1)^2\equiv(x-3)^2\pmod 4$.)
  \item Prove that for every function $f:\Zz_p\to\Zz_p$ there is a unique polynomial $P$ in
    $\Zz_p$ of degree less than $p-1$ such that $f(x)=P(x)$ for each
    $x\in\Zz_p$.
  \item Let $g$ be a generator mod $p$, and let $ab=p-1$. Prove that
    \[\prod_{i=1}^a (x-g^{bi})\equiv x^a-1\pmod p.\]
    What does this tell us about the roots of the cyclotomic polynomials in mod
    $p$?
  \item Consider all $\binom{p-1}k$ products of $k$ elements of $\Zz_p$. Prove
    that their sum is divisible by $p$.
  \item For any positive integer $n<p-1$, prove that
    \[\sum_{i=1}^{p-1} i^n\equiv 0\pmod p.\]
\end{itemize}
Problems:
\begin{itemize}
  \item Prove that if $p$ is prime, then $1+x+x^2+\cdots+x^{p-1}$ is
    irreducible.
  \item Find all integer polynomials $p$ such that
    \begin{itemize}
      \item $p(n)>n$ for all positive integers $n$, and
      \item for each positive integer $n$ there is a positive integer $k$ such
        that $p^{(k)}(1)$ ($p$ repeated $k$ times) is divisible by $n$.
    \end{itemize}
  \item Let $p$ be prime. Find the least residue of the product of $(4-x)$ mod $p$, where $x$ runs
    over all residues mod $p$ except the quadratic residues.
\end{itemize}
