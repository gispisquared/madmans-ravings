\subsection{Largest exponent notation}
  Let $p$ be a prime and let $r$ be a rational number. We let $\nu_p(r)$ be the
  exponent of $p$ in the prime factorisation of $r$.
\begin{itemize}
  \item $\nu_p(a+b)\ge\min(\nu_p(a),\nu_p(b))$.
  \item If $\nu_p(a+b)>\min(\nu_p(a),\nu_p(b))$ then $\nu_p(a)=\nu_p(b)$.
  \item If $a$ is a positive integer, then $\nu_p(a)\le\log_p(a)$.
\end{itemize}
\begin{itemize}
  \item Legendre's Formula:
    \[\nu_p(n!)=\sum_{i=1}^\infty\left\lfloor\frac
      n{p^i}\right\rfloor=\frac{n-s_p(n)}{p-1}<\frac
    n{p-1}.\]
\item Let $p$ be an odd prime, and let $a$ and $b$ be integers such that $p\mid
    a-b$ but $p\nmid a$. Let $k$ be a positive integer.
    Then, $\nu_p\left(a^k-b^k\right)=\nu_p(a-b)+\nu_p(k)$.

\item Let $a$ and $b$ be odd integers, and let $k$ be a positive integer.
    If $k$ is even, we have $\nu_2(a^k-b^k)=\nu_2(a-b)+\nu_2(a+b)+\nu_2(k)-1$.
    If $k$ is odd, we have $\nu_2(a^k-b^k)=\nu_2(a-b)$.
\end{itemize}
\begin{itemize}
  \item Prove that for all positive integers $n$,
    \[\binom{2n}n\mid\lcm(1,2,\ldots,2n).\]
  \item Let $a,b,c$ be positive integers such that $c\mid a^c-b^c$. Prove that
    $c(a-b)\mid a^c-b^c$.
  \item Find all positive integers $n$ and $k$ such that
      \[k!=(2^n-1)(2^n-2)(2^n-4)\cdots(2^n-2^{n-1}).\]
\end{itemize}
