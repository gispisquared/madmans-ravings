% chktex-file 17
\subsection{Existence proofs}
Apart from density arguments, existence proofs in number theory usually rely on
actually constructing the object using theorems about the existence of numbers
satisfying certain properties. These theorems include
Results~\ref{r:b:m:2:1},~\ref{r:b:n:e:3},~\ref{r:i:n:t:2},~\ref{r:i:n:pr:5},~\ref{r:a:n:d:2},~\ref{r:a:n:p:1},~\ref{r:a:n:o:2}
and~\ref{r:a:n:pr:1}.

Here are a few other results that may be helpful, but whose proofs are outside
the scope of Olympiad maths:
\begin{itemize}
  \item (Dirichlet's Theorem)
    For any coprime positive integers $a$ and $b$, there are infinitely many
    positive integers $k$ such that $a+bk$ is prime.
  \item (Bertrand's Postulate)
    For any positive integer $n$, there is a prime in $(n,2n]$. %chktex 9 
  \item (Zsigmondy's Theorem) If $a>b>0$ are coprime integers, then for any
    integer $n\ge 3$ there is a prime number $p$ that divides $a^n-b^n$ and does
    not divide $a^k-b^k$ for any positive integer $k<n$, unless
    $(a,b,n)=(2,1,6)$. The same holds for $a^n+b^n$ with the exception
    $2^3+1^3=9$.
\end{itemize}
Remember properties like Fermat/Euler and Wilson that allow you to control stuff.
CRT is especially useful because it allows
you to combine a bunch of modular conditions into one. Most of the time
Dirichlet then gives you a prime for free.
\begin{itemize}
  \item Prove that if $n$ is not a multiple of $4$, then there are positive
    integers $a$ and $b$ such that $n\mid a^2+b^2+1$.
  \item Prove that there exist infinitely many positive integers $n$ such that
    $n^2+1\mid n!$.
  \item Prove that there are infinitely many positive integers $n$ such that
    $d(n)$ and $\varphi(n)$ are both squares.
  \item Prove that there exists a positive integer $m$ such that the equation
    $\varphi(n)=m$ has at least $2023$ solutions $n$.
\end{itemize}
