\subsection{Polynomials}
A polynomial is just an expression of the form
\[P(x)=a_n x^n+a_{n-1}x^{n-1}+\cdots+a_0x^0,\] where each of the $a_i$s is a
constant and $a_n\ne 0$.
The number $n$ is called the \emph{degree} of the polynomial. The term $a_n x^n$
is called the \emph{leading term}. The expression
$P(x)=0$ is also a polynomial, and it's defined to have degree $-\infty$.

\begin{result}\label{r:i:a:polys:1}
  Let $A(x)$ and $B(x)$ be polynomials. Then,
  \begin{itemize}
    \item $\deg(A\pm B)\le\max(\deg A,\deg B)$. Equality occurs unless the
      leading terms of $A$ and $B$ cancel.
    \item $\deg(A\times B)=\deg(A)+\deg(B)$.
  \end{itemize}
\end{result}

Many proofs in polynomial questions proceed by (strong) induction on the degree. The
following few examples illustrate a few of the ways in which we can reduce a
polynomial to a smaller-degree polynomial.

\begin{result}[Division Algorithm]\label{r:i:a:polys:2}
  Let $A(x)$ and $B(x)$ be polynomials with $B(x)\ne 0$. There exist unique
  polynomials $Q(x)$ and $R(x)$ such that $\deg R<\deg B$ and
  \[A(x)=Q(x)B(x)+R(x).\]
  \hyperlink{sr:i:a:polys:2}{Solution}
\end{result}
We say that a polynomial $P(x)$ divides another polynomial $Q(x)$ if there is a
polynomial $R(x)$ such that $P(x)=Q(x)R(x)$.
With this terminology, an important corollary is that for any real number $r$ and any polynomial
$A(x)$, the polynomial $x-r$ divides the polynomial $A(x)-A(r)$.
\begin{result}[Vieta's Formulas]\label{r:i:a:polys:3}
  If a polynomial \[P(x)=a_n x^n+a_{n-1}x^{n-1}+\cdots+a_0x^0\] has roots
  $r_1,\ldots,r_n$ then for each $i$, \[a_{n-i}=(-1)^i a_n
  \sum_{j_1<\cdots<j_i} \prod_{k=1}^i r_{j_k}.\]
  \hyperlink{sr:i:a:polys:3}{Solution}
\end{result}
