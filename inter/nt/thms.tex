\subsection{Theorems About Mods}
\begin{result}[Wilson's Theorem]{\label{r:i:n:t:1}}
  $(n-1)!\equiv -1\pmod n$ if and only if $n$ is prime.
\end{result}
\begin{result}[Chinese Remainder Theorem]{\label{r:i:n:t:2}}
    Let $a_1,a_2,\ldots,a_k$ be pairwise coprime positive integers, and let
    $b_1,b_2,\ldots,b_k$ be integers. Then there is exactly one least
    residue $x$ mod $a_1a_2\cdots a_n$ such that for each $i$,
    \[b_i\equiv x\pmod {a_i}.\]
\end{result}
\begin{problem}{\label{p:i:n:t:1}}
    Call a lattice point ``visible'' if the greatest
    common divisor of its coordinates is 1. Prove that there exists a 100 × 100
    square on the board none of whose points are visible.
\end{problem}
\begin{result}{\label{r:i:n:t:3}}
    If $\gcd(a,b)=1$, then $\varphi(ab)=\varphi(a)\varphi(b)$.
\end{result}
\begin{problem}{\label{p:i:n:t:2}}
    Prove that for all composite $n>6$, $\varphi(n)\ge\sqrt n$.
\end{problem}
\begin{result}[Euler's Product Formula]{\label{r:i:n:t:4}}
    If $n=\prod p_i^{e_i}$, then
    \[\varphi(n)=n\prod\left(1-\frac 1{p_i}\right).\]
\end{result}
\begin{result}[Euler's Theorem]{\label{r:i:n:t:5}}
    If $\gcd(a,n)=1$, then $a^{\varphi(n)}\equiv 1\pmod n$.
\end{result}
The special case where $n=p$ is prime yields $a^{p-1}\equiv 1\pmod p$, which is
known as Fermat's Little Theorem.
\begin{problem}{\label{p:i:n:t:3}}
  Let $p$ be a prime and let $c$ be a positive integer. Prove that there
    exists a positive integer $x$ such that $x^x\equiv c\pmod p$.
\end{problem}
