\section{Assorted Problems}
\begin{problem}{\label{p:n:n:pr:1}}
  Prove that for each positive integer $n$ there exist $n$ consecutive
    positive integers, none of which is a prime power.
\end{problem}
\begin{problem}{\label{p:n:n:pr:2}}
  Consider a sequence of positive integers $a_1,a_2,\ldots$ which satisfies
    $a_n=a_{n-1}^2+a_{n-2}^2+a_{n-3}^2$ for all $n\ge 3$. Prove that if
    $a_k=1997$ then $k\le 4$.
\end{problem}
\begin{problem}{\label{p:n:n:pr:2b}}
      Prove that for positive integers $m,n>2$ we cannot have $2^m-1\mid
        2^n+1$.
\end{problem}
\begin{problem}{\label{p:n:n:pr:3}}
   For a natural number $N$, consider all distinct perfect squares that
    can be obtained from $N$ by deleting one digit from its decimal
    representation. Prove that the number of such squares is bounded by some
    value that doesn't depend on $N$.
\end{problem}
\begin{problem}{\label{p:n:n:pr:4}}
  Is there a polynomial $f$ of degree $2023$ with integer
    coefficients such that \[f(n), f(f(n)), f(f(f(n))), \cdots\] are pairwise
    relatively prime for any integer $n$?
\end{problem}
\begin{problem}{\label{p:n:n:pr:5}}
  Let $a$ and $b$ be positive irrational numbers such that $\frac 1a+\frac 1b=1$.
    Let $A=\{\lfloor na\rfloor: n\in\mathbb N\}$, and $B=\{\lfloor nb\rfloor:
    n\in\mathbb N\}$. Prove that the sets $A$ and $B$ together contain each
    posiive integer exactly once.
\end{problem}
\begin{problem}{\label{p:n:n:pr:6}}
  Find all functions $f:\Zz\to\Zz$ such that $f(m+f(n))=f(m)-n$ for all
    integers $m$ and $n$.
\end{problem}
\begin{problem}{\label{p:n:n:pr:7}}
  Find all polynomials $P(x)$ with integer coefficients such that if
    $m\mid n$ then $P(m)\mid P(n)$.
\end{problem}
\begin{result}{\label{r:n:n:pr:1}}
    $\displaystyle\sum_{d\mid n}\varphi(d)=n$.
\end{result}
\begin{problem}{\label{p:n:n:pr:8}}
  Solve over integers: $6(6a^2 + 3b^2 + c^2) = 5n^2$.
\end{problem}
\begin{problem}{\label{p:n:n:pr:9}}
  Let $p$ and $q$ be coprime. Prove that
      \[\sum_{i=1}^{q-1}\left\lfloor\frac{ip}{q}\right\rfloor=\frac{(p-1)(q-1)}2.\]
\end{problem}
\begin{problem}{\label{p:n:n:pr:9b}}
        Find all positive integers $n$ such that $3^{n-1}+5^{n-1}\mid
      3^n+5^n$.
\end{problem}
\begin{problem}{\label{p:n:n:pr:10}}
  Prove that for every positive integer $n$, there is a positive integer
    $X$ such that \[X,\ 2X,\ 3X,\ \ldots,\ nX\] are all nontrivial perfect powers.
\end{problem}
\begin{problem}{\label{p:n:n:pr:10b}}
      Let $a,b,c,d$ be positive integers with $ab=cd$. Prove that there
      exist positive integers $p,q,r,s$ such that \(a=pq,b=rs,c=pr,d=qs\).
\end{problem}
\begin{problem}{\label{p:n:n:pr:11}}
Let $a,b,c$ be positive integers such that $a^3+b^3=2^c$. Prove that
      $a=b$.
\end{problem}
\begin{problem}{\label{p:n:n:pr:12}}
  The sequence $\{a_i\}_1^\infty$ is defined by $a_1 = 1$ and
    $a_{n+1}=a_n^2+1$ for $n\ge 1$. Prove that there are infinitely many primes
    which divide some $a_i$.
\end{problem}
\begin{problem}{\label{p:n:n:pr:13}}
    We are given a positive integer $s \ge 2$. For each positive integer
    $k$, we define its twist $k'$ as follows: write $k$ as $as+b$, where $a, b$
    are non-negative integers and $b < s$, then $k' = bs+a$. For the positive
    integer $n$, consider the infinite sequence $d_1, d_2, \dots$ where $d_1=n$
    and $d_{i+1}$ is the twist of $d_i$ for each positive integer $i$.
    Prove that this sequence contains $1$ if and only if the remainder when $n$
    is divided by $s^2-1$ is either $1$ or $s$.
\end{problem}
\begin{problem}{\label{p:n:n:pr:14}}
    Does one of the first $10^8+1$ Fibonacci numbers end with four zeroes?
\end{problem}
\begin{problem}{\label{p:n:n:pr:14b}}
    For each prime $p$ and positive integer $k$, find the least residues of 
    \(\binom {p-1}k\) and \(\frac1p\binom pk\) in mod $p$.
\end{problem}
\begin{problem}{\label{p:n:n:pr:15}}
  Prove that the function $f(n)=\lfloor(1+\sqrt 2)^n\rfloor$ alternates
    between even and odd integers.
\end{problem}
\begin{problem}{\label{p:n:n:pr:16}}
  Prove that for any positive integer $n$ which is not a perfect square,
    there is a positive integer $k$ such that
    \[n=\left\lfloor k+\sqrt k+\frac12\right\rfloor.\]
\end{problem}
\begin{problem}{\label{p:n:n:pr:17}}
   Given are positive integers $a, b$ satisfying $a \geq 2b$. Does there
    exist a polynomial $P(x)$ of degree at least $1$ with coefficients from the
    set $\{0, 1, 2, \ldots, b-1 \}$ such that $P(b) \mid P(a)$?
\end{problem}
\begin{problem}{\label{p:n:n:pr:18}}
  Find all positive integers $a$ for which $1!+2!+\cdots+a!$ is a perfect
    cube.
\end{problem}
\begin{problem}{\label{p:n:n:pr:19}}
  Let $S(n)$ be the sum of the digits of $n$. Find $S(S(S(4444^{4444})))$.
\end{problem}
\begin{problem}{\label{p:n:n:pr:20}}
  Prove that the equation $y^2=x^3+7$ has no integer solutions.
\end{problem}
\begin{problem}{\label{p:n:n:pr:21}}
  Prove that for any positive integer $n$ we have $\sigma(n)\ge d(n)\sqrt n$.
\end{problem}
\begin{problem}{\label{p:n:n:pr:22}}
  Prove that there are infinitely many positive integers which are not the
    sum of a square and a prime.
\end{problem}
\begin{problem}{\label{p:n:n:pr:23}}
  Let $m$ and $c$ be integers.
    Prove that for any infinite sequence $a_1,a_2,\ldots$ of positive integers which
    contains every positive integer exactly once, there
    are integers $x,y,k$ such that $x<y$ and $a_x+a_{x+1}+\cdots+a_y=mk+c$.
\end{problem}
\begin{problem}{\label{p:n:n:pr:24}}
  Let $x$ be an irrational number, and let $a$ and $b$ be real numbers
    such that $0\le a<b\le 1$. Prove that there is an integer $n$ such that
    $a<\{nx\}<b$. Hence prove that there is a power of $2$ whose decimal
    representation starts with $2023$.
\end{problem}
\begin{result}{\label{r:n:n:pr:2}}
  Prove that if $p$ is an odd prime that divides $n^2+1$ for some integer
    $n$, then $p\equiv 1\pmod 4$.
\end{result}
\begin{problem}{\label{p:n:n:pr:25}}
      Prove that every positive integer is a sum of one or more numbers of
      the form $2^r3^s$, where $r$ and $s$ are nonnegative integers and no
      summand divides another.
\end{problem}
\begin{problem}{\label{p:n:n:pr:26}}
      Let $a$ and $b$ be positive integers such that $a\mid b^2\mid a^3\mid
      b^4\mid\cdots$

      Prove that $a=b$.
\end{problem}
\begin{problem}{\label{p:n:n:pr:27}}
  Find all pairs of positive integers $x,y$ such that
    $1+2^x+2^{2x+1}=y^2$.
\end{problem}
\begin{problem}{\label{p:n:n:pr:28}}
  Find all triples $(x, y, z)$ of integers such that
    \[x^3 + 2y^3 + 4z^3 - 6xyz = 0.\]
\end{problem}
\begin{problem}{\label{p:n:n:pr:29}}
  Prove that for any positive integer $c$ and any prime $p$, there is a
    positive integer $x$ such that $x^x\equiv c\pmod p$.
\end{problem}
\begin{problem}{\label{p:n:n:pr:30}}
      Let $m$ and $n$ be positive integers. Prove that
      \[m\mid\gcd(m,n)\binom mn.\]
\end{problem}
\begin{problem}{\label{p:n:n:pr:31}}
  Find all positive integers $a, b, c$ such that $a! \times b! = a! +
    b! + c!$.
\end{problem}
\begin{problem}{\label{p:n:n:pr:32}}
  Prove that any infinite sequence of integers in arithmetic progression has an
    infinite subsequence in geometric progression.
\end{problem}
\begin{problem}{\label{p:n:n:pr:33}}
  Find all positive integers $x, y$ such that $x^2-3xy+2y^2=2023$.
\end{problem}
\begin{problem}{\label{p:n:n:pr:34}}
  Prove that for any two distinct polynomials $P$ and $Q$ with coefficients in
    $\{0,1,\ldots,9\}$, either $P(-2)\ne Q(-2)$ or $P(-5)\ne Q(-5)$.
\end{problem}
\begin{problem}{\label{p:n:n:pr:35}}
  Find all positive integers $a,b,c$ such that $a\mid b+c,\ b\mid c+a,\
    c\mid a+b$.
\end{problem}
\begin{problem}{\label{p:n:n:pr:36}}
  Prove that $\lfloor\sqrt n+\sqrt{n+1}\rfloor=\lfloor\sqrt{4n+1}\rfloor$.
\end{problem}
\begin{problem}{\label{p:n:n:pr:37}}
  Determine all integers $n \geq 2$ such that $\sqrt{n-a^2}$ is an integer
    which divides $n$, where $a$ is the smallest prime divisor of $n$.
\end{problem}
\begin{problem}{\label{p:n:n:pr:38}}
    Let $p>3$ be prime. Define $m=(4^p-1)/3$. Prove that $2^{m-1}\equiv
    1\pmod m$.
\end{problem}
\begin{problem}{\label{p:n:n:pr:39}}
  Find all positive integers $x,y,z$ such that $3^x+4^y=5^z$.
\end{problem}
\begin{problem}{\label{p:n:n:pr:40}}
  Let $m$ be a positive integer for which there exists a positive integer
    $n$ such that $mn$ is a perfect square and $m-n$ is
    prime. Prove that $4m=(m-n+1)^2$.
\end{problem}
\begin{problem}{\label{p:n:n:pr:41}}
  Find all primes $p,q$ for which $pq\mid (5^p-2^p)(5^q-2^q)$.
\end{problem}
\begin{problem}{\label{p:n:n:pr:42}}
  Find all positive integers $n$ such that $1+\lfloor\sqrt n\rfloor$
    divides $n$.
\end{problem}
\begin{problem}{\label{p:n:n:pr:43}}
  Find all pairs of positive integers $x,y$ such that $y^2(x-1)=x^5-1$.
\end{problem}
\begin{problem}{\label{p:n:n:pr:44}}
  Let $n\ge 2$ be an integer and
    $P(x)$ be a polynomial with nonnegative integer coefficients satisfying
    $P(1)=1$ and $x^n P(1/x)=P(x)$ for all $x$.
    Prove that there exist infinitely many
    pairs $x, y$ of positive integers such that $x|P(y)$ and $y|P(x)$.
\end{problem}
\begin{problem}{\label{p:n:n:pr:45}}
  Prove that there is an infinite set of positive integers such that the
    sum of any finite subset is not a perfect power.
\end{problem}
\begin{problem}{\label{p:n:n:pr:46}}
    Find all primes $p$ such that $p^{2022}+p^{2023}$ is a perfect square.
\end{problem}
\begin{problem}{\label{p:n:n:pr:47}}
    Let $S$ be a subset of the set of numbers $\{1, 2, 3,\ldots, 2023\}$
    such that if $a,b$ are in $S$, then $23\nmid a+b$. What is the maximum
    possible size of $S$?
\end{problem}
\begin{problem}{\label{p:n:n:pr:48}}
  Prove that there exists a strictly increasing sequence $\{a_n\}_1^\infty$
    of positive integers such that for any $k\ge 0$, the sequence $\{k+a_n\}$
    contains only finitely many primes.
\end{problem}
\begin{problem}{\label{p:n:n:pr:49}}
    Prove that every positive integer has at least as many divisors which
    are 1 (mod 4) as divisors which are 3 (mod 4).
\end{problem}
\begin{result}{\label{r:n:n:pr:3}}
    If $a,\ b,\ c,\ x$ are integers such that $ax^2+bx+c=0$ then $b^2-4ac$ is a
    perfect square.
\end{result}
\begin{problem}{\label{p:n:n:pr:50}}
  Prove that the equation $x^3+3=4y(y+1)$ has no integer solutions.
\end{problem}
\begin{problem}{\label{p:n:n:pr:51}}
  Does there exist an infinite sequence of integers $a_1,a_2,\ldots$ such
    that $\gcd(a_m,a_n)=1\iff |m-n|=1$?
\end{problem}
\begin{problem}{\label{p:n:n:pr:52}}
  Find all triples of positive integers $x, y, z$ such that
      $x^3+y^3+z^3-3xyz$ is prime.
\end{problem}
\begin{problem}{\label{p:n:n:pr:53}}
  Prove that for each positive integer $n$, there are $n$ consecutive
    positive integers, none of which is a prime power.
\end{problem}
\begin{problem}{\label{p:n:n:pr:54}}
  Find all functions $f:\Nn\to\Nn$ satisfying $f(n+f(n))=f(n)$ for all
    $n$ such that 1 is in the range of $f$.
\end{problem}
\begin{problem}{\label{p:n:n:pr:55}}
  Show that $30$ is the greatest common divisor of all numbers of the form
    $2^{3n}+5^{n+1}+3^{n+2}$, where $n \in \mathbb{N}$.
\end{problem}
\begin{result}{\label{r:n:n:pr:4}}
    If $a,\ b,\ c$ are positive integers such that $a^2+b^2=c^2$, then there
    exist integers $d,x,y$ such that $\{a,b\}=\{d(x^2-y^2),2dxy\}$ and
    $c=d(x^2+y^2)$.
\end{result}
\begin{problem}{\label{p:n:n:pr:56}}
  Prove that for every positive integer $n$, there is a set $S$ of $n$
    distinct positive integers such that every subset of $S$ has a geometric
    mean which is a positive integer.
\end{problem}
\begin{problem}{\label{p:n:n:pr:57}}
    Prove that every arithmetic progression $\{a+nb\}_{n=1}^\infty$ where
    $\gcd(a,b)=1$ has infinitely many terms which are not divisible by any
    perfect square larger than $1$.
\end{problem}
\begin{problem}{\label{p:n:n:pr:58}}
    Prove that for every positive integer $n$ there is a number divisible by
    $n$ consisting of only $1$s and $0$s.
\end{problem}
\begin{problem}{\label{p:n:n:pr:58b}}
    Prove that if $n$ is not a multiple of $5$, there is a number
    divisible by $n$ consisting of only $1$s and $2$s.
\end{problem}
\begin{problem}{\label{p:n:n:pr:59}}
  Prove that if $p$ is prime, then $2^p+3^p$ is not a nontrivial perfect
    power.
\end{problem}
\begin{problem}{\label{p:n:n:pr:60}}
  Let $P(x)$ and $Q(x)$ be polynomials with integer coefficients such that the
    leading coefficient of $P(x)$ is 1. Suppose that $P(n)^n$ divides $Q(n)^{n+1}$ for infinitely
    many positive integers $n$.
    Prove that $P(n)$ divides $Q(n)$ for infinitely many positive integers $n$.
\end{problem}
\begin{problem}{\label{p:n:n:pr:61}}
  Let $f$ be a function defined on the nonnegative integers such that
    $f(2x)=2f(x)$, $f(4x+1)=4f(x)+3$, and $f(4x-1)=2f(2x-1)-1$. Prove that $f$
    is injective.
\end{problem}
\begin{problem}{\label{p:n:n:pr:62}}
  Prove that for every positive integer $n$, there are infinitely many
    terms of the Fibonacci sequence which are divisible by $n$.
\end{problem}
\begin{problem}{\label{p:n:n:pr:63}}
  Find all pairs of positive integers $x,y$ such that $x^2-y!=2001$.
\end{problem}
\begin{problem}{\label{p:n:n:pr:64}}
  Prove that if $m$ and $n$ are natural numbers, then $3^m+3^n+1$ is not a
    perfect square.
\end{problem}
\begin{problem}{\label{p:n:n:pr:65}}
      Let $n$ be a positive integer. Calculate $\gcd((n-1)!+1,n!)$.
\end{problem}
\begin{problem}{\label{p:n:n:pr:66}}
  Let $a,b$ be odd positive integers. Define the sequence $c_n$ by
    choosing $c_1=a,c_2=b$ and for each $i>2$ letting $c_i$ be the largest odd
    divisor of $c_{i-1}+c_{i-2}$. Prove that this sequence is eventually
    constant.
\end{problem}
\begin{problem}{\label{p:n:n:pr:67}}
  Prove that there does not exist a function $f:\Nn\to\Nn$
    such that for any distinct positive integers $i$ and $j$,
    $\gcd(f(i)+j,f(j)+i)=1$.
\end{problem}
\begin{problem}{\label{p:n:n:pr:68}}
  Find all pairs $a,b$ of positive integers such that $2017^a=b^6-32b+1$.
\end{problem}
\begin{problem}{\label{p:n:n:pr:69}}
  Do there exist primes $x,y,z$ such that $x^2+y^3=z^4$?
\end{problem}
\begin{problem}{\label{p:n:n:pr:70}}
      The cells in a jail are numbered from 1 to 100, and there are 100
    buttons also numbered from 1 to 100. For each $i$, the $i$th button opens a
    closed cell and closes an open cell, affecting only the multiples of $i$. 

    For example, if the 47th cell is open and the 94th cell is closed, then
    pressing the 47th button will close the 47th cell and open the 94th cell.

    Initially all cells are closed.
    The warden presses the first button, then the second, and so on, for all
    100 buttons. Which cells are open at the end?
\end{problem}
\begin{problem}{\label{p:n:n:pr:71}}
  Find all pairs of integers $x,y$ such that $x^4+2x^2y+y^3=0$.
\end{problem}
\begin{problem}{\label{p:n:n:pr:72}}
  Find all triples $a,b,c$ of positive integers such that $a\mid bc-1,\
    b\mid ca-1$ and $c\mid ab-1$.
\end{problem}
\begin{problem}{\label{p:n:n:pr:73}}
  Do there exist two quadratics $ax^2+bx+c$ and $(a+1)x^2+(b+1)x+(c+1)$
    which both have two integer roots?
\end{problem}
\begin{problem}{\label{p:n:n:pr:74}}
  Find all primes $p$ and $q$ such that $p+q=(p-q)^3$.
\end{problem}
\begin{problem}{\label{p:n:n:pr:75}}
  Find all integers $a,b$ such that $a^3+(a+1)^3+\cdots+(a+6)^3=b^4+1$.
\end{problem}
\begin{problem}{\label{p:n:n:pr:76}}
    Define a sequence by $a_1=n$ and $a_{i+1}=\frac{a_i(a_i-1)}2$ for each
    $i\ge 1$. For which positive integers $n$ are all values of $a_i$ odd?
\end{problem}
\begin{problem}{\label{p:n:n:pr:76b}}
  Compute the remainder when $2023^{2022}$ is divided by $2021$.
\end{problem}
\begin{problem}{\label{p:n:n:pr:77}}
    Suppose $a_1,a_2, \dots$ is an infinite strictly increasing sequence of
      positive integers and $p_1, p_2, \dots$ is a sequence of distinct primes
      such that $p_n \mid a_n$ for all $n \ge 1$. It turns out that
      $a_{n+1}-a_n=p_{n+1}-p_n$ for all $n\ge 1$. Prove that the sequence $\{a_n\}$
      consists only of prime numbers.
\end{problem}
\begin{problem}{\label{p:n:n:pr:78}}
  Find all monotonically increasing functions $f:\Nn\to\Zz_{\ge 0}$ such that $f(mn)=f(m)+f(n)$ for
    all nonnegative integers $m$ and $n$.
\end{problem}
\begin{problem}{\label{p:n:n:pr:79}}
  Let $n$ be a positive integer. Define a sequence by letting $a_1=n$, and
    for each $i>1$ choosing $a_i$ such that $0\le a_i<i$ and
    $\frac{a_1+\cdots+a_i}i$ is an integer. Prove that this sequence is
    eventually constant.
\end{problem}
\begin{problem}{\label{p:n:n:pr:80}}
  Given are positive integers $n>20$ and $k>1$, such that $k^2$ divides
      $n$. Prove that there exist
      positive integers $a, b, c$ such that $n=ab+bc+ca$.
\end{problem}
