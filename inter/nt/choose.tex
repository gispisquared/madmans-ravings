\section{Choosing Good Mods}
Many number theory problems, especially Diophantine equations, can be solved by
looking at them using an appropriate mod. These are especially common:
\begin{result}\label{r:n:n:c:1}
    Squares are 0, 1 or 4 mod each of \{5,8\}, and 0 or 1 mod 3.
\end{result}
\begin{result}\label{r:n:n:c:2}
    Squares are 0, 1 or $-1$ mod each of \{7,9\}.
\end{result}
In general, for $n$th powers, try looking mod $m$ where $\varphi(m)$ is a
small multiple of $n$.
You can also try choosing a mod which divides a bunch of terms in an
equation.

However, remember that if you find a single solution to a polynomial equation, then that
solution is still a solution in every mod so you won't be able to find a
contradiction.
\begin{problem}{\label{p:n:n:c:1}}
  Find all positive integers $a,b$ such that $a^4+b^4=10a^2b^2-2022$.
\end{problem}
\begin{problem}{\label{p:n:n:c:2}}
  Find all positive integers $n$ such that $2^n+7^n$ is a perfect square.
\end{problem}
\begin{problem}{\label{p:n:n:c:3}}
  Find all pairs of positive integers $x,y$ such that $x!+5=y^3$.
\end{problem}
