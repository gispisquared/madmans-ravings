\documentclass{amsart}
\usepackage{amsmath,amssymb,amsthm,hyperref}
\hypersetup{colorlinks=true}
\newtheorem{problem}{Problem}
\newtheorem{example}{Example}
\title{Some Madman's Ravings}
\author{gispisquared}
\begin{document}
\maketitle
\tableofcontents
%\part{Introduction}
%I've been told to leave the intro till last, and I'm lazy so I will.
\part{Methods of Proof}
Since I refuse to rehash stuff that others have done better, I'll refer you to a
couple of resources about how to write proofs properly:
\begin{itemize}
  \item
    \href{https://artofproblemsolving.com/news/articles/how-to-write-a-solution}
    {How to Write a Maths Solution}
  \item Chapter 1 of the
    \href{https://web.evanchen.cc/textbooks/OTIS-Excerpts.pdf}{OTIS Excerpts}
\end{itemize}

Cool, hopefully now you know how to write proofs. Guess that means every time
you solve a problem you'll get a 7, right?

Now it's time to learn how to actually prove something. There are a few main
methods of proof --- that is, ways in which you can go from the conditions in the
problem to your given condition.

\section{Direct Proof}
This is perhaps the simplest type of proof. The idea is to start only with
the stuff you're given, and finish with what you want to prove.

Time for an example.

\begin{example}
  \hypertarget{ex:1}
  Let $a, b$ be positive real numbers. Prove that \[\frac{a+b}2\ge\sqrt{ab}.\]
\end{example}
\hyperlink{s:1}{Solution}
\part{Theory}
\part{Practice}
\part{Problems}
\part{Proofs}
\hyperlink{ex:1}{Problem}
\begin{proof}
  \hypertarget{s:1}
  Since squares are nonnegative, we have
  \begin{align*}
    \left(\sqrt a-\sqrt b\right)^2 & \ge 0 \\
    \iff a+b-2\sqrt{ab} & \ge 0 \\
    \iff a+b & \ge 2\sqrt{ab}. \qedhere
  \end{align*}
\end{proof}
\end{document}
