%chktex-file 3
\documentclass{amsart}
\usepackage{amsmath,amssymb,amsthm,tcolorbox,hyperref}
\hypersetup{colorlinks=true}
\newtheorem{problem}{Problem}
\newtheorem{example}{Example}
\newcommand{\Rr}{\mathbb{R}}
\title{Some Madman's Ravings}
\author{gispisquared}
\begin{document}
\maketitle
\tableofcontents
\newpage
\section{Introduction}
I'm writing this since I think that most of the resources out there that try to
teach you olympiad maths have all the theory you need to solve IMO-level
problems (and sometimes much more!), but it's much more rare to see people who
tell you how to look at problems and figure out approaches that may, or in some
cases should, work. True, much of this is an individual learning experience, but
I think a lot of this can be written more explicitly.

This may reduce some of the magic of figuring this stuff out for yourself; to
mitigate this effect, I have relegated the solutions to examples to the back so
that you can (and I encourage you to!) try the examples before reading the
solutions.

Since I refuse to rehash stuff that others have done better, I'll refer you to a
couple of resources about how to write proofs properly:
\begin{itemize}
  \item
    \href{https://artofproblemsolving.com/news/articles/how-to-write-a-solution}
    {How to Write a Maths Solution}
  \item \href{https://web.evanchen.cc/handouts/english/english.pdf}{Notes on
    English}
\end{itemize}
Cool, hopefully now you know how to write proofs. Guess that means every time
you solve a problem you'll get a 7, right?
\newpage
\section{Theory}
Now it's time to learn how proofs work.
\subsection{Methods of Proof}
There are a few main methods of proof --- that is, ways in which you can go from
the conditions in the problem to your given condition.
\subsubsection{Direct Proof}
This is perhaps the simplest type of proof. The idea is to start with the stuff
you're given, do some logical deduction, and finish with what you want to prove.
\begin{example}
  \hypertarget{p:1:1:1}
  Prove that the remainder when a perfect square is divided by 4 is either 0 or
  1.
  \hyperlink{s:1:1:1}{Solution}
\end{example}
\subsubsection{Contradiction}
This is where you assume that what you're trying to prove is wrong and try to
derive some kind of logical impossibility. Then the only place where the logic
could have gone wrong was in the assumption so the statement you were trying to
prove must be true.
\begin{example}
  \hypertarget{p:1:2:1}
  Prove that there are infinitely many primes.
  \hyperlink{s:1:2:1}{Solution}
\end{example}
\subsubsection{Contrapositive}
It turns out that the statement $A\implies B$ is logically equivalent to the
statement $\neg B\implies\neg A$. This is probably easiest to see intuitively
with an example: ``If John is a bachelor, then John is not married'' is logically
equivalent to ``If John is married, then John is not a bachelor''. Therefore, if
we're asked to prove $A\implies B$, it's enough to prove $\neg B\implies\neg A$,
which is sometimes easier.
\begin{example}
  \hypertarget{p:1:3:1}
  Let $a,b\in\Rr$ such that $a+b$ is irrational. Prove that at least one of $a$
  and $b$ is irrational.
  \hyperlink{s:1:3:1}{Solution}
\end{example}
\subsubsection{Induction}
Perhaps the hardest to understand of the basic proof techniques, this can be
used to prove properties of positive integers where the property for each
integer can be related to those of previous integers.

Here is the Principle of Mathematical Induction (PMI):
\begin{tcolorbox}
Let $S$ be a set of positive integers such that $1\in S$ and for each $k\in S$,
$k+1\in S$. Then $S$ contains all positive integers.
\end{tcolorbox}

To prove a statement for all positive integers, we let $S$ be the set of all
positive integers for which the statement is true. Then it's enough to prove:
\begin{itemize}
  \item $1\in S$. This is called the \emph{base case}.
  \item If $k\in S$ (the \emph{inductive hypothesis}), then $k+1\in S$. This is
    called the \emph{inductive case}.
\end{itemize}
Then by PMI, $S$ will contain all positive integers.

There are two ways to make induction superficially more powerful, though they're
both equivalent to the usual form of induction:
\begin{itemize}
  \item Say we want to prove a statement for all integers larger than $n$, for
    some $n$. Then it's enough to prove:
    \begin{itemize}
      \item The statement is true for $n$.
      \item If the statement is true for some integer $k>n$, then it's true for
        $k+1$.
    \end{itemize}
    This is equivalent to the normal PMI:\@ to see this, let $S$ be the set of
    all integers $m$ for which the statement is true for all $m+n$.
  \item Say we want to use not just the inductive assumption not just for $k$,
    but for smaller integers as well. Intuitively this should be fine, since
    we've in some sense ``proved this already'' by the time we get to $k+1$.
    Formally, to prove a statement $P(n)$ for all positive integers $n$, it's
    enough to prove:
    \begin{itemize}
      \item $P(1)$.
      \item If $P(1),\ldots,P(k)$ are all true, then $P(k+1)$ is also true.
    \end{itemize}
    This form of proof by induction is called \emph{strong induction}, and
    although most proofs by induction only explicitly use $P(k)$, there's no
    reason to try to make your proof inductive over strong inductive since
    strong induction gives you more assumptions to work with ``for free''.
\end{itemize}
The key idea in both of these reductions to PMI is to somehow encapsulate the
extra information you're trying to assume into the framework of standard PMI.
\newpage
\section{Practice}
\newpage
\section{Solutions}
\hyperlink{p:1:1:1}{Example 1}
\hypertarget{s:1:1:1}
If you start trying some small cases, what you'll eventually find is that if $n$
is an even integer, then $n^2$ leaves a remainder of 0 when divided by 4, and
if $n$ is an odd integer, then $n^2$ leaves a remainder of 1 when divided by
4. Once you've conjectured this, all that's left is to recall what it means for
a number to be even or odd, and then the proof falls out quite naturally:
\begin{proof}
  Let the perfect square be $n^2$. We split into cases depending on the parity
  of $n$.
  \begin{itemize}
    \item If $n$ is even, let $n=2m$ for some integer $m$. Then
      \[n^2=(2m)^2=4m^2,\] which leaves a remainder of 0 when divided by 4.
    \item If $n$ is odd, let $n=2m+1$ for some integer $m$. Then
      \[n^2=(2m+1)^2=4m^2+4m+1=4(m^2+m)+1,\] which leaves a remainder of 1 when
      divided by 4.
  \end{itemize}
  In either case, the remainder left when dividing $n^2$ by 4 is either 0 or 1,
  which is what we wanted to prove.
\end{proof}
\hyperlink{p:1:2:1}{Example 2}
\hypertarget{s:1:2:1}
The key here is to assume, for contradiction, that there are
only finitely many primes. Then we want to prove a suitable contradiction --- a
nice way of doing this is to find a number that isn't 1 but isn't divisible by
any of our finitely many primes. The idea of constructing such a number by
multiplying everything and adding 1 is surprisingly common.
\begin{proof}
  Assume that there were only finitely many primes $p_1,p_2,\ldots,p_n$. Then,
  consider the number $A=p_1p_2\cdots p_n+1$. Then $A$ is a positive integer
  larger than 1, so
  it must have a prime factor, which means that for some $i$, $p_i\mid A$. But
  clearly also $p_i\mid A-1$, so $p_i\mid A-(A-1)=1$, which is our
  contradiction.
\end{proof}
\hyperlink{p:1:3:1}{Example 3}
\hypertarget{s:1:3:1}
The problem itself is not of much interest once you realise that you're meant to
prove the contrapositive. However, this problem is instructive for two main
reasons. First, it shows how much easier it is (in this particular instance) to
prove the contrapositive than the original statement. You could be stuck for
ages trying the original problem, but as soon as you ask ``what if the conclusion
wasn't true'' the problem pretty much solves itself. The second, more subtle,
takeaway here is about how to realise that you should use some kind of indirect
(contrapositive or contradiction) approach. Usually, a conclusion that has some
kind of ``or'' statement in it is a good sign that an indirect approach may be
easier, since negating the conclusion turns the ``or'' into an ``and'', which is
much easier to work with. The exception is when you can split the conditions
into cases that naturally give you each part of the ``or'', like in
\hyperlink{p:1:1:1}{Example 1}.
\begin{proof}
  We prove the contrapositive: that if $a$ and $b$ are both rational, then so is
  $a+b$.

  Let $a=\frac wx,\ b=\frac yz$. Then \[a+b=\frac wx+\frac
  yz=\frac{wz+xy}{xz},\] which is clearly rational.
\end{proof}
\end{document}
