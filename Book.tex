%chktex-file 3
\documentclass{amsart}
\usepackage{amsmath,amssymb,amsthm,tcolorbox,hyperref}
\tcbset{colback=gray!15!black, coltext=white}
\hypersetup{colorlinks=true, linkcolor=blue!50!white}
\newtheorem{problem}{Problem}
\newtheorem{result}{Result}
\pagecolor{black}
\color{white}
\newcommand{\LHS}{\text{LHS}}
\newcommand{\RHS}{\text{RHS}}
\newcommand{\Rr}{\mathbb{R}}
\newcommand{\Nn}{\mathbb{N}}
\title{Some Madman's Ravings}
\author{gispisquared}
\begin{document}
\maketitle
\tableofcontents
\newpage
\section*{Introduction}
\emph{If you can't be bothered reading, see the tldr \hyperlink{tldr}{here}}.

I'm writing this since I think that most of the resources out there that try to
teach you olympiad maths have all the theory you need to solve IMO-level
problems (and sometimes much more!), but it's much more rare to see people who
tell you how to look at problems and figure out approaches that may, or in some
cases should, work. True, much of this is an individual learning experience, but
I think a lot of this can be written more explicitly.

This may reduce some of the magic of figuring this stuff out for yourself; to
mitigate this effect, I have relegated all the proofs and solutions to
the back so that you can try prove everything before reading the solutions.

Apart from the methods of proof chapter, the overwhelming majority of this book
will be problems and solutions. There will be two main types of problems:
\begin{itemize}
  \item \emph{Results}, which are considered important and well-known, and come
    up sporadically (or in some cases consistently) as steps in the harder
    problems.
  \item \emph{Problems}, which will be questions taken from contests (mostly AMC
    and AIMO in Part 2, and AMO, EGMO and IMO in Part 3).
\end{itemize}
The ideas used to prove results are usually considered pretty
basic ideas, though I think that's not because they're simple but because
they're used often enough that they become standard. Then again I usually gave
up far too early and looked at the proofs, so maybe I just didn't give myself
enough time to prove them back when I was learning them.

Since I refuse to rehash stuff that others have done better, I'll refer you to a
couple of resources about how to write proofs properly:
\begin{itemize}
  \item
    \href{https://artofproblemsolving.com/news/articles/how-to-write-a-solution}
    {How to Write a Maths Solution}
  \item \href{https://web.evanchen.cc/handouts/english/english.pdf}{Notes on
    English}
\end{itemize}
Cool, hopefully now you know how to write proofs. Guess that means every time
you solve a problem you'll get a 7, right?

Oh, and apologies for the bad formatting typesetting hyperlinking etc.
\newpage
\part{Theory}
\section{Basics}
\subsection{Methods of Proof}
\subsubsection{Direct Proof}
This is perhaps the simplest type of proof. The idea is to start with the stuff
you're given, do some logical deduction, and finish with what you want to prove.
\begin{result}
  \hypertarget{r:m:1:1}
  The remainder when a perfect square is divided by 4 is either 0 or 1.
  \hyperlink{s:m:1:1}{Solution}
\end{result}
\subsubsection{Contradiction}
This is where you assume that what you're trying to prove is wrong and try to
derive some kind of logical impossibility. Then the only place where the logic
could have gone wrong was in the assumption so the statement you were trying to
prove must be true.
\begin{result}
  \hypertarget{r:m:2:1}
  There are infinitely many primes.
  \hyperlink{s:m:2:1}{Solution}
\end{result}
\subsubsection{Contrapositive}
It turns out that the statement $A\implies B$ is logically equivalent to the
statement $\neg B\implies\neg A$. This is probably easiest to see intuitively
with an result: ``If $x$ is an integer, then $x$ is rational'' is
logically equivalent to ``If $x$ is not rational, then $x$ is not an integer''.
Therefore, if we're asked to prove $A\implies B$, it's enough to
prove $\neg B\implies\neg A$, which is sometimes easier.
\begin{problem}
  \hypertarget{p:m:3:1}
  Let $a,b\in\Rr$ such that $a+b$ is irrational. Prove that at least one of $a$
  and $b$ is irrational.
  \hyperlink{s:m:3:1}{Solution}
\end{problem}
\subsubsection{Induction}
Perhaps the hardest to understand of the basic proof techniques, this can be
used to prove properties of positive integers where the property for each
integer can be related to those of previous integers.

Here is the Principle of Mathematical Induction (PMI):
\begin{tcolorbox}
Let $S$ be a set of positive integers such that $1\in S$ and for each $k\in S$,
$k+1\in S$. Then $S$ contains all positive integers.
\end{tcolorbox}

To prove a statement for all positive integers, we let $S$ be the set of all
positive integers for which the statement is true. Then it's enough to prove:
\begin{itemize}
  \item $1\in S$. This is called the \emph{base case}.
  \item If $k\in S$ (the \emph{inductive hypothesis}), then $k+1\in S$. This is
    called the \emph{inductive case}.
\end{itemize}
Then by PMI, $S$ will contain all positive integers.

There are two ways to make induction superficially more powerful, though they're
both equivalent to the usual form of induction:
\begin{itemize}
  \item Say we want to prove a statement for all integers larger than $n$, for
    some $n$. Then it's enough to prove:
    \begin{itemize}
      \item The statement is true for $n$.
      \item If the statement is true for some integer $k>n$, then it's true for
        $k+1$.
    \end{itemize}
    This is equivalent to the normal PMI\@: to see this, let $S$ be the set of
    all integers $m$ for which the statement is true for all $m+n$.
  \item Say we want to use not just the inductive assumption not just for $k$,
    but for smaller integers as well. Intuitively this should be fine, since
    we've in some sense ``proved this already'' by the time we get to $k+1$.
    Formally, to prove a statement $P(n)$ for all positive integers $n$, it's
    enough to prove:
    \begin{itemize}
      \item $P(1)$.
      \item If $P(1),\ldots,P(k)$ are all true, then $P(k+1)$ is also true.
    \end{itemize}
    This form of proof by induction is called \emph{strong induction}, and
    although most proofs by induction only explicitly use $P(k)$, there's no
    reason to try to make your proof inductive over strong inductive since
    strong induction gives you more assumptions to work with ``for free''.
\end{itemize}
The key idea in both of these reductions to PMI is to somehow encapsulate the
extra information you're trying to assume into the framework of standard PMI\@.
\begin{result}
  \hypertarget{r:m:4:1}
  For all positive integers $n$, \[1+2+\cdots+n=\frac{n(n+1)}2.\]
  \hyperlink{s:m:4:1}{Solution}
\end{result}
To conclude the Methods of Proof section, I'll include one final result that
combines most of what we've covered so far.
\begin{result}
  \hypertarget{r:m:4:2}
  The \emph{Well-Ordering Principle} states that any set of positive integers
  has a least element. It's equivalent to PMI --- that is,
  PMI is true if and only if well-ordering is true.
  \hyperlink{s:m:4:2}{Solution}
\end{result}
I find it intriguing that induction and minimality are really just two sides of
the same coin. Often you will find that a solution is much more natural to think
about and write up in terms of one than the other.
\subsection{Algebra}
\subsubsection{Factorisations}
I won't have any problems attached to these, but they tend to pop up everywhere
so keep an eye out. Here are some common factorisations:
\begin{itemize}
  \item $x^2-a^2=(x+a)(x-a)$
  \item $x^2-2ax+a^2=(x-a)^2$
  \item $x^2+2ax+a^2=(x+a)^2$
  \item $x^3-a^3=(x-a)(x^2+ax+a^2)$
  \item $x^3+a^3=(x+a)(x^2-ax+a^2)$
  \item $x^3+3x^2a+3xa^2+a^3=(x+a)^3$
  \item $x^3-3x^2a+3xa^2-a^3=(x-a)^3$
\end{itemize}
The cases $a=1$ are especially common.
\subsubsection{Systems of equations}
There are a couple of ways of solving these systems --- either you can
isolate one variable, substitute into the rest of the equations, and repeat, or
you can try and combine the equations in such a way that stuff cancels. The
first method is usually fine in school maths and the AMC, but the second is more
likely to be useful in harder Olympiad questions.

Sometimes these techniques won't be enough --- see \hyperlink{n:a:systems}{here}.
\begin{problem}
  \hypertarget{p:b:a:systems:1}
  The difference between two numbers is 20. When 4 is added to each number the
  larger is three times the smaller. What is the larger of the two original
  numbers?
  \hyperlink{s:b:a:systems:1}{Solution}
\end{problem}
\begin{problem}
  \hypertarget{p:b:a:systems:2}
  Find all triples $(x,y,z)$ of real numbers that simultaneously satisfy the
  equations
  \begin{align*}
    xy+1&=2z \\
    yz+1&=2x \\
    zx+1&=2y \\
  \end{align*}
  \hyperlink{s:b:a:systems:2}{Solution}
\end{problem}
\subsubsection{Quadratics}
\begin{result}
  \hypertarget{r:b:a:quad:1}
  Let $m$ and $p$ be given real numbers. All real numbers $x$ such that
  \[x^2-2mx+p=0\] are given by $x=m\pm\sqrt{m^2-p}$.
  \hyperlink{sr:b:a:quad:1}{Solution}
\end{result}
\begin{result}
  \hypertarget{r:b:a:quad:2}
  Let $a,\ b,\ c$ be given real numbers. All real numbers $x$ such that
  \[ax^2+bx+c=0\] are given by \[x=\frac{-b\pm\sqrt{b^2-4ac}}{2a}.\]
  In particular, if we let $\Delta=b^2-4ac$, then the equation has no real roots
  if $\Delta<0$, exactly one real root if $\Delta=0$, and two real roots if
  $\Delta>0$.
  \hyperlink{sr:b:a:quad:2}{Solution}
\end{result}
This number $\Delta$ is called the \emph{discriminant} of the quadratic.

Now, a couple of problems which show how useful both the results and the method
are.
\begin{problem}
  \hypertarget{p:b:a:quad:1}
  [See \hyperlink{b:n:bases}{here} if you don't know what number bases are.]

  The number $x$ is 111 when written in base $b$, but it is 212 when written in
  base $b-2$. What is $x$ is base 10?
  \hyperlink{sp:b:a:quad:1}{Solution}
\end{problem}
\begin{problem}
  \hypertarget{p:b:a:quad:2}
  For each pair of real numbers $(r,s)$, prove that there exists a real numer
  $x$ that satisfies at least one of the following two equations.
  \begin{align*}
    x^2+(r+1)x+s&=0 \\
    rx^2+2sx+s&=0
  \end{align*}
  \hyperlink{sp:b:a:quad:2}{Solution}
\end{problem}
\begin{problem}
  \hypertarget{p:b:a:quad:3}
  Find all real numbers $x$ for which $x^3+3x^2+3x+5=0$.
  \hyperlink{sp:b:a:quad:3}{Solution}
\end{problem}
\subsubsection{Inequalities}
At this level, inequalities are mostly about making stuff into squares or, well,
``mostly-squares''. The guiding principle is to try and find an expression which
you want to be always nonnegative, figure out where it's 0, and write it in
terms of stuff that's 0 there and obviously nonnegative elsewhere.

\begin{result}
  \hypertarget{r:b:a:ineq:1}
  If $a$ and $b$ are real numbers, then \[\frac{a+b}2\ge\sqrt{ab}.\]
  \hyperlink{sr:b:a:ineq:1}{Solution}
\end{result}

\begin{problem}
  \hypertarget{p:b:a:ineq:1}
  The set $S$ consists of distinct integers such that the smallest is 0 and the
  largest is 2015. What is the minimum possible average value of the numbers in
  $S$?
  \hyperlink{sp:b:a:ineq:1}{Solution}
\end{problem}
\subsubsection{Sums of sequences}
\begin{result}
  \hypertarget{r:b:a:sums:1}
  If $n$ is a positive integer and $a$ and $b$ are real numbers, then
  \[\sum_{i=0}^n(a+bi)=\frac{(n+1)(2a+bn)}2.\]
  \hyperlink{sr:b:a:sums:1}{Solution}
\end{result}
\begin{result}
  \hypertarget{r:b:a:sums:2}
  If $n$ is a positive integer and $r$ is a real number distinct from 1, then
  \[\sum_{i=0}^n r^i=\frac{1-r^{n+1}}{1-r}.\]
  \hyperlink{sr:b:a:sums:2}{Solution}
\end{result}
\subsubsection{Polynomials}
A polynomial is just an expression of the form
\[P(x)=a_n x^n+a_{n-1}x^{n-1}+\cdots+a_0x^0,\] where each of the $a_i$s is a
constant and $a_n\ne 0$.
The number $n$ is called the \emph{degree} of the polynomial. The expression
$P(x)=0$ is also a polynomial, and it's defined to have degree $-\infty$.

\begin{result}
  \hypertarget{r:b:a:polys:1}
  If a polynomial $P(x)$ of degree $n$ has a root $r$, then there is a
  polynomial $Q(x)$ of degree $n-1$ such that \[P(x)=(x-r)Q(x).\]
  \hyperlink{sr:b:a:polys:1}{Solution}
\end{result}
\begin{result}
  \hypertarget{r:b:a:polys:2}
  If a polynomial \[P(x)=a_n x^n+a_{n-1}x^{n-1}+\cdots+a_0x^0\] has roots
  $r_1,\ldots,r_n$ then for each $i$, \[a_{n-i}=(-1)^i a_n
  \sum_{j_1<\cdots<j_i} \prod_{k=1}^i r_{j_k}.\]
  \hyperlink{sr:b:a:polys:2}{Solution}
\end{result}
\subsection{Combinatorics}
\subsubsection{Addition and Multiplication Principles}
\begin{problem}
  \hypertarget{p:b:c:am:1}
  A hockey game between two teams is `relatively close' if the number of goals
  scored by the two teams never differ by more than two. In how many ways can
  the first 12 goals of a game be scored if the game is `relatively close'?
  \hyperlink{s:b:c:am:1}{Solution}
\end{problem}
\subsubsection{Permutations and Combinations}
\subsubsection{Double Counting and Combinatorial Identities}
\subsubsection{Venn Diagrams and PIE}
\subsubsection{Supermarket Principle}
\subsubsection{Recurrences}
\subsection{Geometry}
\subsubsection{Congruence}
\subsubsection{Basic angle chasing}
\subsubsection{Similarity}
\subsubsection{Pythagoras}
\subsubsection{Common right-angled triangles}
\subsection{Number Theory}
\subsubsection{Primes and FTOA}
\subsubsection{SFFT}
\subsubsection{Divisibility}
\subsubsection{Number bases}
\hypertarget{b:n:bases}
hi
\section{Not-so-basics}
\subsection{Algebra}
\subsubsection{Systems of equations}
\hypertarget{n:a:systems}
hi
\subsection{Combinatorics}
\subsection{Geometry}
\subsection{Number Theory}
\part{Practice}
\newpage
\part{Solutions}
\hyperlink{r:m:1:1}{Result 1}
\hypertarget{s:m:1:1}
If you start trying some small cases, what you'll eventually find is that if $n$
is an even integer, then $n^2$ leaves a remainder of 0 when divided by 4, and
if $n$ is an odd integer, then $n^2$ leaves a remainder of 1 when divided by
4. Once you've conjectured this, all that's left is to recall what it means for
a number to be even or odd, and then the proof falls out quite naturally:
\begin{proof}
  Let the perfect square be $n^2$. We split into cases depending on the parity
  of $n$.
  \begin{itemize}
    \item If $n$ is even, let $n=2m$ for some integer $m$. Then
      \[n^2=(2m)^2=4m^2,\] which leaves a remainder of 0 when divided by 4.
    \item If $n$ is odd, let $n=2m+1$ for some integer $m$. Then
      \[n^2=(2m+1)^2=4m^2+4m+1=4(m^2+m)+1,\] which leaves a remainder of 1 when
      divided by 4.
  \end{itemize}
  In either case, the remainder left when dividing $n^2$ by 4 is either 0 or 1,
  which is what we wanted to prove.
\end{proof}
\hyperlink{r:m:2:1}{Result 2}
\hypertarget{s:m:2:1}
The key here is to assume, for contradiction, that there are
only finitely many primes. Then we want to prove a suitable contradiction --- a
nice way of doing this is to find a number that isn't 1 but isn't divisible by
any of our finitely many primes. The idea of constructing such a number by
multiplying everything and adding 1 is surprisingly common in Olympiad maths.
\begin{proof}
  Assume that there are only finitely many primes $p_1,p_2,\ldots,p_n$. Then,
  consider the number $A=p_1p_2\cdots p_n+1$. Clearly $A$ is a positive integer
  larger than 1, so it must have a prime factor, which means that for some $i$,
  $p_i\mid A$. But $p_i\mid A-1$, so $p_i\mid A-(A-1)=1$, which is our
  contradiction.
\end{proof}
\hyperlink{p:m:3:1}{Problem 1}
\hypertarget{s:m:3:1}
This is more an exercise in logic than in maths.
You could be stuck for ages trying to prove the problem directly, but as soon as
you try to use some indirect approach (like contrapositive, contradiction, or
assuming one part of the conclusion is false and proving the other is true) the
problem pretty much solves itself. I think the cleanest solution in this
particular case uses the contrapositive.
\begin{proof}
  We prove the contrapositive: that if $a$ and $b$ are both rational, then so is
  $a+b$.

  Let $a=\frac wx,\ b=\frac yz$. Then \[a+b=\frac wx+\frac
  yz=\frac{wz+xy}{xz},\] which is clearly rational.
\end{proof}
\hyperlink{r:m:4:1}{Result 3}
\hypertarget{s:m:4:1}
This is a classic induction problem. Apart from being instructive because it
isolates the idea of induction, it does highlight a minor point. In the
inductive step, it's just as acceptable to assume the problem is true for $k$
and prove it for $k+1$ as to assume the problem is true for $k-1$ and prove it for
$k$. In this particular case, the latter is somewhat easier.
\begin{proof}
  We prove this by induction on $n$.
  
  Base case $n=1$: We have $\LHS=1=\frac{1\times 2}2=\RHS$.

  Inductive step: Assume the problem is true for $n=k-1$. Then,
  \begin{align*}
    1+2+\cdots+k&=(1+2+\cdots+k-1)+k \\
                &=\frac{k(k-1)}2+k \\
                &=\frac{k(k-1)+2k}2 \\
                &=\frac{k(k+1)}2,
  \end{align*}
  so the problem is true for $n=k$.
\end{proof}
\hyperlink{r:m:4:2}{Result 4}
\hypertarget{s:m:4:2}
Since this is an ``if and only if'' problem, we will probably need to find
separate proofs in each direction. 

First, let's use induction to prove well-ordering. Our desired conclusion is
that every nonempty set of positive integers has a smallest element. Intuitively,
what we would like to do is to check if 1 is in it, then if 2 is in it, and so
on until we first find an element that's in it. But this quickly becomes circular
and it's difficult to make airtight. The trick is to utilise proof by
contrapositive --- start with a set of positive integers that has no smallest
element, and prove it's empty using our sequential checking process. Make sure
you actually use PMI somewhere, otherwise it's probably a fakesolve.

Now, let's use well-ordering to prove PMI\@. You'll probably get nowhere if you
aren't completely clear about what you're trying to prove (you may as well
replace PMI by gibberish), so let's write out PMI in full:

Let $S$ be a set of positive integers. If $1\in S$, and if the statement
$\forall a\in S,\ a+1\in S$ is true, then $S$ contains all positive integers.

Once again we use indirect proof --- this time it's proof by contradiction
(contrapositive also works). Let's assume that there is a set $S$ that satisfies
both conditions but doesn't contain all positive integers. We hope to use
well-ordering to find a contradiction.

The key idea is to consider the smallest integer that isn't in $S$, which is
possible by well-ordering. Then the condition implies that either it's not the
smallest, or $1\not\in S$ --- a contradiction either way.

Let's write it up.
\begin{proof}
  First we prove that if PMI is true, then so is well-ordering. Assume PMI, and
  we'll prove the contrapositive of well-ordering: that if $S$ is a set of
  positive integers with no smallest element, then it is empty.

  I prove by strong induction that for each positive integer $n$, $n\not\in S$.
  Clearly this is sufficient.

  Base case $n=1$: if $1\in S$, then 1 would be the smallest element in $S$.
  So since $S$ has no smallest element, 1 is not the smallest element in $S$ so
  1 is not in $S$. (Notice where I used the contrapositive here?)

  Strong inductive step: Assume that for all $i=1,2,\ldots,k,\ i\not\in S$. I
  claim that $k+1\not\in S$. Indeed, if $k+1$ were in $S$, then it would be the
  smallest element of $S$. But since $S$ has no smallest element, $k+1$ can't be
  in $S$.

  This completes the induction, so no integer is in $S$ meaning that $S$ is
  empty as needed.

  Now I prove that if well-ordering is true, then so is PMI\@. Assume for
  contradiction that well-ordering is true but PMI is not. Then, there is a set
  $S$ of positive integers that contains 1 and such that for each $a\in S,\
  a+1\in S$ but that does not contain all positive integers. Then, the set
  $\Nn\setminus S$ is nonempty so by well-ordering it contains a smallest
  element $a$.

  Since we know that $1\in S$, we know that $a\ne 1$. So $a-1$ is a positive
  integer. Since $a-1<a$ and $a$ is the smallest member of $\Nn\setminus S$,
  $a-1\not\in\Nn\setminus S\implies a-1\in S$. So since $a-1\in S,\ a\in S$
  which contradicts the assumption that $a\not\in S$.
\end{proof}
\hyperlink{p:b:a:systems:1}{Problem 2}
\hypertarget{s:b:a:systems:1}
Not much to say here --- interpret as a system of linear equations and solve
however you like. 

Answer: 26.
\begin{proof}
  Let the two numbers be $a$ and $b$, with $a>b$. Then, $b=a-20$ so
  \begin{align*}
    a+4&=3(a-20+4) \\
       &=3a-48 \\
    2a&=52 \\
     a&=26,
 \end{align*}
 so the larger of the two numbers is 26.

 To prove that this actually works, note that if $a=26$ and $b=6$, then $a-b=20$
 and $a+4=30=3\times 10=3(b+4)$ as needed.
\end{proof}
\hyperlink{p:b:a:systems:2}{Problem 3}
\hypertarget{s:b:a:systems:2}
Since we don't like the 1s in our equations, we subtract two equations to get
rid of them. Alternatively, we subtract two equations because that's one of the
most obvious things to do with a system of equations. Either way, once we've
done that the rest of the problem is pretty routine.

Answer: $(x,y,z)=(1,1,1),\left(-2,-2,\frac52\right),\left(-2,\frac52,-2\right),
\left(\frac52,-2,-2\right)$.
\begin{proof}
  Subtract the third equation fron the first:
  \begin{align*}
    xy-xz&=2z-2y \\
    x(y-z)+2(y-z)&=0 \\
    (x+2)(y-z)&=0 \\
  \end{align*}
  So either $x=-2$ or $y=z$. Similarly we can deduce that either $z=-2$ or
  $x=y$. Now we split into four cases:

  \begin{itemize}
    \item $x=-2,\ z=-2$. Then $2y=zx+1=5\implies y=\frac 52$.
    \item $x=-2,\ x=y$. Then similar to the above we get $z=\frac 52$.
    \item $y=z\ z=-2$. In the same way we get $x=\frac 52$.
    \item $y=z,\ x=y$. Then $x^2+1=2x\implies (x-1)^2=0\implies x=1$, so
      $x=y=z=1$.
  \end{itemize}
  So the only solutions are what we claim they are. It is easy to check that
  these solutions all satisfy the original equations.
\end{proof}
\hyperlink{r:b:a:quad:1}{Result 5}
\hypertarget{sr:b:a:quad:1}
First, there are a few ways of seeing that this is the answer.

One way is to try to factorise $x^2-2mx+p$ as $(x-a)(x-b)$. Then $a+b=2m$ and
$ab=p$, and $a$ and $b$ are the values of $x$ we want. Then the key idea is that
the way of using the $a+b=2m$ condition is to let $a=m+c,\ b=m-c$ so that
\[p=ab=(m+c)(m-c)=m^2-c^2,\] so that $c=\sqrt{m^2-p}$.

Another way is to notice that $x^2-2mx+p$ looks a lot like
$x^2-2mx+m^2=(x-m)^2$. I'll do the rest in the proof.

\begin{proof}
  We have
  \begin{align*}
    x^2-2mx+p&=x^2-2mx+m^2+p-m^2 \\
             &=(x-m)^2-\left(m^2-p\right)
  \end{align*}
  If $m^2-p<0$ then clearly there are no solutions. Otherwise, we have
  \begin{align*}
    x^2-2mx+p&=(x-m)^2-\left(\sqrt{m^2-p}\right)^2\\
             &=\left(x-m-\sqrt{m^2-p}\right)\left(x-m+\sqrt{m^2-p}\right),
  \end{align*}
  so it equals 0 if and only if $x=m\pm\sqrt{m^2-p}$.
\end{proof}
\hyperlink{r:b:a:quad:2}{Result 6}
\hypertarget{sr:b:a:quad:2}
The key here is to get this equation into a form such that we can apply the
previous result.
\begin{proof}
  We have
  \begin{align*}
    0&=ax^2+bx+c \\
    0&=x^2+\frac bax+\frac ca \\
     &=x^2-2\frac{-b}{2a}x+\frac ca \\
    x&=\frac{-b}{2a}\pm\sqrt{\frac{b^2}{4a^2}-\frac ca} \\
     &=\frac{-b\pm \sqrt{4a^2\left(\frac{b^2}{4a^2}-\frac ca\right)}}{2a} \\
     &=\frac{-b\pm\sqrt{b^2-4ac}}{2a}
  \end{align*}
  as needed.

  If $\Delta<0$ there are clearly no solutions. If
  $\Delta=0$ the unique solution is $x=\frac{-b}{2a}$. If $\Delta>0$ there are
  two solutions given by our equation.
\end{proof}
\hyperlink{p:b:a:quad:1}{Problem 4}
\hypertarget{sp:b:a:quad:1}
Once again not much to say here --- interpret the number bases  in the usual way
and solve the resulting quadratic.

Answer: 57.
\begin{proof}
  We have
  \begin{align*}
    111_b&=212_{b-2} \\
    b^2+b+1&=2(b-2)^2+(b-2)+2 \\
           &=2b^2-7b+8 \\
          0&=b^2-8b+7 \\
          b&=4\pm\sqrt{4^2-7} \\
           &=4\pm3
  \end{align*}
  But since $b>2$ we get $b=7\implies x=7^2+7+1=57$.

  Finally, 57 is indeed 111 in base 7 and 212 in base 5.
\end{proof}
\hyperlink{p:b:a:quad:2}{Problem 5}
\hypertarget{sp:b:a:quad:2}
The key here is to use the discriminant ($\Delta$ in
\hyperlink{r:b:a:quad:2}{Result 6}). In particular,
it's enough to prove that at least one of the two $\Delta$s is nonnegative. The
easiest way of doing this is to assume that the first is negative and prove that
the second isn't.
\begin{proof}
  Assume that there is no real number $x$ such that $x^2+(r+1)x+s=0$. Then the
  discriminant $(r+1)^2-4s$ is negative, so $4s>(r+1)^2$.
 
  Since $(r+1)^2\ge 0$, we know that $s>0$.
  Also, $4(s-r)>(r+1)^2-4r=(r-1)^2\ge 0$.
  So since $s>0$ and $4(s-r)>0$, their product $4s^2-4sr$ is also positive so
  the discriminant of the second quadratic is positive, meaning that it has at
  least one real solution.
\end{proof}
\hyperlink{p:b:a:quad:3}{Problem 6}
\hypertarget{sp:b:a:quad:3}
At first glance, this looks like our methods can't help since we have a cubic
not a quadratic. However, the same trick used in \hyperlink{r:b:a:quad:1}{Result
5} of recognising a common factorisation does in fact work.

Answer: $x=-1-\sqrt[3]{4}$.
\begin{proof}
  We subtract 4 from both sides to make the LHS into something we recognise:
  \begin{align*}
    x^3+3x^2+3x+1&=-4 \\
    (x+1)^3&=-4 \\
    x+1&=-\sqrt[3]{4} \\
    x&=-1-\sqrt[3]{4}.
  \end{align*}
  To show that this number works, we can either substitute it in and do the
  algebra, or notice that each step above was actually an equivalence so the
  implications run backwards as well.
\end{proof}

\hyperlink{r:b:a:ineq:1}{Result 7}
\hypertarget{sr:b:a:ineq:1}
Since we want to use the fact that squares are nonnegative, we collect all the
terms on one side. The rest is recognition, which can be helped by noticing that
we want equality to occur when $a=b$.
\begin{proof}
  \[\RHS-\LHS=\frac{a+b}2-\sqrt{ab}=\frac{a+b-2\sqrt{ab}}2=\frac{(\sqrt a-\sqrt
  b)^2}2\ge 0,\] as needed.
\end{proof}

\hyperlink{p:b:a:ineq:1}{Problem 7}
\hypertarget{sp:b:a:ineq:1}
We have to have a 0 and a 2015 in the set, but apart from them the rest of the
terms should be as small as possible. This means that we can apply
\hyperlink{r:m:4:1}{Result 3} to get a function we want to minimise. A little
algebraic trickery means it's enough to minimise \[n+\frac{4032}n.\] Then, by
\hyperlink{r:b:a:ineq:1}{Result 7}, the  minimum of this over $\Rr$ is
$2\sqrt{4032}$ at $n=\sqrt{4032}\approx 63.5$, which means either 63 or 64
should minimise the expression over $\Nn$. In fact both do, which means we should
try to force the expression into something that looks like $(n-63)(n-64)$, and
indeed doing that solves the problem.

Answer: 62.
\begin{proof}
  Let $n$ be the number of elements in $S$, and let $S=\{s_1,s_2,\ldots,s_n\}$,
  where the $s_i$s are in increasing order.
  Then \[s_i\ge i-1\, \forall\, i<n,\] and $s_n=2015$, so the average is at
  least
  \begin{align*}
    \frac{0+1+\cdots+n-2+2015}n&=\frac{\frac{(n-2)(n-1)}2+2015}n \\
                              &=\frac{n^2-3n+4032}{2n}\\
                              &=\frac{n^2-127n+4032}{2n}+62 \\
                              &=\frac{(n-63)(n-64)}{2n}+62.
  \end{align*}
  Since $n$ is an integer, the first term is 0 if $n$ is either 63 or 64 and
  positive otherwise, which means that the minimum value is 62, achieved when
  $S$ is either $\{0,1,\ldots,61,2015\}$ or $\{0,1,\ldots,61,62,2015\}$.
\end{proof}
\hyperlink{r:b:a:sums:1}{Result 8}
\hypertarget{sr:b:a:sums:1}
There are three ways I know of doing this. One of them is a standard induction,
but the other two are more interesting.

For the first way, we notice that we already know the special case (see
\hyperlink{r:m:4:1}{Result 3}) where $a=0$ and $b=1$. A little algebra allows us
to reduce the whole problem to this particular case.
\begin{proof}
  We have
  \begin{align*}
    \sum_{i=0}^n(a+bi)&=\sum_{i=0}^n a+\sum_{i=0}^n bi \\
                      &=a(n+1)+b\sum_{i=0}^n i \\
                      &=a(n+1)+b\frac{n(n+1)}2 \\
                      &=\frac{(2a+bn)(n+1)}2,
  \end{align*}
  as needed.
\end{proof}
For the second way, we use a trick called \emph{Gaussian pairing} --- we pair the
first term with the last term and so on --- so that each pair has the same
sum.
\begin{proof}
  We have
  \begin{align*}
    \sum_{i=0}^n(a+bi)&=\sum_{i=0}^n(a+b(n-i)) \\
                      &=\frac12\left(\sum_{i=0}^n(a+bi)+\sum_{i=0}^n(a+b(n-i))\right)
                      \\
                      &=\frac12\left(\sum_{i=0}^n(2a+bn)\right) \\
                      &=\frac{(n+1)(2a+bn)}2,
  \end{align*}
  as needed.
\end{proof}
\hyperlink{r:b:a:sums:2}{Result 9}
\hypertarget{sr:b:a:sums:2}
The main idea here comes from extending a couple of the common factorisations:
\begin{itemize}
  \item $1-r^2=(1-r)(1+r)$
  \item $1-r^3=(1-r)(1+r+r^2)$
\end{itemize}
Let's write it up.
\begin{proof}
  We have \[1-r^{n+1}=(1-r)(1+r+r^2+\cdots+r^n),\] since all the middle terms
  cancel. Dividing both sides by $1-r$ yields the desired result.
\end{proof}
\hyperlink{r:b:a:polys:1}{Result 10}
\hypertarget{sr:b:a:polys:1}
For those of you who have seen polynomial long division, this should seem very
familiar, and indeed the same idea of dividing from highest to lowest degree
works. The neatest and most rigorous way to write it up uses induction.
\begin{proof}
  By strong induction on the degree.

  Base case $n=1$: if $ar-b=0$ then $\frac ba=r$ so $ax-b=a(x-\frac ba)=a(x-r)$,
  as needed. Clearly $Q(x)=a$ has degree 0.

  Inductive step: Write \[P(x)=a_n x^{n-1}(x-r)+P_1(x),\] so that $P_1(x)$ is a
  polynomial of degree less than $n$ and $P_1(r)=P(r)-0=0$.
  By the strong inductive hypothesis there
  is a polynomial $Q_1(x)$ of degree less than $n-1$ such that
  $P_1(x)=(x-r)Q_1(x)$. Then, \[P(x)=(x-r)\left(a_n x^{n-1}+Q_1(x)\right),\] so
  if we let $Q(x)=x^{n-1}+Q_1(x)$ we have $P(x)=(x-r)Q(x)$. Finally, $Q(x)$ has
  degree $n-1$ since $Q_1(x)$ has degree less than $n-1$ and $a_n x^{n-1}$ has
  degree $n-1$.
\end{proof}
\hyperlink{r:b:a:polys:2}{Result 11}
\hypertarget{sr:b:a:polys:2}
First we need to understand what the problem is saying. Try special cases:
$i=0,\ i=n,\ n=1,\ n=2$ and so on, until you know what it's saying.

The idea here is to use the previous result multiple times to factorise our
polynomial fully, then expand it again. Then when we extract the $x^{n-i}$
coefficient, each term that contributes to it is a product where $x$ appears
$n-i$ times, and the rest is a product of $i$ $(r_i)$s and a constant term.
Since each combination of $i$ $(-r_i)$s appears exactly once in the expansion,
we get the claimed formula.

The neatest way of writing this up is to use induction.
\begin{proof}
  By induction on the degree.

  Base case $n=0$: there are no $r_i$s so all that
  we have to prove is $a_0=a_0$, which is obvious.

  Inductive step: Since $r_n$ is a root of $P(x)$, we can write
  $P(x)=(x-r_n)Q(x)$ for some polynomial $Q(x)$ of degree $n-1$. Then we know
  that $Q(x)$ has roots $r_1,\ldots,r_{n-1}$ so by the inductive hypothesis, we
  know that in $Q(x)$:
  \begin{itemize}
    \item The coefficient of $x^{n-i}$ is
      \[(-1)^{i-1}a_n\sum_{j_1<\cdots<j_{i-1}}\prod_{k=1}^{i-1}r_{j_k}.\]
    \item The coefficient of $x^{n-i-1}$ is
      \[(-1)^{i}a_n\sum_{j_1<\cdots<j_i}\prod_{k=1}^i r_{j_k}.\]
  \end{itemize}
  Since the coefficient of $x^{n-i}$ in $P(x)$ is the coefficient of $x^{n-i-1}$
  in $Q(x)$ minus $r_n$ times the coefficient of $x^{n-i}$ in $Q(x)$, it's what
  we claim it is. Since this argument works for each $i$, the induction is
  complete.
\end{proof}
\newpage
\part{TLDR}
\hypertarget{tldr}
This part contains some ideas about how to approach problems; basically,
I've taken what I think are the most important bits and condensed them together
here.

(With apologies to P\'olya)
\begin{itemize}
  \item How are the objects referred to in the problem defined? What properties
    do you know them to have?
  \item Try small or special cases. Can you spot patterns in their structure? In
    how you solve them? Can you prove any of these patterns in general? Do any
    of these patterns help?
  \item Look at stuff that is extremal in some way: biggest, smallest, most
    connected, most disconnected, most composite, prime, whatever
  \item Think about what happens if the problem, or the conclusion, is wrong.
  \item Can you reduce any instance of the problem to a smaller instance? 
    Can you reduce a counterexample to a smaller counterexample?
  \item Have you seen something similar before? Can you use the result or the
    method? Can you introduce some auxiliary element to make its use possible?
  \item Can you draw a diagram to help you understand the problem?
\end{itemize}
\end{document}
