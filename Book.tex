%chktex-file 3
\documentclass{amsart}
\usepackage{amsmath,amssymb,amsthm,tcolorbox,hyperref,booktabs}
\hypersetup{colorlinks=true}
\newtheorem{problem}{Problem}[subsubsection]
\newtheorem{result}{Result}[subsubsection]
\newcommand{\LHS}{\text{LHS}}
\newcommand{\RHS}{\text{RHS}}
\newcommand{\Rr}{\mathbb{R}}
\newcommand{\Nn}{\mathbb{N}}
\title{Some Madman's Ravings}
\author{gispisquared}
\begin{document}
\maketitle
\tableofcontents
\newpage
\section*{Introduction}
I was recently\footnote{February 2022} asked what advice I would give to a
beginner Olympiad student. Partly for your sake if you're reading this, and
partly to inform how I write this, I'll copy (and punctuate) my response here:
\begin{itemize}
  \item Don't believe people (like whoever writes
    \href{https://how-did-i-get-here.com/olympiad/}{hdigh}) when they say methods
    of proof is all you should learn before spamming problems. Maybe that works
    if you're as smart as them, but most of us don't wanna spend most of our
    training reinventing the wheel.
  \item On the other hand, don't waste time looking up the latest crazy theorem
    aopsers start mentioning; those are unlikely to ever be necessary and are
    probably only rarely useful because problem setters and pscs prefer problems
    where the easiest solution involves only canonical Olympiad knowledge.
  \item Most of the problems you're trying right now are gonna be pretty
    trivial for more advanced students, so whenever you solve or
    don't solve a problem figure out why it should have been trivial and
    hopefully this hindsight will start turning into foresight.
\end{itemize}

If I am to be consistent, then, this book's contents should be of three main
types: canonical Olympiad knowledge and techniques, problems on which to
practice these techniques, and solutions with motivation that try to make the
problems seem as trivial as possible. I've also chosen both to include proofs to
the known results (since their proofs are within Olympiad students' grasp and
are often the source of canonical techniques), and to separate these proofs from
the theorem statements in the same way as with problems (so that
students can try to prove the theorems themselves).

There will be two main types of questions:
\begin{itemize}
  \item \emph{Results}, which are considered important and well-known, and come
    up sporadically (or in some cases consistently) as steps in the harder
    problems.
  \item \emph{Problems}, which will be questions taken from contests (mostly AMC
    and AIMO in Part 2, and AMO, EGMO and IMO in Part 3).
\end{itemize}

Since I refuse to rehash stuff that others have done better, I'll refer you to a
couple of resources about how to write proofs properly:
\begin{itemize}
  \item
    \href{https://artofproblemsolving.com/news/articles/how-to-write-a-solution}
    {How to Write a Maths Solution}
  \item \href{https://web.evanchen.cc/handouts/english/english.pdf}{Notes on
    English}
\end{itemize}
Cool, hopefully now you know how to write proofs. Guess that means every time 
you solve a problem you'll get a 7, right?
\newpage
\part{General Recommendations}
This part contains some general ideas about how to approach problems, or what to
try when stuck.

(With apologies to P\'olya)
\begin{itemize}
  \item How are the objects referred to in the problem defined? What properties
    do you know them to have?
  \item Try small or special cases. Can you spot patterns in their structure? In
    how you solve them? Can you prove any of these patterns in general? Do any
    of these patterns help?
  \item Look at stuff that is extremal in some way: biggest, smallest, most
    connected, most disconnected, most composite, prime, whatever
  \item Think about what happens if the problem, or the conclusion, is wrong.
  \item Can you reduce any instance of the problem to a smaller instance? 
    Can you reduce a counterexample to a smaller counterexample?
  \item Have you seen something similar before? Can you use the result or the
    method? Can you introduce some auxiliary element to make its use possible?
  \item Can you draw a diagram to help you understand the problem?
\end{itemize}
\newpage
\part{Theory}
\setcounter{section}{-1}
\section{Notation}
\begin{center}
  \begin{tabular}{cc}
    \toprule
    Notation & Meaning \\ \midrule

    $\Rr$ & The real numbers \\

    \bottomrule
  \end{tabular}
\end{center}
\newpage
\section{Basics}
\subsection{Methods of Proof}
\subsubsection{Direct Proof}
This is perhaps the simplest type of proof. The idea is to start with the stuff
you're given, do some logical deduction, and finish with what you want to prove.
\begin{result}\label{r:m:1:1}
  The remainder when a perfect square is divided by 4 is either 0 or 1.
  \hyperlink{s:m:1:1}{Solution}
\end{result}
\subsubsection{Contradiction}
This is where you assume that what you're trying to prove is wrong and try to
derive some kind of logical impossibility. Then the only place where the logic
could have gone wrong was in the assumption so the statement you were trying to
prove must be true.
\begin{result}\label{r:m:2:1}
  There are infinitely many primes.
  \hyperlink{s:m:2:1}{Solution}
\end{result}
\subsubsection{Contrapositive}
It turns out that the statement $A\implies B$ is logically equivalent to the
statement $\neg B\implies\neg A$. This is probably easiest to see intuitively
with an result: ``If $x$ is an integer, then $x$ is rational'' is
logically equivalent to ``If $x$ is not rational, then $x$ is not an integer''.
Therefore, if we're asked to prove $A\implies B$, it's enough to
prove $\neg B\implies\neg A$, which is sometimes easier.
\begin{problem}\label{p:m:3:1}
  Let $a,b\in\Rr$ such that $a+b$ is irrational. Prove that at least one of $a$
  and $b$ is irrational.
  \hyperlink{s:m:3:1}{Solution}
\end{problem}
\subsubsection{Induction}
Perhaps the hardest to understand of the basic proof techniques, this can be
used to prove properties of positive integers where the property for each
integer can be related to those of previous integers.

Here is the Principle of Mathematical Induction (PMI):
\begin{tcolorbox}
  Let $S$ be a set of positive integers such that $1\in S$ and for each $k\in S$,
  $k+1\in S$. Then $S$ contains all positive integers.
\end{tcolorbox}

To prove a statement for all positive integers, we let $S$ be the set of all
positive integers for which the statement is true. Then it's enough to prove:
\begin{itemize}
  \item $1\in S$. This is called the \emph{base case}.
  \item If $k\in S$ (the \emph{inductive hypothesis}), then $k+1\in S$. This is
    called the \emph{inductive case}.
\end{itemize}
Then by PMI, $S$ will contain all positive integers.

There are two ways to make induction superficially more powerful, though they're
both equivalent to the usual form of induction:
\begin{itemize}
  \item Say we want to prove a statement for all integers larger than $n$, for
    some $n$. Then it's enough to prove:
    \begin{itemize}
      \item The statement is true for $n$.
      \item If the statement is true for some integer $k>n$, then it's true for
        $k+1$.
    \end{itemize}
    This is equivalent to the normal PMI\@: to see this, let $S$ be the set of
    all integers $m$ for which the statement is true for all $m+n$.
  \item Say we want to use not just the inductive assumption not just for $k$,
    but for smaller integers as well. Intuitively this should be fine, since
    we've in some sense ``proved this already'' by the time we get to $k+1$.
    Formally, to prove a statement $P(n)$ for all positive integers $n$, it's
    enough to prove:
    \begin{itemize}
      \item $P(1)$.
      \item If $P(1),\ldots,P(k)$ are all true, then $P(k+1)$ is also true.
    \end{itemize}
    This form of proof by induction is called \emph{strong induction}, and
    although most proofs by induction only explicitly use $P(k)$, there's no
    reason to try to make your proof inductive over strong inductive since
    strong induction gives you more assumptions to work with ``for free''.
\end{itemize}
The key idea in both of these reductions to PMI is to somehow encapsulate the
extra information you're trying to assume into the framework of standard PMI\@.
\begin{result}\label{r:m:4:1}
  For all positive integers $n$, \[1+2+\cdots+n=\frac{n(n+1)}2.\]
  \hyperlink{s:m:4:1}{Solution}
\end{result}
To conclude the Methods of Proof section, I'll include one final result that
combines most of what we've covered so far.
\begin{result}\label{r:m:4:2}
  The \emph{Well-Ordering Principle} states that any set of positive integers
  has a least element. It's equivalent to PMI --- that is,
  PMI is true if and only if well-ordering is true.
  \hyperlink{s:m:4:2}{Solution}
\end{result}
I find it intriguing that induction and minimality are really just two sides of
the same coin. Often you will find that a solution is much more natural to think
about and write up in terms of one than the other.
\subsection{Algebra}
\subsubsection{Factorisations}
I won't have any problems attached to these, but they tend to pop up everywhere
so keep an eye out. Here are some common factorisations:
\begin{itemize}
  \item $x^2-a^2=(x+a)(x-a)$
  \item $x^2-2ax+a^2=(x-a)^2$
  \item $x^2+2ax+a^2=(x+a)^2$
  \item $x^3-a^3=(x-a)(x^2+ax+a^2)$
  \item $x^3+a^3=(x+a)(x^2-ax+a^2)$
  \item $x^3+3x^2a+3xa^2+a^3=(x+a)^3$
  \item $x^3-3x^2a+3xa^2-a^3=(x-a)^3$
\end{itemize}
The cases $a=1$ are especially common.
\subsubsection{Systems of equations}
There are a couple of ways of solving these systems --- either you can
isolate one variable, substitute into the rest of the equations, and repeat, or
you can try and combine the equations in such a way that stuff cancels. The
first method is usually fine in school maths and the AMC, but the second is more
likely to be useful in harder Olympiad questions.

Sometimes these techniques won't be enough --- see Section~\ref{n:a:systems}.
\begin{problem}\label{p:b:a:systems:1}
  The difference between two numbers is 20. When 4 is added to each number the
  larger is three times the smaller. What is the larger of the two original
  numbers?
  \hyperlink{s:b:a:systems:1}{Solution}
\end{problem}
\begin{problem}\label{p:b:a:systems:2}
  Find all triples $(x,y,z)$ of real numbers that simultaneously satisfy the
  equations
  \begin{align*}
    xy+1&=2z \\
    yz+1&=2x \\
    zx+1&=2y \\
  \end{align*}
  \hyperlink{s:b:a:systems:2}{Solution}
\end{problem}
\subsubsection{Quadratics}
\begin{result}\label{r:b:a:quad:1}
  Let $m$ and $p$ be given real numbers. All real numbers $x$ such that
  \[x^2-2mx+p=0\] are given by $x=m\pm\sqrt{m^2-p}$.
  \hyperlink{sr:b:a:quad:1}{Solution}
\end{result}
\begin{result}\label{r:b:a:quad:2}
  Let $a,\ b,\ c$ be given real numbers. All real numbers $x$ such that
  \[ax^2+bx+c=0\] are given by \[x=\frac{-b\pm\sqrt{b^2-4ac}}{2a}.\]
  In particular, if we let $\Delta=b^2-4ac$, then the equation has no real roots
  if $\Delta<0$, exactly one real root if $\Delta=0$, and two real roots if
  $\Delta>0$.
  \hyperlink{sr:b:a:quad:2}{Solution}
\end{result}
This number $\Delta$ is called the \emph{discriminant} of the quadratic.

Now, a couple of problems which show how useful both the results and the method
are.
\begin{problem}\label{p:b:a:quad:1}
  [See Section~\ref{b:n:bases} if you don't know what number bases are.]

  The number $x$ is 111 when written in base $b$, but it is 212 when written in
  base $b-2$. What is $x$ is base 10?
  \hyperlink{sp:b:a:quad:1}{Solution}
\end{problem}
\begin{problem}\label{p:b:a:quad:2}
  For each pair of real numbers $(r,s)$, prove that there exists a real numer
  $x$ that satisfies at least one of the following two equations.
  \begin{align*}
    x^2+(r+1)x+s&=0 \\
    rx^2+2sx+s&=0
  \end{align*}
  \hyperlink{sp:b:a:quad:2}{Solution}
\end{problem}
\begin{problem}\label{p:b:a:quad:3}
  Find all real numbers $x$ for which $x^3+3x^2+3x+5=0$.
  \hyperlink{sp:b:a:quad:3}{Solution}
\end{problem}
\subsubsection{Inequalities}
At this level, inequalities are mostly about making stuff into squares or, well,
``mostly squares''. The guiding principle is to try and find an expression which
you want to be always nonnegative, figure out where it's 0, and write it in
terms of stuff that's 0 there and obviously nonnegative elsewhere.

\begin{result}\label{r:b:a:ineq:1}
  If $a$ and $b$ are real numbers, then \[\frac{a+b}2\ge\sqrt{ab}.\]
  \hyperlink{sr:b:a:ineq:1}{Solution}
\end{result}

\begin{problem}\label{p:b:a:ineq:1}
  The set $S$ consists of distinct integers such that the smallest is 0 and the
  largest is 2015. What is the minimum possible average value of the numbers in
  $S$?
  \hyperlink{sp:b:a:ineq:1}{Solution}
\end{problem}
\subsubsection{Sums of sequences}
\begin{result}\label{r:b:a:sums:1}
  If $n$ is a positive integer and $a$ and $b$ are real numbers, then
  \[\sum_{i=0}^n(a+bi)=\frac{(n+1)(2a+bn)}2.\]
  \hyperlink{sr:b:a:sums:1}{Solution}
\end{result}
\begin{result}\label{r:b:a:sums:2}
  If $n$ is a positive integer and $r$ is a real number distinct from 1, then
  \[\sum_{i=0}^n r^i=\frac{1-r^{n+1}}{1-r}.\]
  \hyperlink{sr:b:a:sums:2}{Solution}
\end{result}
\begin{problem}\label{p:b:a:sums:1}
  Prove that for any positive integers $n$ and $a$, the sum
  \[n+(n+1)+(n+2)+\cdots+(n+a)\] is never a power of 2.
  \hyperlink{sp:b:a:sums:1}{Solution}
\end{problem}
\subsection{Combinatorics}
\subsubsection{Addition and Multiplication Principles}
This is the most basic idea in combinatorics. If you can make one from $a$
choices and then one from $b$ choices, the total number of ways you can do this
is $ab$. If you can make either one from $a$ choices or one from $b$ choices,
the total number of ways you can do this is $a+b$.

Many easier combinatorics problems boil down to splitting them into cases, and
then applying addition and multiplication principles to count each case.
\begin{result}\label{r:b:c:am:1}
  The number of ways of choosing $k$ things from $n$, where order matters, is
  \[n(n-1)\cdots(n-k+1)=\frac{n!}{k!}.\]
  \hyperlink{sr:b:c:am:1}{Solution}
\end{result}
\begin{problem}\label{p:b:c:am:1}
  A hockey game between two teams is `relatively close' if the numbers of goals
  scored by the two teams never differ by more than two. In how many ways can
  the first 12 goals of a game be scored if the game is `relatively close'?
  \hyperlink{sp:b:c:am:1}{Solution}
\end{problem}
\subsubsection{Combinations}
\begin{result}\label{r:b:c:c:1}
  The number of ways of choosing $k$ things from $n$, where order doesn't
  matter, is \[\frac{n!}{(n-k)!(n-k)!}.\]
  \hyperlink{sr:b:c:c:1}{Solution}
\end{result}
\begin{problem}{\label{p:b:c:c:1}}
  How many pairs $(a,b)$ of 3-digit palindromes are there with $a>b$ and
  $a-b$ also a 3-digit palindrome?
  \hyperlink{sp:b:c:c:1}{Solution}
\end{problem}
\subsubsection{Inclusion-Exclusion}
So you try to count something using the addition principle, but you realise
you've counted some things twice. Obviously, the way you fix it is you subtract
the stuff you doubled up. For more tedious problems this could also be
overcounted, but this can be continued and hopefully the stuff you've
overcounted gets easier to count each time.
\begin{problem}{\label{p:b:c:pie:1}}
  How many integers between 1 and 1000 inclusive are divisible by at least one
  of 4, 6 and 10?
\end{problem}
\subsubsection{Bijections}
A bijection is simply a way of counting one set of stuff by identifying each
element in the set with an element from another set that's easier to count.
We've already used a bijection in our solution to Problem~\ref{p:b:c:c:1}.

To come up with a bijection, it's often useful to list out the elements of each
set for some small cases, and pair them up in what looks like a natural way.
Often that natural way will generalise.

\begin{result}{\label{r:b:c:b:1}}
  The number of ways of putting $n$ identical things into $k$ boxes is
  \[\binom{n+k-1}{k-1}.\]
\end{result}
\begin{problem}{\label{p:b:c:b:1}}
  How many paths are there from the bottom left to the top right of a $4\times
  7$ grid, if you can only go up or right?
\end{problem}
\subsection{Geometry}
\subsubsection{Parallels}
You should know what parallel lines are: they point in the same direction and
never meet.

Let $ABC$ and $DEF$ be points in that order along two parallel lines, and let
$X$ and $Y$ be points on $BE$ such that $XBEY$ is in that order.
Then
\[\angle ABX=\angle CBE=\angle DEX=\angle FEY,\] the other four angles are
equal, and the angles in these two sets are supplementary (add up to 
$180^\circ$).
\begin{result}{\label{r:b:g:p:1}}
  For any three points $A, B, C$, \[\angle ABC+\angle
  BCA+\angle CAB=180^\circ.\]
\end{result}
\subsubsection{Congruence}
Two triangles $ABC$ and $XYZ$ are called \emph{congruent} if \[AB=XY,\ AC=XZ,\
BC=YZ,\]
\[\angle BAC=\angle YXZ,\ \angle ABC=\angle XYZ,\ \angle ACB=\angle
XZY.\]
We write ``$ABC$ is congruent to $XYZ$'' as $\triangle ABC\cong\triangle XYZ$,
and in this case we also know that the areas of $ABC$ and $XYZ$ are equal.

There are congruence tests which enable us to determine when two triangles are
congruent, so that knowing some of these equalities we can deduce the others.

Triangles $ABC$ and $XYZ$ are congruent if any of the following hold:
\begin{itemize}
  \item $AB=XY$, $AC=XZ$ and $BC=YZ$. (SSS)
  \item $AB=XY$, $AC=XZ$ and $\angle BAC=\angle YXZ$. (SAS)
  \item $AB=XY$ and $\angle BAC=\angle YXZ$ and $\angle ABC=\angle XYZ$. (AAS)
\end{itemize}
Note: SSA (two sides and an unincluded angle) on its own isn't enough, but we
can fix it:
\begin{itemize}
  \item $AB=XY$ and $AC=XZ$ and $\angle ABC=\angle XYZ$ and $AB<AC$. (Fixed SSA)
\end{itemize}
In the special case where $\angle ABC=\angle XYZ$, this is known as RHS.\@
\begin{result}{\label{r:b:g:c:1}}
  Let $ABC$ be a triangle, and let $M$ be the midpoint of $BC$.
  The following statements are equivalent:
  \begin{itemize}
    \item $AB=AC$
    \item $\angle ABC=\angle ACB$
    \item $AM\perp BC$.
  \end{itemize}
\end{result}
\begin{result}{\label{r:b:g:c:2}}
  Let $ABCD$ be a parallelogram (that is, $AB||CD$ and $BC||DA$). If $AC$
  intersect $BD$ at $P$, then $P$ is the midpoint of $AC$ and of $BD$.
\end{result}
\subsubsection{Circles}
A \emph{circle} is the set of points equidistant from a given point.
\begin{result}{\label{r:b:g:ci:1}}
  If $A,\ B,\ C$ are points on a circle centred at $O$ with $A$ on the same side
  of $BC$ as $O$, then $\angle BOC=2\angle BAC$.
\end{result}
\begin{result}{\label{r:b:g:ci:2}}
  If $A,\ B,\ C$ are points on a circle centred at $O$ with $A$ and $O$ on
  opposite sides of $BC$, then $\angle BOC=360^\circ-2\angle BAC$.
\end{result}
\begin{result}{\label{r:b:g:ci:3}}
  If $B$ and $C$ lie on the same side of $AD$ then there is a circle passing
  through all of $A,\ B,\ C,\ D$ if and only if $\angle ABD=\angle ACD$.
\end{result}
\begin{result}{\label{r:b:g:ci:4}}
  If $B$ and $D$ lie on opposite sides of $AC$, then there is a circle passing
  through all of $A,\ B,\ C,\ D$ if and only if $\angle ABC+\angle
  ADC=180^\circ$.
\end{result}
\subsubsection{Similarity}
\subsubsection{Pythagoras}
\subsubsection{Common right-angled triangles}
\subsection{Number Theory}
\subsubsection{Primes and FTOA}
\subsubsection{SFFT}
\subsubsection{Divisibility}
\subsubsection{Modular Arithmetic}
\subsubsection{Number bases}\label{b:n:bases}
hi
\section{Not-so-basics}
\subsection{Algebra}
\subsubsection{Systems of equations}\label{n:a:systems}
hi
\subsubsection{Polynomials}
A polynomial is just an expression of the form
\[P(x)=a_n x^n+a_{n-1}x^{n-1}+\cdots+a_0x^0,\] where each of the $a_i$s is a
constant and $a_n\ne 0$.
The number $n$ is called the \emph{degree} of the polynomial. The expression
$P(x)=0$ is also a polynomial, and it's defined to have degree $-\infty$.

\begin{result}\label{r:n:a:polys:1}
  If a polynomial $P(x)$ of degree $n$ has a root $r$, then there is a
  polynomial $Q(x)$ of degree $n-1$ such that \[P(x)=(x-r)Q(x).\]
  \hyperlink{sr:n:a:polys:1}{Solution}
\end{result}
\begin{result}\label{r:n:a:polys:2}
  If a polynomial \[P(x)=a_n x^n+a_{n-1}x^{n-1}+\cdots+a_0x^0\] has roots
  $r_1,\ldots,r_n$ then for each $i$, \[a_{n-i}=(-1)^i a_n
  \sum_{j_1<\cdots<j_i} \prod_{k=1}^i r_{j_k}.\]
  \hyperlink{sr:n:a:polys:2}{Solution}
\end{result}
\subsection{Combinatorics}
\subsubsection{Recurrences}
\subsection{Geometry}
\subsection{Number Theory}
\newpage
\part{Practice}
\newpage
\part{Solutions}
\setcounter{section}{0}
\renewcommand*{\theHsection}{sols.\the\value{section}}
\section{Basics}
\subsection{Methods of Proof}
\subsubsection*{Result~\ref{r:m:1:1}}
\hypertarget{s:m:1:1}
If you start trying some small cases, what you'll eventually find is that if $n$
is an even integer, then $n^2$ leaves a remainder of 0 when divided by 4, and
if $n$ is an odd integer, then $n^2$ leaves a remainder of 1 when divided by
4. Once you've conjectured this, all that's left is to recall what it means for
a number to be even or odd, and then the proof falls out quite naturally:
\begin{proof}
  Let the perfect square be $n^2$. We split into cases depending on the parity
  of $n$.
  \begin{itemize}
    \item If $n$ is even, let $n=2m$ for some integer $m$. Then
      \[n^2=(2m)^2=4m^2,\] which leaves a remainder of 0 when divided by 4.
    \item If $n$ is odd, let $n=2m+1$ for some integer $m$. Then
      \[n^2=(2m+1)^2=4m^2+4m+1=4(m^2+m)+1,\] which leaves a remainder of 1 when
      divided by 4.
  \end{itemize}
  In either case, the remainder left when dividing $n^2$ by 4 is either 0 or 1,
  which is what we wanted to prove.
\end{proof}
\subsubsection*{Result~\ref{r:m:2:1}}
\hypertarget{s:m:2:1}
The key here is to assume, for contradiction, that there are
only finitely many primes. Then we want to prove a suitable contradiction --- a
nice way of doing this is to find a number that isn't 1 but isn't divisible by
any of our finitely many primes. The idea of constructing such a number by
multiplying everything and adding 1 is surprisingly common in Olympiad maths.
\begin{proof}
  Assume that there are only finitely many primes $p_1,p_2,\ldots,p_n$. Then,
  consider the number $A=p_1p_2\cdots p_n+1$. Clearly $A$ is a positive integer
  larger than 1, so it must have a prime factor, which means that for some $i$,
  $p_i\mid A$. But $p_i\mid A-1$, so $p_i\mid A-(A-1)=1$, which is our
  contradiction.
\end{proof}
\subsubsection*{Problem~\ref{p:m:3:1}}
\hypertarget{s:m:3:1}
This is more an exercise in logic than in maths.
You could be stuck for ages trying to prove the problem directly, but as soon as
you try to use some indirect approach (like contrapositive, contradiction, or
assuming one part of the conclusion is false and proving the other is true) the
problem pretty much solves itself. I think the cleanest solution in this
particular case uses the contrapositive.
\begin{proof}
  We prove the contrapositive: that if $a$ and $b$ are both rational, then so is
  $a+b$.

  Let $a=\frac wx,\ b=\frac yz$. Then \[a+b=\frac wx+\frac
  yz=\frac{wz+xy}{xz},\] which is clearly rational.
\end{proof}
\subsubsection*{Result~\ref{r:m:4:1}}
\hypertarget{s:m:4:1}
This is a classic induction problem. Apart from being instructive because it
isolates the idea of induction, it does highlight a minor point. In the
inductive step, it's just as acceptable to assume the problem is true for $k$
and prove it for $k+1$ as to assume the problem is true for $k-1$ and prove it
for $k$. In this particular case, the latter is somewhat easier.
\begin{proof}
  We prove this by induction on $n$.

  Base case $n=1$: We have $\LHS=1=\frac{1\times 2}2=\RHS$.

  Inductive step: Assume the problem is true for $n=k-1$. Then,
  \begin{align*}
    1+2+\cdots+k&=(1+2+\cdots+k-1)+k \\
                &=\frac{k(k-1)}2+k \\
                &=\frac{k(k-1)+2k}2 \\
                &=\frac{k(k+1)}2,
  \end{align*}
  so the problem is true for $n=k$.
\end{proof}
\subsubsection*{Result~\ref{r:m:4:2}}
\hypertarget{s:m:4:2}
Since this is an ``if and only if'' problem, we will probably need to find
separate proofs in each direction. 

First, let's use induction to prove well-ordering. Our desired conclusion is
that any nonempty set of positive integers has a smallest element. Intuitively,
what we would like to do is to check if 1 is in it, then if 2 is in it, and so
on until we find an element that's in it. But this quickly becomes circular
and it's difficult to make airtight. The trick is to utilise proof by
contrapositive --- start with a set of positive integers that has no smallest
element, and prove it's empty using our sequential checking process. Make sure
you actually use PMI somewhere, otherwise it's probably a fakesolve.

Now, let's use well-ordering to prove PMI\@. You'll probably get nowhere if you
aren't completely clear about what you're trying to prove (you may as well
replace PMI by gibberish), so let's write out PMI in full:

Let $S$ be a set of positive integers. If $1\in S$, and if the statement
$\forall a\in S,\ a+1\in S$ is true, then $S$ contains all positive integers.

Once again we use indirect proof --- this time it's proof by contradiction
(contrapositive also works). Let's assume that there is a set $S$ that satisfies
both conditions but doesn't contain all positive integers. We hope to use
well-ordering to find a contradiction.

The key idea is to consider the smallest integer that isn't in $S$, which is
possible by well-ordering. Then the condition implies that either it's not the
smallest, or $1\not\in S$ --- a contradiction either way.

Let's write it up.
\begin{proof}
  First we prove that if PMI is true, then so is well-ordering. Assume PMI, and
  we'll prove the contrapositive of well-ordering: that if $S$ is a set of
  positive integers with no smallest element, then it is empty.

  I prove by strong induction that for each positive integer $n$, $n\not\in S$.
  Clearly this is sufficient.

  Base case $n=1$: if $1\in S$, then 1 would be the smallest element in $S$.
  So since $S$ has no smallest element, 1 is not the smallest element in $S$ so
  1 is not in $S$. (Notice where I used the contrapositive here?)

  Strong inductive step: Assume that for all $i=1,2,\ldots,k,\ i\not\in S$. I
  claim that $k+1\not\in S$. Indeed, if $k+1$ were in $S$, then it would be the
  smallest element of $S$. But since $S$ has no smallest element, $k+1$ can't be
  in $S$.

  This completes the induction, so no integer is in $S$ meaning that $S$ is
  empty as needed.

  Now I prove that if well-ordering is true, then so is PMI\@. Assume for
  contradiction that well-ordering is true but PMI is not. Then, there is a set
  $S$ of positive integers that contains 1 and such that for each $a\in S,\
  a+1\in S$ but that does not contain all positive integers. Then, the set
  $\Nn\setminus S$ is nonempty so by well-ordering it contains a smallest
  element $a$.

  Since we know that $1\in S$, we know that $a\ne 1$. So $a-1$ is a positive
  integer. Since $a-1<a$ and $a$ is the smallest member of $\Nn\setminus S$,
  $a-1\not\in\Nn\setminus S\implies a-1\in S$. So since $a-1\in S,\ a\in S$
  which contradicts the assumption that $a\not\in S$.
\end{proof}
\subsection{Algebra}
\subsubsection*{Problem~\ref{p:b:a:systems:1}}
\hypertarget{s:b:a:systems:1}
Not much to say here --- interpret as a system of linear equations and solve
however you like. 

Answer: 26.
\begin{proof}
  Let the two numbers be $a$ and $b$, with $a>b$. Then, $b=a-20$ so
  \begin{align*}
    a+4&=3(a-20+4) \\
       &=3a-48 \\
    2a&=52 \\
    a&=26,
  \end{align*}
  so the larger of the two numbers is 26.

  To prove that this actually works, note that if $a=26$ and $b=6$, then $a-b=20$
  and $a+4=30=3\times 10=3(b+4)$ as needed.
\end{proof}
\subsubsection*{Problem~\ref{p:b:a:systems:2}}
\hypertarget{s:b:a:systems:2}
Since we don't like the 1s in our equations, we subtract two equations to get
rid of them. Alternatively, we subtract two equations because that's one of the
most obvious things to do with a system of equations. Either way, once we've
done that the rest of the problem is pretty routine.

Answer: $(x,y,z)=(1,1,1),\left(-2,-2,\frac52\right),\left(-2,\frac52,-2\right),
\left(\frac52,-2,-2\right)$.
\begin{proof}
  Subtract the third equation fron the first:
  \begin{align*}
    xy-xz&=2z-2y \\
    x(y-z)+2(y-z)&=0 \\
    (x+2)(y-z)&=0 \\
  \end{align*}
  So either $x=-2$ or $y=z$. Similarly we can deduce that either $z=-2$ or
  $x=y$. Now we split into four cases:

  \begin{itemize}
    \item $x=-2,\ z=-2$. Then $2y=zx+1=5\implies y=\frac 52$.
    \item $x=-2,\ x=y$. Then similar to the above we get $z=\frac 52$.
    \item $y=z\ z=-2$. In the same way we get $x=\frac 52$.
    \item $y=z,\ x=y$. Then $x^2+1=2x\implies (x-1)^2=0\implies x=1$, so
      $x=y=z=1$.
  \end{itemize}
  So the only solutions are what we claim they are. It is easy to check that
  these solutions all satisfy the original equations.
\end{proof}
\subsubsection*{Result~\ref{r:b:a:quad:1}}
\hypertarget{sr:b:a:quad:1}
First, there are a few ways of seeing that this is the answer.

One way is to try to factorise $x^2-2mx+p$ as $(x-a)(x-b)$. Then $a+b=2m$ and
$ab=p$, and $a$ and $b$ are the values of $x$ we want. Then the key idea is that
the way of using the $a+b=2m$ condition is to let $a=m+c,\ b=m-c$ so that
\[p=ab=(m+c)(m-c)=m^2-c^2,\] so that $c=\sqrt{m^2-p}$.

Another way is to notice that $x^2-2mx+p$ looks a lot like
$x^2-2mx+m^2=(x-m)^2$. I'll do the rest in the proof.

\begin{proof}
  We have
  \begin{align*}
    x^2-2mx+p&=x^2-2mx+m^2+p-m^2 \\
             &=(x-m)^2-\left(m^2-p\right)
  \end{align*}
  If $m^2-p<0$ then clearly there are no solutions. Otherwise, we have
  \begin{align*}
    x^2-2mx+p&=(x-m)^2-\left(\sqrt{m^2-p}\right)^2\\
             &=\left(x-m-\sqrt{m^2-p}\right)\left(x-m+\sqrt{m^2-p}\right),
  \end{align*}
  so it equals 0 if and only if $x=m\pm\sqrt{m^2-p}$.
\end{proof}
\subsubsection*{Result~\ref{r:b:a:quad:2}}
\hypertarget{sr:b:a:quad:2}
The key here is to get this equation into a form such that we can apply the
previous result.
\begin{proof}
  We have
  \begin{align*}
    0&=ax^2+bx+c \\
    0&=x^2+\frac bax+\frac ca \\
     &=x^2-2\frac{-b}{2a}x+\frac ca \\
    x&=\frac{-b}{2a}\pm\sqrt{\frac{b^2}{4a^2}-\frac ca} \\
     &=\frac{-b\pm \sqrt{4a^2\left(\frac{b^2}{4a^2}-\frac ca\right)}}{2a} \\
     &=\frac{-b\pm\sqrt{b^2-4ac}}{2a}
  \end{align*}
  as needed.

  If $\Delta<0$ there are clearly no solutions. If
  $\Delta=0$ the unique solution is $x=\frac{-b}{2a}$. If $\Delta>0$ there are
  two solutions given by our equation.
\end{proof}
\subsubsection*{Problem~\ref{p:b:a:quad:1}}
\hypertarget{sp:b:a:quad:1}
Once again not much to say here --- interpret the number bases  in the usual way
and solve the resulting quadratic.

Answer: 57.
\begin{proof}
  We have
  \begin{align*}
    111_b&=212_{b-2} \\
    b^2+b+1&=2(b-2)^2+(b-2)+2 \\
           &=2b^2-7b+8 \\
    0&=b^2-8b+7 \\
    b&=4\pm\sqrt{4^2-7} \\
     &=4\pm3
  \end{align*}
  Therefore, since $b>2$ we get $b=7\implies x=7^2+7+1=57$.

  Finally, 57 is indeed 111 in base 7 and 212 in base 5.
\end{proof}
\subsubsection*{Problem~\ref{p:b:a:quad:2}}
\hypertarget{sp:b:a:quad:2}
The key here is to use the discriminant ($\Delta$ in
Result~\ref{r:b:a:quad:2}). In particular,
it's enough to prove that at least one of the two $\Delta$s is nonnegative. The
easiest way of doing this is to assume that the first is negative and prove that
the second isn't.
\begin{proof}
  Assume that there is no real number $x$ such that $x^2+(r+1)x+s=0$. Then the
  discriminant $(r+1)^2-4s$ is negative, so $4s>(r+1)^2$.

  Since $(r+1)^2\ge 0$, we know that $s>0$.
  Also, $4(s-r)>(r+1)^2-4r=(r-1)^2\ge 0$.
  So since $s>0$ and $4(s-r)>0$, their product $4s^2-4sr$ is also positive so
  the discriminant of the second quadratic is positive, meaning that it has at
  least one real solution.
\end{proof}
\subsubsection*{Problem~\ref{p:b:a:quad:3}}
\hypertarget{sp:b:a:quad:3}
At first glance, this looks like our methods can't help since we have a cubic
not a quadratic. However, the same trick used in Result~\ref{r:b:a:quad:1}
of recognising a common factorisation does in fact work.

Answer: $x=-1-\sqrt[3]{4}$.
\begin{proof}
  We subtract 4 from both sides to make the LHS into something we recognise:
  \begin{align*}
    x^3+3x^2+3x+1&=-4 \\
    (x+1)^3&=-4 \\
    x+1&=-\sqrt[3]{4} \\
    x&=-1-\sqrt[3]{4}.
  \end{align*}
  To show that this number works, we can either substitute it in and do the
  algebra, or notice that each step above was actually an equivalence so the
  implications run backwards as well.
\end{proof}

\subsubsection*{Result~\ref{r:b:a:ineq:1}}
\hypertarget{sr:b:a:ineq:1}
Since we want to use the fact that squares are nonnegative, we collect all the
terms on one side. The rest is recognition, which can be helped by noticing that
we want equality to occur when $a=b$.
\begin{proof}
  \[\RHS-\LHS=\frac{a+b}2-\sqrt{ab}=\frac{a+b-2\sqrt{ab}}2=\frac{(\sqrt a-\sqrt
  b)^2}2\ge 0,\] as needed.
\end{proof}

\subsubsection*{Problem~\ref{p:b:a:ineq:1}}
\hypertarget{sp:b:a:ineq:1}
We have to have a 0 and a 2015 in the set, but apart from them the rest of the
terms should be as small as possible. This means that we can apply
{Result 3} to get a function we want to minimise. A little
algebraic trickery means it's enough to minimise \[n+\frac{4032}n.\] Then, by
{Result 7}, the  minimum of this over $\Rr$ is
$2\sqrt{4032}$ at $n=\sqrt{4032}\approx 63.5$, which means either 63 or 64
should minimise the expression over $\Nn$. In fact both do, so we should
try to force the expression into something that looks like $(n-63)(n-64)$, and
indeed doing that solves the problem.

Answer: 62.
\begin{proof}
  Let $n$ be the number of elements in $S$, and let $S=\{s_1,s_2,\ldots,s_n\}$,
  where the $s_i$s are in increasing order.
  Then \[s_i\ge i-1\, \forall\, i<n,\] and $s_n=2015$, so the average is at
  least
  \begin{align*}
    \frac{0+1+\cdots+n-2+2015}n&=\frac{\frac{(n-2)(n-1)}2+2015}n \\
                               &=\frac{n^2-3n+4032}{2n}\\
                               &=\frac{n^2-127n+4032}{2n}+62 \\
                               &=\frac{(n-63)(n-64)}{2n}+62.
  \end{align*}
  Since $n$ is an integer, the first term is 0 if $n$ is either 63 or 64 and
  positive otherwise, which means that the minimum value is 62, achieved when
  $S$ is either $\{0,1,\ldots,61,2015\}$ or $\{0,1,\ldots,61,62,2015\}$.
\end{proof}
\subsubsection*{Result~\ref{r:b:a:sums:1}}
\hypertarget{sr:b:a:sums:1}
There are three ways I know of doing this. One of them is a standard induction,
but the other two are more interesting.

For the first way, we notice that we already know the special case (see
Result~\ref{r:m:4:1}) where $a=0$ and $b=1$. A little algebra allows us
to reduce the whole problem to this particular case.
\begin{proof}
  We have
  \begin{align*}
    \sum_{i=0}^n(a+bi)&=\sum_{i=0}^n a+\sum_{i=0}^n bi \\
                      &=a(n+1)+b\sum_{i=0}^n i \\
                      &=a(n+1)+b\frac{n(n+1)}2 \\
                      &=\frac{(2a+bn)(n+1)}2,
  \end{align*}
  as needed.
\end{proof}
For the second way, we use a trick called \emph{Gaussian pairing} --- we pair
the first term with the last term and so on --- so that each pair has the same
sum.
\begin{proof}
  We have
  \begin{align*}
    \sum_{i=0}^n(a+bi)&=\sum_{i=0}^n(a+b(n-i)) \\
                      &=\frac12\left(\sum_{i=0}^n(a+bi)+
                      \sum_{i=0}^n(a+b(n-i))\right)
                      \\
                      &=\frac12\left(\sum_{i=0}^n(2a+bn)\right) \\
                      &=\frac{(n+1)(2a+bn)}2,
  \end{align*}
  as needed.
\end{proof}
\subsubsection*{Result~\ref{r:b:a:sums:2}}
\hypertarget{sr:b:a:sums:2}
The main idea here comes from extending a couple of the common factorisations:
\begin{itemize}
  \item $1-r^2=(1-r)(1+r)$
  \item $1-r^3=(1-r)(1+r+r^2)$
\end{itemize}
Let's write it up.
\begin{proof}
  We have \[1-r^{n+1}=(1-r)(1+r+r^2+\cdots+r^n),\] since all the middle terms
  cancel. Dividing both sides by $1-r$ yields the desired result.
\end{proof}
\subsubsection*{Problem~\ref{p:b:a:sums:1}}
\hypertarget{sp:b:a:sums:1}
So you sum your sequence using Result~\ref{r:b:a:sums:1}, which gets you a
closed form. 

Then you're left having to prove that $a+1$ and $2n+a$ cannot both be powers of
2. Playing around with special cases tells you at least one is odd, and from
there it's easy to finish.
\begin{proof}
  From Result~\ref{r:b:a:sums:1}, we have that the given sum is equal to
  \[\frac{(a+1)(2n+a)}2.\] So assume, for contradiction, that this is a power
  of 2. Then we have that twice the sum is also a power of 2, so $(a+1)(2n+a)$
  is a power of 2. Since $a+1$ and $2n+a$ both divide powers of 2, they must
  each be a power of 2. Since $a$ and $n$ are positive integers, both $a+1$ and
  $2n+a$ are at least 2 so they're even. So their difference, $2n-1$, must also
  be even, which is a contradiction.
\end{proof}
\subsection{Combinatorics}
\subsubsection*{Result~\ref{r:b:c:am:1}}
\hypertarget{sr:b:c:am:1}
Not much to say here: just apply the multiplication principle.
\begin{proof}
  There are $n$ ways to choose the first thing, $n-1$ ways to choose the second
  (since the first has been chosen), $n-2$ to choose the third, and so on up to
  $n-k+1$ for the $k$th. Thus the total number of ways is
  \[n(n-1)\cdots(n-k+1),\] as claimed.
\end{proof}
\subsubsection*{Problem~\ref{p:b:c:am:1}}
\hypertarget{sp:b:c:am:1}
So you try some small cases because of course you do, and you notice that every
2 goals apart from the first 2, the number of ways is multiplied by 3. With the
help of like, a tree diagram, it's easy to see that after an odd number of goals
there are 3 ways of scoring the next 2. The rest is easy:

Answer: $972$.
\begin{proof}
  There are 2 ways for the first goal to be scored.

  Assume an odd number of goals has been scored. Then the difference between the
  two teams' goals is 1, so we can WLOG team A has one more goal than team B.
  Then the next two goals can be AB, BA, BB so there are 3 ways for the next two
  goals to be scored.

  Finally, there are 2 ways for the last 2 goals to be scored.

  Thus, the total number of ways for 12 goals to be scored is $2\times 3^5\times
  2=972$.
\end{proof}
\subsubsection*{Result~\ref{r:b:c:c:1}}
\hypertarget{sr:b:c:c:1}
The key to proving this is to notice that we have already proven
Result~\ref{r:b:c:am:1} where order does matter. To go from order mattering to
order not mattering, we need to find out how much we've overcounted by: that is,
how many permutations, where order matters, go to the same combination, where
order doesn't.
\begin{proof}
  By Result~\ref{r:b:c:am:1}, there are $\frac{n!}{k!}$ ways of choosing $k$
  things from $n$, where order matters.

  Let there be $x$ ways of choosing $k$ things from $n$, where order doesn't
  matter. Then for each such way, each permutation of those $k$ things is
  counted in our $\frac{n!}{(n-k)!}$ from above. Applying
  Result~\ref{r:b:c:am:1} again, we have that $k!\times x=\frac{n!}{(n-k)!}$,
  which rearranges to the claimed formula.
\end{proof}
\subsubsection{Problem~\ref{p:b:c:c:1}}
\hypertarget{sp:b:c:c:1}
So upon reading the problem we can let our palindromes be $\overline{xyx}$ and
$\overline{ztz}$. Clearly, $x>z$, after which it's clear why $y\ge t$.

So by the multiplication principle, it suffices to count two things:
\begin{itemize}
  \item The number of ways of choosing digits $x$ and $z$ with $x>z>0$
  \item The number of ways of choosing digits $y$ and $t$ with $y\ge t$.
\end{itemize}
Both these things are quite easy to count using this section's formula.

Answer: 1980.
\begin{proof}
  Let the palindromes be $\overline{xyx}$ and $\overline{ztz}$. Since their
  difference is 3 digits and $a>b$, we have $x>z$. 

  Since the second digit of our subtraction can't carry, we must have $y\ge t$.
  So it's enough to count the number of ways we can choose $x$ and $z$, and
  separately the number of ways we can choose $y$ and $t$.

  Since $x$ and $z$ are distinct digits from 1 to 9, the number of choosing them
  is $\binom92=36$. Since $y$ and $t$ are not-necessarily-distinct digits from
  0 to 9, the number of ways of choosing them is $\binom{10}2+10=55$. Thus, the
  answer is $36\times 55=1980$.
\end{proof}
\section{Not-so-basics}
\subsection{Algebra}
\subsubsection*{Result~\ref{r:n:a:polys:1}}
\hypertarget{sr:n:a:polys:1}
For those of you who have seen polynomial long division, this should seem very
familiar, and indeed the same idea of dividing from highest to lowest degree
works. The neatest and most rigorous way to write it up uses induction.
\begin{proof}
  By strong induction on the degree.

  Base case $n=1$: if $ar-b=0$ then $\frac ba=r$ so $ax-b=a\left(x-\frac
  ba\right)=a(x-r)$, as needed. Clearly $Q(x)=a$ has degree 0.

  Inductive step: Write \[P(x)=a_n x^{n-1}(x-r)+P_1(x),\] so that $P_1(x)$ is a
  polynomial of degree less than $n$ and $P_1(r)=P(r)-0=0$.
  By the strong inductive hypothesis there
  is a polynomial $Q_1(x)$ of degree less than $n-1$ such that
  $P_1(x)=(x-r)Q_1(x)$. Then, \[P(x)=(x-r)\left(a_n x^{n-1}+Q_1(x)\right),\] so
  if we let $Q(x)=x^{n-1}+Q_1(x)$ we have $P(x)=(x-r)Q(x)$. Finally, $Q(x)$ has
  degree $n-1$ since $Q_1(x)$ has degree less than $n-1$ and $a_n x^{n-1}$ has
  degree $n-1$.
\end{proof}
\subsubsection*{Result~\ref{r:n:a:polys:2}}
\hypertarget{sr:n:a:polys:2}
First we need to understand what the problem is saying. Try special cases:
$i=0,\ i=n,\ n=1,\ n=2$ and so on, until you know what it's saying.

The idea here is to use the previous result multiple times to factorise our
polynomial fully, then expand it again. Then when we extract the $x^{n-i}$
coefficient, each term that contributes to it is a product where $x$ appears
$n-i$ times, and the rest is a product of $i$ $(r_i)$s and a constant term.
Since each combination of $i$ $(-r_i)$s appears exactly once in the expansion,
we get the claimed formula.

The neatest way of writing this up is to use induction.
\begin{proof}
  By induction on the degree.

  Base case $n=0$: there are no $r_i$s so all that
  we have to prove is $a_0=a_0$, which is obvious.

  Inductive step: Since $r_n$ is a root of $P(x)$, we can write
  $P(x)=(x-r_n)Q(x)$ for some polynomial $Q(x)$ of degree $n-1$. Then we know
  that $Q(x)$ has roots $r_1,\ldots,r_{n-1}$ so by the inductive hypothesis, we
  know that in $Q(x)$:
  \begin{itemize}
    \item The coefficient of $x^{n-i}$ is
      \[(-1)^{i-1}a_n\sum_{j_1<\cdots<j_{i-1}}\prod_{k=1}^{i-1}r_{j_k}.\]
    \item The coefficient of $x^{n-i-1}$ is
      \[(-1)^{i}a_n\sum_{j_1<\cdots<j_i}\prod_{k=1}^i r_{j_k}.\]
  \end{itemize}
  Since the coefficient of $x^{n-i}$ in $P(x)$ is the coefficient of $x^{n-i-1}$
  in $Q(x)$ minus $r_n$ times the coefficient of $x^{n-i}$ in $Q(x)$, it's what
  we claim it is. Since this argument works for each $i$, the induction is
  complete.
\end{proof}
\newpage
\part{Other resources}
\section{Basics}
\begin{itemize}
  \item \href{https://artofproblemsolving.com/resources}{AOPS Resources}
  \item
    \href{https://www.its.caltech.edu/~kpilch/olympiad/MethodsOfProof-Complete.pdf}
    {Methods of Proof}
  \item \href{https://artofproblemsolving.com/news/articles/how-to-write-a-solution}
    {How to write a solution}
  \item OTIS Excerpts, chapter 1
\end{itemize}

\section{NT}
\begin{itemize}
  \item
    \href{https://numbertheoryguydotcom.files.wordpress.com/2018/10/main1.pdf}
    {Intermediate Number Theory}
  \item
    \href{http://s3.amazonaws.com/aops-cdn.artofproblemsolving.com/resources/articles/olympiad-number-theory.pdf}
      {Olympiad Number Theory through Challenging Problems}
  \item \href{https://drive.google.com/file/d/1BcJTLjQaelZ4w_70oHKyImC2I8zLfyrt/view?usp=sharing}
    {Modern Olympiad Number Theory}
\end{itemize}

\section{Algebra}
\begin{itemize}
  \item OTIS Excerpts, chapters 2 to 5
  \item
    \href{https://alexanderrem.weebly.com/uploads/7/2/5/6/72566533/polynomials.pdf}
    {Polynomials}
  \item
    \href{https://alexanderrem.weebly.com/uploads/7/2/5/6/72566533/sequences.pdf}
    {Sequences}
\end{itemize}

\section{Combi}
\begin{itemize}
  \item \href{http://math.sun.ac.za/swagner/Combinatorics.pdf}{Combinatorics}
  \item
    \href{https://drive.google.com/file/d/1sQtirXxkEfWYuGSKDZ-d7VGYkR_idebY/view}
    {Olympiad Combinatorics}
  \item
    \href{https://drive.google.com/file/d/1pkgUjWs4ArL9Huq712NO1RtuAvXjr3d9/view?usp=sharing}
    {Combinatorics and Combinatorial Geometry}
\end{itemize}

\section{Geo}
\begin{itemize}
  \item
    \href{https://books.google.com/books?id=47UaDAAAQBAJ&lpg=PP1&pg=PP1#v=onepage&q&f=false}
    {EGMO Chapters 1 to 3}
  \item
    \href{https://www.maa.org/sites/default/files/pdf/ebooks/pdf/EGMO_chapter8.pdf}
    {EGMO Chapter 8}
  \item \href{https://www.olympiadgeometry.com/the-book.html} {A Beautiful
      Journey through Olympiad Geometry}
  \item \href{https://yufeizhao.com/olympiad/geolemmas.pdf} {Lemmas in Olympiad
      Geometry}
\end{itemize}

\section{Problems}
\begin{itemize}
  \item \href{https://mathscomp.ms.unimelb.edu.au/past-papers/} {Melbourne Uni
      Maths Comp past papers}
  \item \href{https://www.bmoc.maths.org/home/bmolot.pdf} {BMO past papers}
  \item 
    \href{https://artofproblemsolving.com/community/c14_international_contests}
    {International contest collection}
    (see especially EGMO, IMO,  IMO Shortlist, RMM)
\end{itemize}

\section{Assorted}
\begin{itemize}
  \item \href{https://web.evanchen.cc/textbooks/OTIS-Excerpts.pdf}{OTIS
      Excerpts}
  \item \href{https://how-did-i-get-here.com/olympiad/}{HDIGH Olympiad page}
  \item \href{https://how-did-i-get-here.com/olympiad-resources/}{HDIGH Olympiad
      resources}
  \item \href{https://web.evanchen.cc/olympiad.html}{Evan Chen's page}
  \item \href{https://www.math.cmu.edu/~ploh/olympiad.shtml}{Po-Shen Loh's page}
  \item \href{https://cjquines.com/math/competition-handouts}{CJ Quines' page}
  \item \href{https://yufeizhao.com/olympiad/}{Yufei Zhao's page}
  \item \href{https://alexanderrem.weebly.com/math-competitions.html}{Alexander
      Remonov's page}
  \item \href{https://www.its.caltech.edu/~kpilch/olympiad.html}{Konrad Pilch's
      page}
  \item
    \href{https://archive.org/download/ProblemSolvingStrategies/Problem\%20Solving\%20Strategies.pdf}
    {Problem Solving Strategies}
\end{itemize}

\part*{Todos}
\begin{itemize}
  \item Master listing of results
  \item Observations to add:
    \begin{itemize}
      \item Angles to similarity to lengths to similarity to conclusion
      \item Induct on the degree for polynomials
    \end{itemize}
  \item Asymptote figures
\end{itemize}
\end{document}
