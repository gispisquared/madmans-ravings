\documentclass{amsart}
\usepackage{amsmath,amssymb,amsthm,hyperref}
\hypersetup{colorlinks=true}
\newtheorem{problem}{Problem}
\newtheorem{example}{Example}
\newcommand{\Rr}{\mathbb{R}}
\title{Some Madman's Ravings}
\author{gispisquared}
\begin{document}
\maketitle
\tableofcontents
\newpage
\part{Introduction}
I'm writing this since I think that most of the resources out there that try to
teach you olympiad maths have all the theory you need to solve IMO-level
problems (and sometimes much more!), but it's much more rare to see people who
tell you how to look at problems and figure out approaches that may, or in some
cases should, work. True, much of this is an individual learning experience, but
I think a lot of this can be written more explicitly.

This may reduce some of the magic of figuring this stuff out for yourself; to
mitigate this effect, I have relegated the solutions to examples to the back so
that you can (and I encourage you to!) try the examples before reading the
solutions.

Since I refuse to rehash stuff that others have done better, I'll refer you to a
couple of resources about how to write proofs properly:
\begin{itemize}
  \item
    \href{https://artofproblemsolving.com/news/articles/how-to-write-a-solution}
    {How to Write a Maths Solution}
  \item \href{https://web.evanchen.cc/handouts/english/english.pdf}{Notes on
    English}
\end{itemize}

Cool, hopefully now you know how to write proofs. Guess that means every time
you solve a problem you'll get a 7, right?
\newpage
\part{Theory}
Now it's time to learn how to actually solve problems.
\section{Methods of Proof}
There are a few main methods of proof --- that is, ways in which you can go from
the conditions in the problem to your given condition.

\subsection{Direct Proof}
This is perhaps the simplest type of proof. The idea is to start with the stuff
you're given, do some logical deduction, and finish with what you want to prove.

I'll give a few examples.

\begin{example}
  \hypertarget{ex:1}
  Let $a, b\in\Rr^+$. Prove that \[\frac{a+b}2\ge\sqrt{ab}.\]
  \hyperlink{s:1}{Solution}
\end{example}
\begin{example}
  \hypertarget{ex:2}
  Prove that if we divide any perfect square by 4, the remainder is either 0 or
  1.
  \hyperlink{s:1}{Solution}
\end{example}
\newpage
\part{Problems}
\newpage
\part{Proofs}
\begin{proof}
  \hyperlink{ex:1}{Problem}
  \hypertarget{s:1}
  Since squares are nonnegative, we have
  \begin{align*}
    \left(\sqrt a-\sqrt b\right)^2 & \ge 0 \\
    \iff a+b-2\sqrt{ab} & \ge 0 \\
    \iff a+b & \ge 2\sqrt{ab} \\
    \iff \frac{a+b}2 & \ge \sqrt{ab},
  \end{align*}
  which is what we wanted to prove.
\end{proof}
\begin{proof}
  \hyperlink{ex:2}{Problem}
  \hypertarget{s:2}
  Let our perfect square be $a^2$. We split into cases depending on whether $a$
  is even or odd.
  \begin{itemize}
    \item If $a$ is even, write $a=2b$. Then \[a^2=(2b)^2=4b\] so the remainder is
      0 in this case.
    \item If $a$ is odd, write $a=2b+1$. Then
      \[a^2=(2b+1)^2=4b^2+4b+1=4(b^2+b)+1\] so the remainder is 1 in this case.
  \end{itemize}
  In either case, the remainder is either 0 or 1 as needed.
\end{proof}
\end{document}
