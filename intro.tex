\chapter*{Introduction}
I was recently\footnote{February 2022} asked what advice I would give to a
beginner Olympiad student. Partly for your sake if you're reading this, and
partly to inform how I write this, I'll copy (and punctuate) my response here:
\begin{itemize}
  \item Don't believe people (like whoever writes
    \href{https://how-did-i-get-here.com/olympiad/}{hdigh}) when they say methods
    of proof is all you should learn before spamming problems. Maybe that works
    if you're as smart as them, but most of us don't wanna spend most of our
    training reinventing the wheel.
  \item On the other hand, don't waste time looking up the latest crazy theorem
    aopsers start mentioning; those are unlikely to ever be necessary and are
    probably only rarely useful because problem setters and pscs prefer problems
    where the easiest solution involves only canonical Olympiad knowledge.
  \item Most of the problems you're trying right now are gonna be pretty
    trivial for more advanced students, so whenever you solve or
    don't solve a problem figure out why it should have been trivial and
    hopefully this hindsight will start turning into foresight.
\end{itemize}

If I am to be consistent, then, this book's contents should be of three main
types: canonical Olympiad knowledge and techniques, problems on which to
practice these techniques, and solutions with motivation that try to make the
problems seem as trivial as possible. I've also chosen both to include proofs to
the known results (since their proofs are within Olympiad students' grasp and
are often the source of canonical techniques), and to separate these proofs from
the theorem statements in the same way as with problems (so that
students can try to prove the theorems themselves). 

There will be two main types of questions:
\begin{itemize}
  \item \emph{Results}, which are considered important and well-known, and come
    up sporadically (or in some cases consistently) as steps in the harder
    problems. Every result comes with an implied ``Prove that'' attached.
  \item \emph{Problems}, which will be questions taken from contests (mostly AMC
    and AIMO in Part 2, and AMO, EGMO and IMO in Part 3).
\end{itemize}

Since I refuse to rehash stuff that others have done better, I'll refer you to a
couple of resources about how to write proofs properly:
\begin{itemize}
  \item
    \href{https://artofproblemsolving.com/news/articles/how-to-write-a-solution}
    {How to Write a Maths Solution}
  \item \href{https://web.evanchen.cc/handouts/english/english.pdf}{Notes on
    English}
\end{itemize}
Cool, hopefully now you know how to write proofs. Guess that means every time 
you solve a problem you'll get a 7, right?

Of course, to learn you'll need to put the required effort into maths --- you're
unlikely to learn too much just from reading problems and solutions. I've
tried to make sure that all problems are accessible using only prior knowledge
(something like high school maths up to y9 or so) and prior exposition;
therefore, you should at least try to solve problems before reading the
solutions.
\section*{General Recommendations}
This contains some general ideas about how to approach problems, or what to
try when stuck.

(With apologies to P\'olya)
\begin{itemize}
  \item How are the objects referred to in the problem defined? What properties
    do you know them to have?
  \item Try small or special cases. Can you spot patterns in their structure? In
    how you solve them? Can you prove any of these patterns in general? Do any
    of these patterns help?
  \item Look at stuff that is extremal in some way: biggest, smallest, most
    connected, most disconnected, most composite, prime, whatever
  \item Think about what happens if the problem, or the conclusion, is wrong.
  \item Can you reduce any instance of the problem to a smaller instance? 
    Can you reduce a counterexample to a smaller counterexample?
  \item Have you seen something similar before? Can you use the result or the
    method? Can you introduce some auxiliary element to make its use possible?
  \item Can you draw a diagram to help you understand the problem?
\end{itemize}
And some more specific recommendations:
\begin{itemize}
  \item If you're stuck on a polynomial question, try inducting on the degree.
  \item If you need to prove something is unique, assume there are two things
    with the same properties and prove they're the same.
  \item In geometry, make sure your diagrams are large and accurate. If you're
    stuck, try looking at a diagram and trying to guess which points look like
    they form similar triangles or cyclic quadrilaterals.
\end{itemize}
