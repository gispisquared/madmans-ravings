\subsection{Polynomials}
\subsubsection*{Result~\ref{r:n:a:polys:2}}
\hypertarget{sr:n:a:polys:2}
It is natural to split the proof into two parts: existence and uniqueness. 
Uniqueness is the easier part: it can be proven by the usual method of assuming
two representations exist, and proving they are in fact the same.

The proof of existence proceeds by induction on the degree.
The key is to reduce $A(x)$ to a polynomial with smaller
degree by cancelling the leading term. If we do this in a way that lets us
control $Q(x)$ and $R(x)$, the induction will work.
\begin{proof}
  First we prove such a representation exists, by strong induction on the degree
  of $A$. Let $d=\deg B$.

  Base cases $\deg A<d$: clearly $Q(x)=0,\ R(x)=B(x)$ satisfies all the
  conditions.

  Inductive step: let $a_n x^n$ be the leading term of $A$, and let $b_d x^d$ be
  the leading term of $B$.
  We may write \[A(x)=\frac{a_n}{b_d}x^{n-d} B(x)+A_1(x)\] for some polynomial
  $A_1$. Since the first term on the RHS cancels the $a_n x^n$, this polynomial
  $A_1(x)$ can be represented as $Q_1(x)B(x)+R_1(x)$. But then choosing
  $Q(x)=Q_1(x)+\frac{a_n}{b_3}x^{n-d}B(x),\ R(x)=R_1(x)$ provides a
  representation for $A$.

  Now we prove the representation is unique. Assume there are polynomials
  $Q_1,R_1,Q_2,R_2$ such that \[A(x)=B(x)Q_1(x)+R_1(x)=B(x)Q_2(x)+R_2(x).\]
  Then, $B(x)(Q_1(x)-Q_2(x))=R_2(x)-R_1(x)$. The degree of the RHS is less
  than $d$; therefore, so is the degree of the LHS.\@
  But since the degree of the
  LHS is at least $D$ unless $Q_1=Q_2$, we get $Q_1=Q_2$ and therefore
  $R_1=R_2$.
\end{proof}
\subsubsection*{Result~\ref{r:n:a:polys:3}}
\hypertarget{sr:n:a:polys:3}
First we need to understand what the problem is saying. Try special cases:
$i=0,\ i=n,\ n=1,\ n=2$ and so on, until you know what it's saying.

The idea here is to use the corollary from the 
previous result multiple times to factorise our
polynomial fully, then expand it again. Then when we extract the $x^{n-i}$
coefficient, each term that contributes to it is a product where $x$ appears
$n-i$ times, and the rest is a product of $i$ $(r_i)$s and a constant term.
Since each combination of $i$ $(-r_i)$s appears exactly once in the expansion,
we get the claimed formula.

The neatest way of writing this up is to use induction.
\begin{proof}
  By induction on the degree.

  Base case $n=0$: there are no $r_i$s so all that
  we have to prove is $a_0=a_0$, which is obvious.

  Inductive step: Since $r_n$ is a root of $P(x)$, we can write
  $P(x)=(x-r_n)Q(x)$ for some polynomial $Q(x)$ of degree $n-1$. Then we know
  that $Q(x)$ has roots $r_1,\ldots,r_{n-1}$ so by the inductive hypothesis, we
  know that in $Q(x)$:
  \begin{itemize}
    \item The coefficient of $x^{n-i}$ is
      \[(-1)^{i-1}a_n\sum_{j_1<\cdots<j_{i-1}}\prod_{k=1}^{i-1}r_{j_k}.\]
    \item The coefficient of $x^{n-i-1}$ is
      \[(-1)^{i}a_n\sum_{j_1<\cdots<j_i}\prod_{k=1}^i r_{j_k}.\]
  \end{itemize}
  Since the coefficient of $x^{n-i}$ in $P(x)$ is the coefficient of $x^{n-i-1}$
  in $Q(x)$ minus $r_n$ times the coefficient of $x^{n-i}$ in $Q(x)$, it's what
  we claim it is. Since this argument works for each $i$, the induction is
  complete.
\end{proof}
