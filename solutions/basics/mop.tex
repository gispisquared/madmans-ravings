\subsection{Methods of Proof}
\subsubsection*{Result~\ref{r:b:m:1:1}}
\hypertarget{s:m:1:1}
If you start trying some small cases, what you'll eventually find is that if $n$
is an even integer, then $n^2$ leaves a remainder of 0 when divided by 4, and
if $n$ is an odd integer, then $n^2$ leaves a remainder of 1 when divided by
4. Once you've conjectured this, all that's left is to recall what it means for
a number to be even or odd, and then the proof falls out quite naturally:
\begin{proof}
  Let the perfect square be $n^2$. We split into cases depending on the parity
  of $n$.
  \begin{itemize}
    \item If $n$ is even, let $n=2m$ for some integer $m$. Then
      \[n^2=(2m)^2=4m^2,\] which leaves a remainder of 0 when divided by 4.
    \item If $n$ is odd, let $n=2m+1$ for some integer $m$. Then
      \[n^2=(2m+1)^2=4m^2+4m+1=4(m^2+m)+1,\] which leaves a remainder of 1 when
      divided by 4.
  \end{itemize}
  In either case, the remainder left when dividing $n^2$ by 4 is either 0 or 1,
  which is what we wanted to prove.
\end{proof}
\subsubsection*{Result~\ref{r:b:m:2:1}}
\hypertarget{s:m:2:1}
The key here is to assume, for contradiction, that there are
only finitely many primes. Then we want to prove a suitable contradiction --- a
nice way of doing this is to find a number that isn't 1 but isn't divisible by
any of our finitely many primes. The idea of constructing such a number by
multiplying everything and adding 1 is surprisingly common in Olympiad maths.
\begin{proof}
  Assume that there are only finitely many primes $p_1,p_2,\ldots,p_n$. Then,
  consider the number $A=p_1p_2\cdots p_n+1$. Clearly $A$ is a positive integer
  larger than 1, so it must have a prime factor, which means that for some $i$,
  $p_i\mid A$. But $p_i\mid A-1$, so $p_i\mid A-(A-1)=1$, which is our
  contradiction.
\end{proof}
\subsubsection*{Problem~\ref{p:b:m:3:1}}
\hypertarget{s:m:3:1}
This is more an exercise in logic than in maths.
You could be stuck for ages trying to prove the problem directly, but as soon as
you try to use some indirect approach (like contrapositive, contradiction, or
assuming one part of the conclusion is false and proving the other is true) the
problem pretty much solves itself. I think the cleanest solution in this
particular case uses the contrapositive.
\begin{proof}
  We prove the contrapositive: that if $a$ and $b$ are both rational, then so is
  $a+b$.

  Let $a=\frac wx,\ b=\frac yz$. Then \[a+b=\frac wx+\frac
  yz=\frac{wz+xy}{xz},\] which is clearly rational.
\end{proof}
\subsubsection*{Result~\ref{r:b:m:4:1}}
\hypertarget{s:m:4:1}
This is a classic induction problem. Apart from being instructive because it
isolates the idea of induction, it does highlight a minor point. In the
inductive step, it's just as acceptable to assume the problem is true for $k$
and prove it for $k+1$ as to assume the problem is true for $k-1$ and prove it
for $k$. In this particular case, the latter is somewhat easier.
\begin{proof}
  We prove this by induction on $n$.

  Base case $n=1$: We have $\LHS=1=\frac{1\times 2}2=\RHS$.

  Inductive step: Assume the problem is true for $n=k-1$. Then,
  \begin{align*}
    1+2+\cdots+k&=(1+2+\cdots+k-1)+k \\
                &=\frac{k(k-1)}2+k \\
                &=\frac{k(k-1)+2k}2 \\
                &=\frac{k(k+1)}2,
  \end{align*}
  so the problem is true for $n=k$.
\end{proof}
\subsubsection*{Result~\ref{r:b:m:4:2}}
\hypertarget{s:m:4:2}
This is a good example of how solutions to the same problem can look quite
different when written up with PMI or with well-ordering.

First assume PMI\@. A direct induction will prove the statement quite easily.

Now we can try assuming well-ordering. Looking for a contradiction, assume there
is some positive integer $a<1$. A contradiction is easiest to find by applying
well-ordering to the set of all positive integers.

Let's write this up.
\begin{proof}
  First, assume PMI\@.

  Base case $n=1$: clearly $1\ge 1$.

  Inductive step: assume that $k\ge 1$.
  Then $k+1>k\ge 1$, completing the induction.

  Now, assume well-ordering. Assume there's some $a\in\mathbb N$ with $a<1$.
  Since $\mathbb N$ is nonempty, we may let $b$ be its smallest element.
  Since $ab$ is a positive integer, we have $b\le ab$.
  On the other hand, since $a<1$ we get that $ab<b$.
  This contradiction concludes the proof.
\end{proof}
\subsubsection*{Result~\ref{r:b:m:4:3}}
\hypertarget{s:m:4:3}
Since this is an ``if and only if'' problem, we will probably need to find
separate proofs in each direction. 

First, let's use induction to prove well-ordering. Our desired conclusion is
that any nonempty set of positive integers has a smallest element. Intuitively,
what we would like to do is to check if 1 is in it, then if 2 is in it, and so
on until we find an element that's in it. Once we've found that element, we wish
to prove that it's the smallest element. However, this is hard to write up
because we'd like our process to ``finish'' after a finite number of steps,
while induction only talks about things that are true for all positive integers.
To make our lives easier, we instead try to prove the contrapositive of
well-ordering: that if $S$ is a set of positive integers with no smallest
element then $S$ is empty.

Now, our argument looks similar: we claim that 1 is not in $S$, that 2 is not in
$S$, and so on forever. To make sure that we can actually keep repeating this
argument, we need to show that if we found an element in $S$ it must be the
smallest element. Proving this requires tweaking our inductive
assumption.

Now, let's use well-ordering to prove PMI\@. You'll probably get nowhere if you
aren't completely clear about what you're trying to prove (you may as well
replace PMI by gibberish), so let's write out PMI in full:

Let $S$ be a set of positive integers. If $1\in S$, and if the statement
$\forall a\in S,\ a+1\in S$ is true, then $S$ contains all positive integers.

Once again we use indirect proof --- this time it's proof by contradiction
(contrapositive also works). Let's assume that there is a set $S$ that satisfies
both conditions but doesn't contain all positive integers. We hope to use
well-ordering to find a contradiction.

The key idea is to consider the smallest integer that isn't in $S$, which is
possible by well-ordering. Then the condition implies that either it's not the
smallest, or $1\not\in S$ --- a contradiction either way.

Let's write it up.
\begin{proof}
  First we prove that if PMI is true, then so is well-ordering. Assume PMI, and
  we'll prove the contrapositive of well-ordering: that if $S$ is a set of
  positive integers with no smallest element, then it is empty.

  I prove by strong induction that for each positive integer $n$, if $m\in S$
  then $m\ge n+1$.

  Base case $n=1$: if $1\in S$, then 1 would be the smallest element in $S$
  (since if $m\in S$ then $m$ is a positive integer so $m\ge 1$).
  So since $S$ has no smallest element, 1 is not the smallest element in $S$ so
  taking the contrapositive we get that 1 is not in $S$.

  Inductive step: Assume that for any $m\in S$ we know that $m\ge k+1$. I
  claim that $k+1\not\in S$. Indeed, if $k+1$ were in $S$, then it would be the
  smallest element of $S$. But since $S$ has no smallest element, $k+1$ can't be
  in $S$. Therefore, if $m\in S$ we know $m-(k+1)$ is a positive integer so
  $m-(k+1)\ge 1$, so $m\ge k+2$, as required for the inductive step.

  This completes the induction. Now if $n\in S$ we've proven that $n\ge n+1$, a
  contradiction so $S$ must be empty.

  Now I prove that if well-ordering is true, then so is PMI\@. Assume for
  contradiction that well-ordering is true but PMI is not. Then, there is a set
  $S$ of positive integers that contains 1 and such that for each $a\in S,\
  a+1\in S$ but that does not contain all positive integers. Then, the set
  $\Nn\setminus S$ is nonempty so by well-ordering it contains a smallest
  element $a$.

  Since we know that $1\in S$, we know that $a\ne 1$. So since $a\ge 1$,
  $a-1$ is a positive
  integer. Since $a-1<a$ and $a$ is the smallest member of $\Nn\setminus S$,
  $a-1$ is not in $\Nn\setminus S$ so $a-1\in S$. So $(a-1)+1=a\in S$,
  which contradicts the assumption that $a\not\in S$.
\end{proof}
