\subsection{Algebra}
\subsubsection*{Problem~\ref{p:b:a:systems:1}}
\hypertarget{s:b:a:systems:1}
Not much to say here --- interpret as a system of linear equations and solve
however you like. 

Answer: 26.
\begin{proof}
  Let the two numbers be $a$ and $b$, with $a>b$. Then, $b=a-20$ so
  \begin{align*}
    a+4&=3(a-20+4) \\
       &=3a-48 \\
    2a&=52 \\
    a&=26,
  \end{align*}
  so the larger of the two numbers is 26.

  To prove that this actually works, note that if $a=26$ and $b=6$, then $a-b=20$
  and $a+4=30=3\times 10=3(b+4)$ as needed.
\end{proof}
\subsubsection*{Problem~\ref{p:b:a:systems:2}}
\hypertarget{s:b:a:systems:2}
Since we don't like the 1s in our equations, we subtract two equations to get
rid of them. Alternatively, we subtract two equations because that's one of the
most obvious things to do with a system of equations. Either way, once we've
done that the rest of the problem is pretty routine.

Answer: $(x,y,z)=(1,1,1),\left(-2,-2,\frac52\right),\left(-2,\frac52,-2\right),
\left(\frac52,-2,-2\right)$.
\begin{proof}
  Subtract the third equation fron the first:
  \begin{align*}
    xy-xz&=2z-2y \\
    x(y-z)+2(y-z)&=0 \\
    (x+2)(y-z)&=0 \\
  \end{align*}
  So either $x=-2$ or $y=z$. Similarly we can deduce that either $z=-2$ or
  $x=y$. Now we split into four cases:

  \begin{itemize}
    \item $x=-2,\ z=-2$. Then $2y=zx+1=5\implies y=\frac 52$.
    \item $x=-2,\ x=y$. Then similar to the above we get $z=\frac 52$.
    \item $y=z\ z=-2$. In the same way we get $x=\frac 52$.
    \item $y=z,\ x=y$. Then $x^2+1=2x\implies (x-1)^2=0\implies x=1$, so
      $x=y=z=1$.
  \end{itemize}
  So the only solutions are what we claim they are. It is easy to check that
  these solutions all satisfy the original equations.
\end{proof}
\subsubsection*{Result~\ref{r:b:a:quad:1}}
\hypertarget{sr:b:a:quad:1}
First, there are a few ways of seeing that this is the answer.

One way is to try to factorise $x^2-2mx+p$ as $(x-a)(x-b)$. Then $a+b=2m$ and
$ab=p$, and $a$ and $b$ are the values of $x$ we want. Then the key idea is that
the way of using the $a+b=2m$ condition is to let $a=m+c,\ b=m-c$ so that
\[p=ab=(m+c)(m-c)=m^2-c^2,\] so that $c=\sqrt{m^2-p}$.

Another way is to notice that $x^2-2mx+p$ looks a lot like
$x^2-2mx+m^2=(x-m)^2$. I'll do the rest in the proof.

\begin{proof}
  We have
  \begin{align*}
    x^2-2mx+p&=x^2-2mx+m^2+p-m^2 \\
             &=(x-m)^2-\left(m^2-p\right)
  \end{align*}
  If $m^2-p<0$ then clearly there are no solutions. Otherwise, we have
  \begin{align*}
    x^2-2mx+p&=(x-m)^2-\left(\sqrt{m^2-p}\right)^2\\
             &=\left(x-m-\sqrt{m^2-p}\right)\left(x-m+\sqrt{m^2-p}\right),
  \end{align*}
  so it equals 0 if and only if $x=m\pm\sqrt{m^2-p}$.
\end{proof}
\subsubsection*{Result~\ref{r:b:a:quad:2}}
\hypertarget{sr:b:a:quad:2}
The key here is to get this equation into a form such that we can apply the
previous result.
\begin{proof}
  We have
  \begin{align*}
    0&=ax^2+bx+c \\
    0&=x^2+\frac bax+\frac ca \\
     &=x^2-2\frac{-b}{2a}x+\frac ca \\
    x&=\frac{-b}{2a}\pm\sqrt{\frac{b^2}{4a^2}-\frac ca} \\
     &=\frac{-b\pm \sqrt{4a^2\left(\frac{b^2}{4a^2}-\frac ca\right)}}{2a} \\
     &=\frac{-b\pm\sqrt{b^2-4ac}}{2a}
  \end{align*}
  as needed.

  If $\Delta<0$ there are clearly no solutions. If
  $\Delta=0$ the unique solution is $x=\frac{-b}{2a}$. If $\Delta>0$ there are
  two solutions given by our equation.
\end{proof}
\subsubsection*{Problem~\ref{p:b:a:quad:1}}
\hypertarget{sp:b:a:quad:1}
Once again not much to say here --- interpret the number bases  in the usual way
and solve the resulting quadratic.

Answer: 57.
\begin{proof}
  We have
  \begin{align*}
    111_b&=212_{b-2} \\
    b^2+b+1&=2(b-2)^2+(b-2)+2 \\
           &=2b^2-7b+8 \\
    0&=b^2-8b+7 \\
    b&=4\pm\sqrt{4^2-7} \\
     &=4\pm3
  \end{align*}
  Therefore, since $b>2$ we get $b=7\implies x=7^2+7+1=57$.

  Finally, 57 is indeed 111 in base 7 and 212 in base 5.
\end{proof}
\subsubsection*{Problem~\ref{p:b:a:quad:2}}
\hypertarget{sp:b:a:quad:2}
The key here is to use the discriminant ($\Delta$ in
Result~\ref{r:b:a:quad:2}). In particular,
it's enough to prove that at least one of the two $\Delta$s is nonnegative. The
easiest way of doing this is to assume that the first is negative and prove that
the second isn't.
\begin{proof}
  Assume that there is no real number $x$ such that $x^2+(r+1)x+s=0$. Then the
  discriminant $(r+1)^2-4s$ is negative, so $4s>(r+1)^2$.

  Since $(r+1)^2\ge 0$, we know that $s>0$.
  Also, $4(s-r)>(r+1)^2-4r=(r-1)^2\ge 0$.
  So since $s>0$ and $4(s-r)>0$, their product $4s^2-4sr$ is also positive so
  the discriminant of the second quadratic is positive, meaning that it has at
  least one real solution.
\end{proof}
\subsubsection*{Problem~\ref{p:b:a:quad:3}}
\hypertarget{sp:b:a:quad:3}
At first glance, this looks like our methods can't help since we have a cubic
not a quadratic. However, the same trick used in Result~\ref{r:b:a:quad:1}
of recognising a common factorisation does in fact work.

Answer: $x=-1-\sqrt[3]{4}$.
\begin{proof}
  We subtract 4 from both sides to make the LHS into something we recognise:
  \begin{align*}
    x^3+3x^2+3x+1&=-4 \\
    (x+1)^3&=-4 \\
    x+1&=-\sqrt[3]{4} \\
    x&=-1-\sqrt[3]{4}.
  \end{align*}
  To show that this number works, we can either substitute it in and do the
  algebra, or notice that each step above was actually an equivalence so the
  implications run backwards as well.
\end{proof}

\subsubsection*{Result~\ref{r:b:a:ineq:1}}
\hypertarget{sr:b:a:ineq:1}
Since we want to use the fact that squares are nonnegative, we collect all the
terms on one side. The rest is recognition, which can be helped by noticing that
we want equality to occur when $a=b$.
\begin{proof}
  \[\RHS-\LHS=\frac{a+b}2-\sqrt{ab}=\frac{a+b-2\sqrt{ab}}2=\frac{(\sqrt a-\sqrt
  b)^2}2\ge 0,\] as needed.
\end{proof}

\subsubsection*{Problem~\ref{p:b:a:ineq:1}}
\hypertarget{sp:b:a:ineq:1}
We have to have a 0 and a 2015 in the set, but apart from them the rest of the
terms should be as small as possible. This means that we can apply
Result~\ref{r:b:m:4:1} to get a function we want to minimise. A little
algebraic trickery means it's enough to minimise \[n+\frac{4032}n.\] Then, by
Result~\ref{r:b:a:ineq:1}, the  minimum of this over $\Rr$ is
$2\sqrt{4032}$ at $n=\sqrt{4032}\approx 63.5$, which means either 63 or 64
should minimise the expression over $\Nn$. In fact both do, so we should
try to force the expression into something that looks like $(n-63)(n-64)$, and
indeed doing that solves the problem.

Answer: 62.
\begin{proof}
  Let $n$ be the number of elements in $S$, and let $S=\{s_1,s_2,\ldots,s_n\}$,
  where the $s_i$s are in increasing order.
  Then \[s_i\ge i-1\, \forall\, i<n,\] and $s_n=2015$, so the average is at
  least
  \begin{align*}
    \frac{0+1+\cdots+n-2+2015}n&=\frac{\frac{(n-2)(n-1)}2+2015}n \\
                               &=\frac{n^2-3n+4032}{2n}\\
                               &=\frac{n^2-127n+4032}{2n}+62 \\
                               &=\frac{(n-63)(n-64)}{2n}+62.
  \end{align*}
  Since $n$ is an integer, the first term is 0 if $n$ is either 63 or 64 and
  positive otherwise, which means that the minimum value is 62, achieved when
  $S$ is either $\{0,1,\ldots,61,2015\}$ or $\{0,1,\ldots,61,62,2015\}$.
\end{proof}
\subsubsection*{Result~\ref{r:b:a:sums:1}}
\hypertarget{sr:b:a:sums:1}
There are three ways I know of doing this. One of them is a standard induction,
but the other two are more interesting.

For the first way, we notice that we already know the special case (see
Result~\ref{r:b:m:4:1}) where $a=0$ and $b=1$. A little algebra allows us
to reduce the whole problem to this particular case.
\begin{proof}
  We have
  \begin{align*}
    \sum_{i=0}^n(a+bi)&=\sum_{i=0}^n a+\sum_{i=0}^n bi \\
                      &=a(n+1)+b\sum_{i=0}^n i \\
                      &=a(n+1)+b\frac{n(n+1)}2 \\
                      &=\frac{(2a+bn)(n+1)}2,
  \end{align*}
  as needed.
\end{proof}
For the second way, we use a trick called \emph{Gaussian pairing} --- we pair
the first term with the last term and so on --- so that each pair has the same
sum.
\begin{proof}
  We have
  \begin{align*}
    \sum_{i=0}^n(a+bi)&=\sum_{i=0}^n(a+b(n-i)) \\
                      &=\frac12\left(\sum_{i=0}^n(a+bi)+
                      \sum_{i=0}^n(a+b(n-i))\right)
                      \\
                      &=\frac12\left(\sum_{i=0}^n(2a+bn)\right) \\
                      &=\frac{(n+1)(2a+bn)}2,
  \end{align*}
  as needed.
\end{proof}
\subsubsection*{Result~\ref{r:b:a:sums:2}}
\hypertarget{sr:b:a:sums:2}
The main idea here comes from extending a couple of the common factorisations:
\begin{itemize}
  \item $1-r^2=(1-r)(1+r)$
  \item $1-r^3=(1-r)(1+r+r^2)$
\end{itemize}
We get the $a=1$ case of the general formula given in~\ref{b:a:factor}.
Let's write it up.
\begin{proof}
  We have \[1-r^{n+1}=(1-r)(1+r+r^2+\cdots+r^n),\] since all the middle terms
  cancel. Dividing both sides by $1-r$ yields the desired result.
\end{proof}
\subsubsection*{Problem~\ref{p:b:a:sums:1}}
\hypertarget{sp:b:a:sums:1}
So you sum your sequence using Result~\ref{r:b:a:sums:1}, which gets you a
closed form. 

Then you're left having to prove that $a+1$ and $2n+a$ cannot both be powers of
2. Playing around with special cases tells you at least one is odd, and from
there it's easy to finish.
\begin{proof}
  From Result~\ref{r:b:a:sums:1}, we have that the given sum is equal to
  \[\frac{(a+1)(2n+a)}2.\] So assume, for contradiction, that this is a power
  of 2. Then we have that twice the sum is also a power of 2, so $(a+1)(2n+a)$
  is a power of 2. Since $a+1$ and $2n+a$ both divide powers of 2, they must
  each be a power of 2. Since $a$ and $n$ are positive integers, both $a+1$ and
  $2n+a$ are at least 2 so they're even. So their difference, $2n-1$, must also
  be even, which is a contradiction.
\end{proof}
