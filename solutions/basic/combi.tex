\section{Combinatorics}
\paragraph*{Result~\ref{r:b:c:am:1}}
\hypertarget{sr:b:c:am:1}
Not much to say here: just apply the multiplication principle.
\begin{proof}
  There are $n$ ways to choose the first thing, $n-1$ ways to choose the second
  (since the first has been chosen), $n-2$ to choose the third, and so on up to
  $n-k+1$ for the $k$th. Thus the total number of ways is
  \[n(n-1)\cdots(n-k+1),\] as claimed.
\end{proof}
\paragraph*{Problem~\ref{p:b:c:am:1}}
\hypertarget{sp:b:c:am:1}
So you try some small cases because of course you do, and you notice that every
2 goals apart from the first 2, the number of ways is multiplied by 3. With the
help of like, a tree diagram, it's easy to see that after an odd number of goals
there are 3 ways of scoring the next 2. The rest is easy:

Answer: $972$.
\begin{proof}
  There are 2 ways for the first goal to be scored.

  Assume an odd number of goals has been scored. Then the difference between the
  two teams' goals is 1, so we can WLOG team A has one more goal than team B.
  Then the next two goals can be AB, BA, BB so there are 3 ways for the next two
  goals to be scored.

  Finally, there are 2 ways for the last 2 goals to be scored.

  Thus, the total number of ways for 12 goals to be scored is $2\times 3^5\times
  2=972$.
\end{proof}
\paragraph*{Result~\ref{r:b:c:c:1}}
\hypertarget{sr:b:c:c:1}
The key to proving this is to notice that we have already proven
Result~\ref{r:b:c:am:1} where order does matter. To go from order mattering to
order not mattering, we need to find out how much we've overcounted by: that is,
how many permutations, where order matters, go to the same combination, where
order doesn't.
\begin{proof}
  By Result~\ref{r:b:c:am:1}, there are $\frac{n!}{k!}$ ways of choosing $k$
  things from $n$, where order matters.

  Let there be $x$ ways of choosing $k$ things from $n$, where order doesn't
  matter. Then for each such way, each permutation of those $k$ things is
  counted in our $\frac{n!}{(n-k)!}$ from above. Applying
  Result~\ref{r:b:c:am:1} again, we have that $k!\times x=\frac{n!}{(n-k)!}$,
  which rearranges to the claimed formula.
\end{proof}
\paragraph*{Problem~\ref{p:b:c:c:1}}
\hypertarget{sp:b:c:c:1}
So upon reading the problem we can let our palindromes be $\overline{xyx}$ and
$\overline{ztz}$. Clearly, $x>z$, after which it's clear why $y\ge t$.

So by the multiplication principle, it suffices to count two things:
\begin{itemize}
  \item The number of ways of choosing digits $x$ and $z$ with $x>z>0$
  \item The number of ways of choosing digits $y$ and $t$ with $y\ge t$.
\end{itemize}
Both these things are quite easy to count using this section's formula.

Answer: 1980.
\begin{proof}
  Let the palindromes be $\overline{xyx}$ and $\overline{ztz}$. Since their
  difference is 3 digits and $a>b$, we have $x>z$. 

  Since the second digit of our subtraction can't carry, we must have $y\ge t$.
  So it's enough to count the number of ways we can choose $x$ and $z$, and
  separately the number of ways we can choose $y$ and $t$.

  Since $x$ and $z$ are distinct digits from 1 to 9, the number of choosing them
  is $\binom92=36$. Since $y$ and $t$ are not-necessarily-distinct digits from
  0 to 9, the number of ways of choosing them is $\binom{10}2+10=55$. Thus, the
  answer is $36\times 55=1980$.
\end{proof}
